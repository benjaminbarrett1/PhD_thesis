\chapter{Cut points and cut pairs}\label{chapter:detecting_cut_pairs}

Our goal in this chapter is to relate the existence of cut points and cut pairs in the boundary of a hyperbolic group (or a relatively hyperbolic group with virtually cyclic peripheral subgroups) to the geometry of some large ball in the Cayley graph or cusped space associated go that group.
The novel material in this chapter is taken from~\cite[Section 1--2]{barrett18}.
Refer to Chapter~\ref{chapter:hyperbolic_groups} for definitions and key properties of the objects used in this chapter.

We begin the chapter by relating the connectedness of the complement of a pair of points (or a single point) in the boundary to the connectedness of a cylindrical region around a geodesic connecting those points (respectively a geodesic ray ending at that point.)
Importantly, this result is quantitative: we give explicit bounds on the size of the cylinder.
This is the goal of Section~\ref{section:geodesics}.

Then, in Section~\ref{section:geodesics_in_thin_part} we study the geometry of this cylindrical region when the geodesic penetrates deep into a horoball in the cusped space.
Since we assume that the peripheral subgroups are virtually cyclic, the geometry of the cusps is particularly easy to understand.

Finally, in Section~\ref{section:pumping_lemma} we shall apply an argument based on the pumping lemma to show that the existence of a cut point or cut pair in the boundary of a group is determined by the geometry of a finite ball in the cusped space of known size.
Supposing that there is a cut pair in the boundary, we show that the geodesic $\gamma$ connecting the points in a cut pair may be assumed to be periodic and with bounded period: we take a short subsegment $\gamma\restricted{[a, b]}$ of that geodesic such that both the geodesic and the components of the thickened cylinder are identical in small neighbourhoods of $a$ and $b$ and form a new (local) geodesic that also connects the two points in a (possibly different) cut pair by concatenating infinitely many copies of $\gamma\restricted{[a, b]}$. 
A similar method is used in~\cite{cashenmacura11} to control cut pairs in the decomposition space associated to a line pattern in a free group. 
By characterising circles as Peano continua in which every pair of points is a cut pair, we do the same for the property of having circular boundary: we show that the boundary of a relatively hyperbolic group is homeomorphic to a circle if and only if the cusped space contains a periodic local geodesic with bounded period such that the limit set of that geodesic is not a cut pair.
This completes the proof of Theorem~\ref{theorem:intro_topological_properties_features} and is a key step in the proofs of Theorems~\ref{theorem:intro_detecting_topological_properties} and~\ref{theorem:computing_JSJ_decompositions}.

\section{Cylinders, cut points and cut pairs} \label{section:geodesics}

\subsection{Spaces and constants} 

Let $G$ be a group and let $\calH$ be a finite set of virtually cyclic subgroups of $G$.
We work under the assumption that $G$ is hyperbolic relative to $\calH$.
A virtually cyclic subgroup of a hyperbolic group is always quasi-convex, and is almost malnormal if and only if it is maximal among virtually cyclic subgroups of $G$. 
Then applying Theorem~\ref{theorem:quasiconvex_almost_malnormal} gives the following lemma.

\begin{lemma}\label{lem:vcycperipheral} 
  Suppose that $G$ is hyperbolic and that $\calH$ is a finite set of virtually cyclic subgroups of $G$.
  Suppose further that each group in $\calH$ is maximal virtually cyclic and no two elements of $\calH$ are conjugate.
  Then $G$ is hyperbolic relative to $\calH$.
\end{lemma}

Let $S$ be a finite generating set for $G$ such that $S \intersection H$ is a generating set for $H$ for each $H \in \calH$.
Then let $X = X(G, \calH, S)$ be the cusped space associated to the triple $(G, \calH, S)$ as in Definition~\ref{definition:cusped_space}.
Note that this is a hyperbolic metric space.
Let $1 \in X$ be the vertex corresponding to the identity element of $G$.

We make the additional assumption that $X$ satisfies a double dagger condition.
Note that by Proposition~\ref{proposition:double_dagger_condition_satisfied} this is guaranteed if the boundary $\boundary(G, \calH)$ is connected and does not contain a cut point, and by Proposition~\ref{proposition:ddag_implies_connected} this implies that $\boundary(G, \calH)$ is connected.
We now fix some constants for the remainder of this chapter.

\begin{enumerate}
  \item Fix $\delta$ such that $X$ is $\delta$-hyperbolic.
  \item Let $C = 3\delta$ as in Lemma~\ref{lemma:close_geodesic_ray}.
  \item Let $D$ be chosen so that for any $(8\delta + 1)$-local-geodesic $\gamma$ and any geodesic $\gamma'$ with the same endpoints in $X \union \boundary X$ as $\gamma$, the Hausdorff distance between $\gamma$ and $\gamma'$ is at most $D$. (See Theorem~\ref{theorem:local_geodesics}.)
  \item Let $M = 6(C + 45\delta) + 2\delta + 3$ and let $n$ be such that $X$ satisfies $\ddag(n)$.
  \item Let $d_\rmv$ be a visual metric on $\boundary G$ and fix $a$, $k_1$ and $k_2$ such that it satisfies the following inequality. (See Proposition~\ref{proposition:visual_metric}.)
    \begin{align}
    k_1 a^{-\gromprod[1]{p}{q}} \leq d_\rmv(p, q) \leq k_2 a^{-\gromprod[1]{p}{q}}.
    \end{align}
    In fact, $a$, $k_1$ and $k_2$ can be taken to be $2^{1/4\delta}$, $3-2\sqrt{2}$ and $1$ respectively.
\end{enumerate}

\subsection{Cylinders in the cusped space}\label{section:cylinders_and_the_boundary} 

We now relate the existence of cut points and cut pairs in the boundary $\boundary(G, \calH)$ to the connectedness of thickened cylinders around local geodesics in $X$.
Let $\gamma$ be an $(8\delta+1)$-local-geodesic in $X$ with either one or both end points in $\boundary X$. 
(In other words, $\gamma$ is either a local geodesic ray or a bi-infinite local geodesic.)
Note that the end points of $\gamma$ at infinity are well defined by Remark~\ref{remark:boundary_through_quasi_geodesics}.
We shall denote by $\Lambda\gamma$ the limit set of $\gamma$ in $\boundary X$, which is either one or two points.
By translating by an element of $G$ we may assume that the point $1 \in X$ is $\gamma(0)$.

We define a subset of $X$, the connectivity of which will be seen to reflect the connectivity of $\boundary X - \Lambda \gamma$. 

\begin{definition}
  For $R \geq 0$ let $N_R(\gamma)$ be the closed $R$-neighbourhood of $\gamma$ and for $0 \leq r \leq R \leq \infty$ let $N_{r, R}(\gamma)$ be $\{x \in X \colon r \leq d(x, \gamma) \leq R\}$. 
  For $K \geq 0$ let $C_K(\gamma)$ be $\{x \in X \colon d(x, \gamma) = K\}$. 
  Finally, for $0 \leq r \leq K \leq R \leq \infty$ let $\Cyl_{r, R, K}(\gamma)$ be the union of those connected components of $N_{r, R}(\gamma)$ that meet $C_K(\gamma)$. 
  Note that $\Cyl_{r, R, K}(\gamma)$ contains $N_{K, R}(\gamma)$.
  See Figure~\ref{figure:cylinder} for an illustration of such a subset of $X$.
\end{definition}

\begin{figure}
  \begin{center}
    \input{detecting_cut_pairs/cylinder.pdf_tex}
  \end{center}
  \caption{
    \label{figure:cylinder}
    The three connected components of $N_{r, \infty}(\gamma)$ are shaded red, blue and green.
    The small green component does not correspond to a component of $\boundary X - \Lambda\gamma$ and is discarded when $N_{r, \infty}(\gamma)$ is replaced by $\Cyl_{r, \infty, K}(\gamma)$.
    The union of the red and blue components is $\Cyl_{r, \infty, K}(\gamma)$.
  }
\end{figure}

\begin{definition}\label{definition:shadows}
  For a component $U$ of $\Cyl_{r, \infty, K}(\gamma)$ define its \emph{shadow} $\shadow U$ to be the set of points $p$ in $\boundary X$ such that for any geodesic ray $\alpha$ from $1$ to $p$, $\alpha(t)$ is in $U$ for $t$ sufficiently large. 
\end{definition}

The condition of Definition~\ref{definition:shadows} that all geodesic rays with a particular end point are eventually contained in a component $U$ of $\Cyl_{r, \infty, K}$ is equivalent to a seemingly weaker condition.

\begin{lemma}\label{lemma:equivalent_shadow_definition}
  Let $\alpha$ be a geodesic ray with $\alpha(0) = 1$.
  Suppose that $t\geq 0$ and $d(\alpha(t), \gamma) \geq 3\delta + \max\{r, D\}$.
  Let $U$ be the component of $N_{r,\infty}(\gamma)$ containing $\alpha(t)$.
  Then $U$ meets $C_K(\gamma)$ for any $K \geq r$ and $\alpha(\infty)$ is contained in $\shadow U$.
\end{lemma}

\begin{proof}
  Let $\alpha'$ be another geodesic ray with $\alpha'(0) = 1$ and $\alpha'(\infty) = \alpha(\infty)$.
  Let $s$ be greater than $t + d_\mathrm{Haus}(\alpha,\alpha')$. 
  Then the triangle $\Delta(1, \alpha(s), \alpha(s'))$, two edges of which are subsegments of $\alpha$ and $\alpha'$, is $\delta$-slim. 
  It follows that $d(\alpha'(t), \alpha) \leq \delta$, and then $d(\alpha'(t), \alpha(t)) \leq 2\delta$, so $d(\alpha'(t), \gamma) \geq \delta + \max\{r, D\}$ and $\alpha'(t) \in U$.

  We now show that $\alpha'(s) \in U$ for $s \geq t$.
  If not then there exists $s \geq t$ such that $d(\alpha(s), \gamma) < r$; let $d(\alpha(s), \gamma(s')) < r$.
  Then the triangle $\Delta(1, \alpha'(s), \gamma(s'))$ is $\delta$-slim, so $\alpha'(t)$ is within a distance $\delta$ of $[1, \gamma(s')] \union [\gamma(s'), \alpha(s)]$.
  But $[1, \gamma(s')]$ is contained in a $D$-neighbourhood of $\gamma$, while $[\gamma(s'), \alpha(s)]$ is contained in an $r$-neighbourhood of $\gamma$, so this is impossible.

  Therefore $\alpha'(s) \in U$ for $s \geq t$ for any geodesic ray $\alpha'$ with $\alpha'(0) = 1$ and $\alpha'(\infty) = \alpha(\infty)$.
  In particular, this is true for $\alpha' = \alpha$ and $\alpha$ diverges arbitrarily far from $\gamma$, so $U$ meet $C_K(\gamma)$.
  Furthermore, $\alpha(\infty) \in \shadow(U)$.
\end{proof}

\begin{lemma}\label{lemma:disjoint_cover_by_shadows} Let $r > D$ and let $K \geq r$. Then $\boundary X$ is covered by shadows:
  \begin{align}
    \bigunion_U\shadow U = \boundary X - \Lambda\gamma,
  \end{align}
  where the union is taken over the set of connected components of\/ $\Cyl_{r, \infty, K}(\gamma)$. 
  Furthermore, $\shadow U \intersection \shadow V = \emptyset$ for distinct components $U$ and $V$ of\/ $\Cyl_{r, \infty, K}(\gamma)$.
\end{lemma}

\begin{proof} 
  The statement that shadows are disjoint is clear from the definition of the shadow. 
  The assumption that $r > D$ ensures that when $U$ is a component of $\Cyl_{r, \infty, K}(\gamma)$, $\shadow U$ does not contain a point in $\Lambda\gamma$, since any geodesic ray from $1$ to a point in $\Lambda\gamma$ is contained in a $D$-neighbourhood of $\gamma$.

  If $p$ is any point in $\boundary X - \Lambda\gamma$ then any geodesic ray $\alpha$ from $1$ to $p$ diverges arbitrarily far from $\gamma$, so $d(\alpha(t), \gamma) \geq 3\delta + \max\{r, D\}$ for some $t \geq 0$. 
  Let $U$ be the component of $N_{r, \infty}(\gamma)$ that contains $\alpha(t)$.
  Then $U$ is a component of $\Cyl_{r, \infty, K}(\gamma)$ and $p \in \shadow U$ by Lemma~\ref{lemma:equivalent_shadow_definition}.
\end{proof}

\begin{lemma}\label{lemma:non_empty_shadows} 
  Let $U$ be a component of $\Cyl_{r, \infty, K}(\gamma)$. Then $\shadow U$ is non-empty as long as $K \geq C + 3\delta + \max\{r, D\}$.
\end{lemma}

\begin{proof} 
  Let $x \in C_K(\gamma) \intersection U$. 
  Then by Lemma~\ref{lemma:close_geodesic_ray} there exists a geodesic ray $\alpha$ with $\alpha(0) = 1$ and $t \geq 0$ such that $d(\alpha(t), x) \leq C$. 
  Then $\alpha(t) \in U$ and $d(\alpha(t), \gamma) \geq 3\delta + \max\{r, D\}$, so $\alpha(\infty) \in \shadow U$ by Lemma~\ref{lemma:equivalent_shadow_definition}.
\end{proof}

\begin{lemma}\label{lemma:shadows_clopen} 
  Let $r > D$, let $K \geq r$ and let $U$ be a component of $\Cyl_{r, \infty, K}(\gamma)$.
  Then $\shadow U$ is closed and open in $\boundary X - \Lambda\gamma$.
\end{lemma}

\begin{proof} 
  Let $p$ be a point in $\shadow U$ and let $p = \alpha(\infty)$ where $\alpha$ is a geodesic ray with $\alpha(0) = 1$.
  For $t \geq 0$ let $V_{t}(\alpha)$ be the set of end points of geodesic rays $\beta$ from $1$ such that $d(\beta(t), \alpha(t)) < 2\delta + 1$. 
  From Proposition~\ref{proposition:equivalent_boundary_topologies}, the collection $\{V_t(\alpha) \suchthat t \in \naturals\}$ of such sets forms a fundamental system of neighbourhoods of $p \in \boundary X$. 

  Then there exists $t_0$ such that for $t \geq t_0$, $d(\alpha(t), \gamma) \geq 5\delta + 1 + \max\{r, D\}$, and therefore $\alpha(t_0) \in U$ by Lemma~\ref{lemma:equivalent_shadow_definition}.
  We claim that $V_{t_0}(\alpha) \subset \shadow U$.
  Let $\beta$ be a geodesic ray with $\beta(0) = 1$ and $d(\beta(t_0), \alpha(t_0)) \leq 2\delta + 1$.
  Then $\beta(t_0) \in U$ and $d(\beta(t_0), \gamma) \geq 3\delta + \max\{r, D\}$, so $\beta(\infty) \in \shadow U$ by Lemma~\ref{lemma:equivalent_shadow_definition}.

  Since $p$ was arbitrary in $\shadow U$, it follows that $\shadow U$ is open. 
  As $U$ ranges over the connected components of $\Cyl_{r, \infty, K}(\gamma)$, $\shadow U$ ranges over a cover of $\boundary X - \Lambda\gamma$ by disjoint open subsets by Lemma~\ref{lemma:disjoint_cover_by_shadows} since $r > D$, so each is also closed.
\end{proof}

To complete our proof of the fact that connectedness of $\boundary X - \Lambda\gamma$ is equivalent to connectedness of $\Cyl_{r, \infty, K}(\gamma)$ for appropriate values of $r$ and $K$ we prove the following proposition, in which we show that the double dagger condition can be used to push paths in a component $U$ of $\Cyl_{r, \infty, K}(\gamma)$ onto $\shadow U$.
The proof of this Proposition is similar to on the proof of~\cite[Proposition 3.2]{bestvinamess91}.

\begin{proposition}\label{proposition:shadows_connected} 
  Assume that $r$ satisfies the following inequality.
  \begin{align}
    r > 2\log_a\left(\frac{k_2}{k_1}\frac{n-1}{1-a^{-1}}\right) + M + 12\delta + D.
  \end{align}
  If $U$ is a connected component of $\Cyl_{r, \infty, K}(\gamma)$ then $\shadow U$ is contained in one connected component of $\boundary X - \Lambda \gamma$.
\end{proposition}

\begin{proof} 
  Let $p$ and $q$ be points in $\shadow U$ and let $\alpha_1$ and $\alpha_2$ be geodesic rays from $1$ to $p$ and $q$ respectively. 
  Then there exist $t_1$ and $t_2$ such that $\alpha_1(t_1)$ and $\alpha_2(t_2)$ are in $U$. 
  Let $\phi\colon [0,\ell] \to X$ be a path in $U$ parametrised by arc length connecting the two points $\alpha_1(t_1)$ and $\alpha_2(t_2)$.

  For each integer $i$ in $[0,\ell]$ let $z_i$ be a point within a distance $C$ of $\phi(i)$ so that there is a geodesic ray $\beta_i$ from $1$ with $\beta_i(m_i) = z_i$. 
  We can assume that $\beta_0 = \alpha_1$ and $\beta_\ell = \alpha_2$.  
  Then, following the argument of~\cite[Proposition 3.2]{bestvinamess91}, we show that $\beta_i(\infty)$ and $\beta_{i+1}(\infty)$ can be connected in $\boundary X - \Lambda\gamma$ for each $i$. 
  For notational convenience we prove it for $i=0$.

  Note first that $d(\beta_0(m_0), \beta_1(m_1)) \leq 2C + 1$, which implies that $\mod{m_0 - m_1} \leq 2C+1$, and therefore $d(\beta_0(m_0), \beta_1(m_0)) \leq 4C + 2 \leq M$.
  Then, using the condition $\ddag(n)$, choose a sequence of points $w_0 = \beta_0(m_0),w_1,\dotsc,w_n=\beta_1(m_0)$ such that $d(w_j,w_{j+1}) \leq 1$ and $d(1, w_j) \geq m_0 - C - 45\delta$ for each $j$.
  Choose geodesic rays $\beta_{1/n},\beta_{2/n},\dotsc,\beta_{n-1/n}$ such that $d(w_j, \beta_{j/n}) \leq C$.

  We claim that $d(\beta_{j/n}(m_0 + 1), \beta_{j+1/n}(m_0 + 1)) \leq M$.
  To see this, choose $s_j$ such that $d(\beta_{j/n}(s_j), w_j) \leq C$, so $d(\beta_{j/n}(s_j), \beta_{j+1/n}(s_{j+1})) \leq 2C + 1$.
  If $s_j \geq m_0 + 1$ then, since the triangle $\Delta(1, \beta_{j/n}(s_j), \beta_{j+1/n}(s_{j+1}))$ is $\delta$-slim, $d(\beta_{j/n}(m_0 + 1), \beta_{j+1/n}) \leq \delta + 2C + 1$, so $d(\beta_{j/n}(m_0 + 1), \beta_{j+1/n}(m_0 +1)) \leq 2(\delta + 2C + 1) \leq M$.
  Alternatively, if $s_j \leq m_0 + 1$, then $\mod{s_j - (m_0 + 1)} \leq 2C + 45\delta + 1$, so by the triangle inequality $d(\beta_{j/n}(m_0 + 1), \beta_{j+1/n}(m_0 + 1)) \leq 2C + 1 + 2(2C + 45\delta + 1) \leq M$.

  Now proceed by induction: using the condition $\ddag(n)$ as above, define geodesic rays $\beta_t$ for each $n$-adic rational $t$ in $[0, 1]$ inductively on the power $k$ of the denominator of $t$ to satisfy the following inequality for each $j$ with $0 \leq j \leq n^k-1$.
  \begin{align} 
    d\left(\beta_{j/n^k}(m_i + k), \beta_{j+1/n^k}(m_i + k)\right) \leq M.
  \end{align}

  The triangle inequality gives the following lower bound on the Gromov product of these points.
  \begin{align}
    \gromprod[1]{\beta_{j/n^k}(\infty)}{\beta_{j+1/n^k}(\infty)} & \geq \liminf_{n_1,n_2}\gromprod[1]{\beta_{j/n^k}(n_1)}{\beta_{j+1/n^k}(n_2)}\\
                                                                 & \geq \gromprod[1]{\beta_{j/n^k}(m_0 + k)}{\beta_{j+1/n^k}(m_0 + k)}\\
                                                                 & = m_0 + k - M/2
  \end{align}
  Recall that we defined a visual metric $d_\rmv$ on $\boundary X$ with base point $1$, visual parameter $a$ and multiplicative constants $k_1$ and $k_2$. 
  We obtain:
  \begin{align}\label{align:visual_metric_inequality}
    d_\rmv\left(\beta_{j/n^k}(\infty), \beta_{j+1/n^k}(\infty)\right) \leq k_2a^{-m_0 - k + M/2}.
  \end{align}

  Inductively applying the triangle inequality, we arrive at the following inequality for each $n$-adic rational $t \in [0,1]$.
  \begin{align}
    d_\rmv\left(\beta_0(\infty), \beta_t(\infty)\right) \leq \frac{k_2(n-1)a^{-m_0 + M/2}}{1-a^{-1}}
  \end{align}
  Define a path $\psi \colon [0,1] \to \boundary X$ with $\psi(t) = \beta_t(\infty)$ for each $n$-adic rational $t$ in $[0, 1]$. 

  This extends continuously to a path from $\beta_0(\infty)$ to $\beta_1(\infty)$ by the uniform continuity of the map $t \to \beta_t(\gamma)$ defined on the $n$-adic rationals, which is established by applying the triangle inequality and Equation~\ref{align:visual_metric_inequality} repeatedly.
  This path is contained in the ball of radius $k_2(n-1)a^{-m_0 + M/2}/(1-a^{-1})$ around $\beta_0(\infty)$.

  We now bound below the distance $d_\rmv(\beta_0(\infty), \Lambda\gamma)$. 
  Let $\gamma'$ be a geodesic ray from $1$ to $\gamma(\infty)$, so the Hausdorff distance between $\gamma$ and $\gamma'$ is at most $D$.  
  By Lemma~\ref{lemma:gromov_product_inequality}, we have the following inequality.
  \begin{align}
    \gromprod[1]{\beta_0(\infty)}{\gamma'(\infty)} \leq \liminf_{n_1,n_2}\gromprod[1]{\beta_0(n_1)}{\gamma'(n_2)} + 2\delta.
  \end{align}
  Let $n_1$ and $n_2$ each be at least $m_0$. 
  Certainly $d(\beta_0(m_0), \gamma') > \delta$ since $r > \delta + D$, so there exists a point $p$ on $[\beta_0(n_1), \gamma'(n_2)]$ within a distance $\delta$ of $\beta_0(m_0)$. 
  In fact, $d(\beta_0(m_0), \gamma) > 2\delta + D$, so $d(\beta_0, \gamma'(m_0)) > 2\delta$. 
  Therefore there exists a point $q$ on $[\beta_0(n_1), \gamma'(n_2)]$ within a distance $\delta$ of $\gamma'(m_0)$. 

  Suppose that $q$ is closer to $\beta_0(n_1)$ than $p$. 
  Then by considering the geodesic triangle with vertices $\beta_0(n_1)$, $\beta_0(m_0)$ and $p$ we see that $q$ is within distance $2\delta$ of $\beta_0$. 
  Therefore the distance from $\gamma'(m_0)$ to $\beta_0(m_0)$ is at most $6\delta$. 
  But we assumed that $r > 6\delta + D$, which gives a contradiction. 
  This implies that $d(\beta_0(n_1), \gamma'(n_2))$ is equal to the sum of the distances $d(\beta_0(n_1), p)$, $d(p, q)$ and $d(q, \gamma'(n_2))$.

  Then we have the following inequality.
  \begin{align} \gromprod[1]{\beta_0(n_1)}{\gamma'(n_2)} - \gromprod[1]{\beta_0(m_0)
      }{\gamma'(m_0)} 
      & = d(\beta_0(n_1), \beta_0(m_0)) - d(\beta_0(n_1), p) \\
      & + d(\beta_0(m_0), \gamma'(m_0)) - d(p, q) \\
      & + d(\gamma'(m_0), \gamma'(n_2)) - d(q, \gamma'(n_2))\\
      & \leq \delta + 2\delta + \delta = 4\delta
  \end{align}

  This implies a lower bound on the distance from $\beta_0(\infty)$ to $\gamma(\infty)$ with respect to the visual metric.
  \begin{align}
    d_\rmv(\beta_0(\infty), \gamma(\infty)) \geq k_1a^{-m_0 + (r-D)/2 - 6\delta}.
  \end{align}
  In the case that $\gamma$ is bi-infinite, $d_\rmv(\beta_0(\infty), \gamma(-\infty))$ similarly satisfies the same bound.  
  Therefore, by the assumption on $r$ the path constructed from $\beta_0(\infty)$ to $\beta_1(\infty)$ avoids $\Lambda\gamma$.
\end{proof}

This completes the proof of the assertion that the connectedness of $\Cyl_{r, \infty, K}(\gamma)$ determines the connectedness of $\boundary X - \Lambda \gamma$.
We now show that this connectedness can be detected without looking too far from $\gamma$: we show that for $R$ at least some predetermined constant, $\Cyl_{r, R, K}(\gamma)$ has the same number of components as $\Cyl_{r, \infty, K}(\gamma)$.

\begin{proposition}\label{proposition:replace_infinity_by_R} 
  Let $R \geq 2\delta + 2D + \max\{r + 4\delta + 1, K\}$.
  Then the inclusion map $\Cyl_{r, R, K}(\gamma) \hookrightarrow \Cyl_{r, \infty, K}(\gamma)$ induces a bijection between the sets of connected components of those subspaces of $X$.
\end{proposition}

\begin{proof} 
  Surjectivity is clearly guaranteed by the fact that $R \geq K$. 
  For injectivity, let $x$ and $y$ be points in $C_K(\gamma)$ that lie in the same connected component of $\Cyl_{r, \infty, K}(\gamma)$. 
  We show that the shortest path from $x$ to $y$ in $N_{r, \infty}(\gamma)$ stays within a distance $R$ of $\gamma$. 
  This argument is illustrated in Figure~\ref{figure:replace_infinity_by_R}.
  Let $\phi\colon [0,\ell] \to X$ be such a shortest path parametrised by arc length. 
  Suppose that $d(\phi(s),\gamma) > R$. 
  Let $[t_0,t_1]$ be a maximal subinterval of $[0, \ell]$ containing $s$ such that $d(\phi(t),\gamma) \geq r + 4\delta + 1$ for $t \in [t_0,t_1]$. 

  Then for $t \in [t_0,t_1]$, $\phi\restricted{[t-4\delta-1,t+4\delta+1] \intersection[t_0, t_1]}$ has image in $N_{r+4\delta+1, \infty}(\gamma)$.  
  Therefore any geodesic segment from $\phi(\min\{t-4\delta-1, t_0\})$ to $\phi(\max\{t+4\delta+1, t_1\})$ is contained in $N_{r, \infty}(\gamma)$, so by minimality of the length of $\phi$, $\phi\restricted{[t-4\delta-1,t+4\delta+1] \intersection [t_0, t_1]}$ is a geodesic. 
  This means that $\phi\restricted{[t_0,t_1]}$ is an $(8\delta+2)$-local geodesic. 
  Therefore by Theorem~\ref{theorem:local_geodesics} it is contained in a $D$-neighbourhood of any geodesic from $\phi(t_0)$ to $\phi(t_1)$.

  By maximality of $[t_0,t_1]$, either $d(\phi(t_0),\gamma) = r + 4\delta+1$ or $t_0=0$, so certainly $d(\phi(t_0),\gamma) \leq \max\{r+4\delta+1,K\}$, and similarly $d(\phi(t_1),\gamma)$ satisfies the same inequality. 
  By $\delta$-hyperbolicity applied to the geodesic quadrilateral with vertices $\phi(t_0)$, $\phi(t_1)$ and the points $\gamma(s_0)$ and $\gamma(s_1)$ on $\gamma$ minimising the distances to $\phi(t_0)$ and $\phi(t_1)$, any geodesic from $\phi(t_0)$ to $\phi(t_1)$ is contained in a $2\delta + \max\{r + 4\delta + 1, K\}$ neighbourhood of a geodesic from $\gamma(s_0)$ to $\gamma(s_1)$, so is a subset of $N_{2\delta+\max\{r+4\delta+1, K\} + D}(\gamma)$. 
  Hence $d(\phi(s),\gamma) \leq 2\delta + \max\{r+4\delta+2,K\} + 2D$, which is a contradiction. 
\end{proof}

\begin{figure}
  \begin{center}
    \input{detecting_cut_pairs/local_geodesic.pdf_tex}
  \end{center}
  \caption{
    \label{figure:replace_infinity_by_R}
    The configuration of points and paths in the proof of Proposition~\ref{proposition:replace_infinity_by_R}.
    We show that the shortest path from $x$ to $y$ that avoids $N_r(\gamma)$ is contained in $N_R(\gamma)$, so the cusped space cannot contain a large ``hole'' obstructing such paths as shown in the diagram.
  }
\end{figure}

From the results of this section we draw the following conclusion, which was main goal of this section.

\begin{proposition}\label{proposition:summary_of_cylinder_results} 
  The map that sends a component $U$ of $\Cyl_{r, R, K}(\gamma)$ to the shadow (with respect to the base point $1 \in \gamma$) of the component of $\Cyl_{r, \infty, K}(\gamma)$ containing $U$ is a well defined bijection between the set of connected components of $\Cyl_{r, R, K}(\gamma)$ and the set of connected components of $\boundary X - \Lambda\gamma$ as long as $r$, $R$ and $K$ are taken to simultaneously satisfy the conditions of Lemmas~\ref{lemma:disjoint_cover_by_shadows}, \ref{lemma:non_empty_shadows}, \ref{lemma:shadows_clopen} and Propositions \ref{proposition:shadows_connected} and~\ref{proposition:replace_infinity_by_R}.\qed\end{proposition}

\begin{remark}\label{rem:rKRcomputable} 
  The conditions on $r$, $R$ and $K$ depend only on $\delta$, $n$ and $\lambda$. 
  Suitable values for $r$, $R$ and $K$ can be computed from these data.
\end{remark}

We end this section with the following proposition, which shows that it is possible to detect which components of $N_{r, R}(\gamma)$ meet $C_K(\gamma)$ just by looking at a ball of a bounded size in $X$.
Therefore it is possible to construct bounded subsets of $A_{r, R, K}(\gamma)$ algorithmically.

\begin{proposition}\label{proposition:cylinder_disconnected_locally} 
  Suppose that $\gamma$ is a bi-infinite $\lambda$-quasi-geodesic and that neither point in $\Lambda\gamma$ is a cut point. 
  Let $r$ and $R$ be chosen to satisfy Propositions~\ref{proposition:shadows_connected} and~\ref{proposition:replace_infinity_by_R}. 
  Let $T$ be at least $\log_a(2k_1/k_2) + 3D + 2\delta + K$. 
  Then every component of\/ $\Cyl_{r, R, K}(\gamma)$ meets $C_K(\gamma) \intersection \Ball_{T}(\gamma(t))$ for any $t$ such that $\gamma(t)$ is in $X_0$.  
\end{proposition}

\begin{proof} 
  Without loss of generality assume that $\gamma(t) = 1$.
  Let $U$ be a component of $\Cyl_{r, R, K}(\gamma)$ and let $U'$ be the component of $\Cyl_{r, \infty, K}(\gamma)$ containing $U$.
  Then it is sufficient to show that $U'$ meets $C_K(\gamma) \intersection \Ball_{T}(1)$ since $U' \intersection C_K(\gamma) = U \intersection C_K(\gamma)$ by Proposition~\ref{proposition:replace_infinity_by_R}. 

  Suppose that $U'$ does does not meet $C_K(\gamma) \intersection \Ball_{T}(1)$.
  We show then that $\shadow U$ must then be clustered around the two points in $\Lambda\gamma$.
  Let $p$ be a point in $\shadow U$ and let $\alpha$ be a geodesic ray from $1$ to $p$.
  Then $\alpha(s) \in U \intersection C_K(\gamma)$ for some $s$; by assumption $d(\alpha(s), v) \geq T$.

  Let $\gamma'$ be a geodesic connecting the points of $\Lambda\gamma$ so that the Hausdorff distance between $\gamma$ and $\gamma'$ is at most $D$.
  Parametrise $\gamma'$ so that $d(\gamma'(0), 1) \leq D$.
  Then $d(\alpha(s), \gamma') \leq K + D$; let $d(\alpha(s), \gamma'(s')) \leq K + D$.
  This implies that $d(\gamma'(s'), 1)\geq T - D - K$.
  We therefore have the following inequality.
  \begin{align}
    \gromprod[1]{\alpha(s)}{\gamma'(s')} \geq T - D - K.
  \end{align}
  Assume that $s'\geq0$; this implies that
  \begin{align}
    \gromprod[1]{p}{\gamma'(\infty)} & \geq \liminf_{m, n \to \infty}\gromprod[1]{\alpha(m)}{\gamma'(n)} \\
                                     & \geq \gromprod[1]{\alpha(s)}{\gamma'(s')} - D \\
                                     & \geq T - 2D - K.
  \end{align}
  So $d_\rmv(p, \gamma'(\infty)) \leq k_2a^{-T + 2D + K}$. 
  Similarly, if $s' \leq 0$, $d_\rmv(p, \gamma'(-\infty)) \leq k_2a^{-T + 2D + K}$. 
  Therefore $\shadow U'$ is contained in a $k_2a^{-T + 2D + K}$ neighbourhood of $\Lambda\gamma$.  
  Also, for any $s$, the geodesic from $\gamma(t+s)$ to $\gamma(t-s)$ passes within a distance $D$ of $\gamma(t)$, so $\gromprod[1]{\gamma(t+s)}{\gamma(t-s)} \leq D$. 
  Then $\gromprod[1]{\gamma(\infty)}{\gamma(-\infty)} \leq 2\delta + D$, and so $d_\rmv(\gamma(\infty), \gamma(-\infty)) \geq k_1a^{-(2\delta + D)}$. 
  It follows by the inequality satisfied by $T$ that the closed balls of radius $k_2a^{-T + 2D + K}$ around $\gamma(\infty)$ and $\gamma(-\infty)$ are disjoint. 
  By Proposition~\ref{proposition:shadows_connected} $\shadow U'$ is connected, so is contained in one of these two balls, say in the ball around $\gamma(\infty)$.  
  But then $\shadow U$ is a non-empty proper subset of $\boundary X - \{\gamma(\infty)\}$ that is closed and open, so $\gamma(\infty)$ is a cut point, which is a contradiction.
\end{proof}

\section{Cylinders in the thin part of the cusped space}\label{section:geodesics_in_thin_part}

As in the previous section, let $G$ be a hyperbolic group and let $\calH$ be a finite set of pairwise non-conjugate maximal virtually cyclic subgroups of $G$, so that $G$ is hyperbolic relative to $\calH$. 
Let $S$ be a generating set for $G$ such that $S \intersection H$ generates $H$ for each $H$ in $\calH$. 
Let $X$ be the cusped space associated to $(G, \calH, S)$.
Let $\gamma$ be a $(8\delta + 1)$-local-geodesic in $X$ with one or both end points in $\boundary X$.

In this section we study the connectedness of the intersection of $\Cyl_{r, R, K}(\gamma)$ with the thin part of $X$.
Because of our assumption that the peripheral subgroups are virtually cyclic, the geometry of the cusps of $X$ is quite simple. 
We begin to restricting to the case in which $\gamma$ is a vertical geodesic ray contained in a horoball corresponding to a subgroup $H \in \calH$.
Then the vertex set of $\Hor(\Cay(H, S\intersection H))$ can be identified with $H \times \integers_{\geq 0}$ and for any $k$ there is an inclusion of the Cayley graph $\Cay(H, S \intersection H) \hookrightarrow h^{-1}(k) \intersection \Hor(\Cay(H, S \intersection H))$ mapping a vertex $g \in \Cay(H, S)$ to $(g, k)$. 
For $k = 0$ this inclusion is an isomorphism of graphs. 

Let $d_H$ be the path metric in $\Cay(H, S \intersection H)$ and let $\Cay(H, S \intersection H)$ be $\delta_H$-hyperbolic with respect to this metric. 
Let $\alpha$ be a bi-infinite geodesic in $\Cay(H, S)$ with respect to $d_H$. 
Then any point in $\Cay(H, S)$ is within a distance of at most $2\delta_H + 1$ of $\alpha$.

Let $\gamma \colon [0, \infty) \to X$ be a vertical geodesic ray with $\gamma(0) = (\alpha(0), 0)$. 
The $\delta$-thin part of the horoball is convex by~\cite[Lemma 3.26]{grovesmanning08}, so for any point $x$ in $\Hor(\Gamma(H, S \intersection H))$ with $h(x) \geq \delta$, the distance from $x$ to $\gamma$ is minimised by a geodesic comprising a vertical segment followed by a single horizontal edge.
Then for any $k \geq \delta$, the vertex set of $h^{-1}(k) \intersection N_{r, R}(\gamma)$ is
\begin{align}
  \{g \in H \colon 2^{k + r - 2} < d_H(g, \alpha(0)) \leq 2^{k + R - 1}\} \times \{k\}.
\end{align} 
We will denote by $Y_k$ the set $h^{-1}(k) \intersection N_{r, R}(\gamma)$. 

Assume now that $k \geq \max\{\log_2(2\delta_H + 1),\delta\}$. Then every vertex in $h^{-1}(k)$, and therefore every vertex in $Y_k$, is adjacent to a vertex in $\alpha \times \{k\}$. 
Therefore $Y_k$ contains connected components $Y_k^+$ and $Y_k^-$ with
\begin{align}
  \alpha\restricted{[2^{k+r-1}, 2^{k+R-1}]} \times \{k\} & \subset Y_k^+,\\
  \alpha\restricted{[-2^{k+R-1}, -2^{k+r-1}]} \times \{k\} & \subset Y_k^-,\\
\end{align}
and each of these components meets $C_K(\gamma)$. 
Therefore each of the sets $Y_k^\pm$ is a subset of a component of $\Cyl_{r, R, K}(\gamma)$. 
Any vertex in the complement of these two components of $Y_k$ is adjacent to a point in $\{(\alpha(\pm 2^{k+r-2}), k)\}$, so is contained in the following set.
\begin{align}
  \{g \in H \colon d_H(g, \alpha(0)) \leq 2^{k + r - 1}\} \times \{k\}.
\end{align}
Therefore only those vertices of $Y_k$ that are in $Y_k^+ \union Y_k^-$ are adjacent in $X$ to vertices of $Y_{k+1}$. 
Furthermore, $Y_k^+$ is adjacent to $Y_{k+1}^+$ and not to $Y_{k+1}^-$ and likewise for $Y_k^-$. 
Finally, any vertex that is in $Y_{k+1}$ but not in $Y_{k+1}^{+} \union Y_{k+1}^{-}$ is adjacent to a vertex in $Y_{k}^{+} \union Y_{k}^{-}$.

Therefore, if $k \geq \max\{\log_2(2\delta_H + 1),\delta\}$ then $\Cyl_{r, R, K}(\gamma) \intersection h^{-1}[k, \infty) \intersection \Hor(\Cay(H, S \intersection H))$ contains two unbounded components $Y_{\geq k}^+$ and $Y_{\geq k}^-$ containing $\union_{l \geq k} Y_l^+$ and $\union_{l \geq k} Y_l^-$ respectively and the complement of these two components is contained in
\begin{align}
  \{g \in H \colon d_H(g, \alpha(0)) \leq 2^{k + r}\} \times \{k\}.
\end{align}

To make precise the consequences of this description of $\Cyl_{r, R, K}(\gamma)$,
we make the following definition, now allowing $\gamma$ to be any geodesic segment, geodesic ray or bi-infinite geodesic in $X$.

\begin{definition}\label{definition:loose_ends}
  Let $\gamma$ be a geodesic segment, geodesic ray or bi-infinite geodesic in $X$.
  We call an end point of $\gamma$ a \emph{loose end} if it is in $X$ (as opposed to $\boundary X$), and this end point is a local maximum of the function $h \composed \gamma$.
  (In other words, if the end point is at the start of $\gamma$ then $\gamma$ is descending vertically at that point, while if the end point is at the end of $\gamma$ then $\gamma$ is ascending vertically at that point.)
  Then let $\hat\gamma$ be the bi-infinite path obtained by concatenating $\gamma$ with geodesic rays at any loose ends of $\gamma$.
  Note that this can be done in a unique way, since the loose ends of $\gamma$ have positive height.
\end{definition}

Note that if the height of the loose ends of $\gamma$ is at least $8\delta + 1$, $\hat\gamma$ is a $(8\delta + 1)$-local-geodesic.

With $\gamma$ as in Definition~\ref{definition:loose_ends}, let $\Cyl_{r, R, K}'(\gamma)$ be the space obtained by truncating $\Cyl_{r, R, K}(\hat\gamma)$ at the loose ends of $\gamma$:
\begin{align}
  \Cyl_{r, R, K}(\hat\gamma) - \left(h^{-1}[k, \infty) \intersection N_R (\hat\gamma - \gamma)\right).
\end{align} 
This space is shown in Figure~\ref{figure:geodesic_in_thin_part}.

\begin{figure}
  \begin{center}
    \input{detecting_cut_pairs/geodesic_in_thin_part.pdf_tex}
  \end{center}
  \caption{
    \label{figure:geodesic_in_thin_part}
    The union of the red and blue components of $\Cyl'_{r, R, K}(\gamma)$, where $\gamma$ is a geodesic with a loose end at height $k$. 
    Note that extending $\gamma$ by a geodesic ray (to obtain $\hat\gamma$) does not change the number of connected components of the cylinder.
  }
\end{figure}

The results of this section together give the following lemma, which we shall
use to control the depth to which geodesics in $X$ connecting cut pairs in
$\boundary X$ penetrate into the thin part of $X$.

\begin{lemma}\label{lemma:geodesics_in_horoballs} 
  Let $\gamma$ be a geodesic segment, geodesic ray or bi-infinite geodesic in $X$ such that any ends of $\gamma$ not in $\boundary X$ are loose ends of $\gamma$ and have height at least $\min\{R, \log_2(2\deltaH + 1), 8\delta + 1\}$. 
  Let $\deltaH$ be such that each $H \in \calH$ is $\deltaH$-hyperbolic with respect to the generating set $H \intersection S$.  
  Then the inclusion $\Cyl'_{r, R, K}(\gamma) \hookrightarrow \Cyl_{r, R, K}(\hat\gamma)$ induces a bijection between the sets of connected components of those spaces.\qed
\end{lemma}

\section{The geometry of finite balls in the cusped space}\label{section:pumping_lemma}

In this section we apply the results of Sections~\ref{section:geodesics} and~\ref{section:geodesics_in_thin_part} to prove some of the results referred to in the introduction.
We show that certain topological properties of the boundary of the pair $(G, \calH)$ are determined by the geometry of a large ball in the cusped space.
All of these results require the assumption that the cusped space satisfies a double dagger condition.

\subsection{Cut points}

The first topological feature we treat is a cut point. 
As discussed in Sections~\ref{section:local_connectedness_of_gromov_boundary} and~\ref{section:bowditch_boundary} there is never a cut point in the boundary in the Gromov boundary of a hyperbolic group, but cut points are possible in the Bowditch boundary of relatively hyperbolic groups.
Note also that we require a double dagger condition, which is not guaranteed if the boundary does contain a cut point.

\begin{proposition}\label{proposition:cut_point_feature} 
  Let $G$ be a group and let $\calH$ be a finite set of virtually cyclic subgroups of $G$ so that $G$ is hyperbolic relative to $\calH$.
  Let $S$ be a finite generating set for $G$ such that $S \intersection H$ generates $H$ for each $H \in \calH$.
  Suppose that the cusped space $X$ associated to the triple $(G, \calH, S)$ is $\delta$-hyperbolic and satisfies $\ddag(n)$.
  Fix $r$, $R$ and $K$ satisfying Proposition~\ref{proposition:summary_of_cylinder_results}.
  Let $\deltaH$ be large enough that $\Cay(H,S\intersection H)$ is $\deltaH$-hyperbolic for each $H \in \calH$ and let $k \geq \log_2(2\deltaH + 1)$.
  Then $\boundary(G, \calH, S)$ contains a cut point if and only if there is a vertical geodesic segment $\gamma \colon [0, k] \to X$ with $\gamma(0) = 1$ such that $\Cyl'_{r, R, K}(\gamma)$ is disconnected.
\end{proposition}
  
\begin{proof} 
  It is shown in~\cite[Theorem 0.2]{bowditch99a} that any cut point in $\boundary(G, \calH)$ must be the limit point of $gHg^{-1}$ for some $H \in \calH$ and $g \in G$. 
  Therefore $\boundary(G, \calH)$ contains a cut point if and only if $\Lambda H$ is a cut point for some $H \in \calH$.  
  For $H \in \calH$ let $\gamma_H$ be the vertical geodesic ray contained in $\Hor(\Cay(H, S \intersection H))$ with $\gamma_H(0)=1$.
  Note that $\gamma_H(\infty) = \Lambda H$.

  By Proposition~\ref{proposition:summary_of_cylinder_results}, $\gamma_H(\infty)$ is a cut point if and only if $\Cyl_{r, R, K}(\gamma_H)$ is disconnected.
  By Lemma~\ref{lemma:geodesics_in_horoballs} $\Cyl_{r, R, K}(\gamma_H)$ is disconnected if and only if $\Cyl_{r, R, K}'(\gamma_H\restricted{[0,k]})$ is disconnected.
\end{proof}

\subsection{Cut pairs}\label{section:detecting_cut_pairs}

We now assume that $\boundary(G,\calH)$ does not contain a cut point and give criteria for the existence of a cut pair in $\boundary(G, \calH)$.
In this situation all results of Section~\ref{section:cylinders_and_the_boundary} are applicable.
We use a pumping lemma argument: we aim to replace an arbitrary geodesic joining the two points in a cut pair in $\boundary X$ with a periodic local geodesic with bounded period that also joins the points of a (possibly different) cut pair.

Before stating the theorem we define some constants. 
Let the cusped space $X$ be $\delta$-hyperbolic and satisfy the double dagger condition $\ddag(n)$.
Fix $\deltaH$ so that $\Cay(H,S\intersection H)$ is $\deltaH$-hyperbolic for each $H \in \calH$.
Take $\lambda = \max\{(12\delta+1)/(4\delta+1), 2\delta\}$ so that any $(8\delta+1)$-local-geodesic is a $\lambda$-quasi-geodesic as in Theorem~\ref{theorem:local_geodesics} and fix $D$ so that any $(8\delta+1)$-local geodesic is contained in a $D$-neighbourhood of a geodesic with the same end points.
Fix $r$, $R$ and $K$ to simultaneously satisfy the conditions of the results of Section~\ref{section:cylinders_and_the_boundary} as summarised in Proposition~\ref{proposition:summary_of_cylinder_results}.
Fix $T$ to satisfy the condition of Proposition~\ref{proposition:cylinder_disconnected_locally}. 
We also make the following definitions.
\begin{align}
     k &= \max\{8\delta + 1, \log_2(2\deltaH + 1), T + R\}.\\
  \rho &= (2R + \lambda + 1)\lambda^2 + \lambda + R.\\
  \eta &= \max\{(8\delta + 1)/2, \lambda(T + K + \lambda), \lambda(R + r + \lambda), \lambda(R + \rho + \lambda), \lambda(2R + \lambda + 1)\}.
\end{align}
To summarise, we take $r \ll K \ll R \ll \rho \ll \eta$ and $k \gg R$.

Let $\Val$ be the maximum valence of any vertex in $X_{k + R}$ and let $V$ be the maximum number of vertices in any ball of radius $\rho$ around any vertex in $X_{k + R}$. 
Then define $N_{\min}$ and $N_{\max} \gg \eta$ as follows.
\begin{align}
  N_\text{min} &= \max\{8\delta+1, \lambda(2R+\lambda+1) + 1\}\\
  N_\text{max} &= N_\text{min} \left(k + R + 1\right) \Val^{2\eta} 2^V + 1
\end{align}

\begin{theorem}\label{theorem:cut_pair_feature} 
  There is a cut pair in $\boundary(G, \calH)$ if and only if and only if $X$ contains one of the following two features:
\begin{enumerate}
  \item A short coarsely separating geodesic segment at shallow depth in $X$: a geodesic segment $\gamma \colon [a - \eta, b + \eta] \to X$ contained in $X_{k + R}$ satisfying the following conditions.
    Such a feature is shown in Figure~\ref{figure:pumping_lemma}.
  \begin{enumerate}
    \item\label{cutpairshort} The segment is short: $N_\text{min} \leq b - a \leq N_{max}$.
    \item\label{cutpairmatch} The ends of the segment match: $h(\gamma(a)) = h(\gamma(b))$, so there exists $g \in G$ such that $\gamma(b) = g \cdot \gamma(a)$, and $\gamma\restricted{[b - \eta, b + \eta]} = g\cdot \gamma\restricted{[a - \eta, a + \eta]}$.
    \item\label{cutpairnontrivial} The segment coarsely separates $X$: there is a partition $\calP$ of the vertices of $N_{r, R}(\gamma) \intersection N_R(\gamma\restricted{[a,b]})$ into two subsets such that adjacent vertices lie in the same subset and each of the sets meets $C_K(\gamma) \intersection \Ball_{T}(\gamma(c))$ for some $c \in [a, b]$ such that $\gamma(c) \in X_0$.
    \item\label{cutpairpartition} The ends of the partition match: the partition on the vertices of $N_{r, R}(\gamma) \intersection \Ball_{{\rho}}(\gamma(b))$ induced by the restriction of $\calP$ to that subset is the same as the translate by $g$ of the partition on the vertices of $N_{r, R}(\gamma) \intersection \Ball_{{\rho}}(\gamma(a))$ obtained by restricting $\calP$. 
      Note that $N_{r, R}(\gamma) \intersection \Ball_{\rho}(\gamma(b))$ is equal to $g \cdot N_{r, R}(\gamma) \intersection \Ball_{\rho}(\gamma(a))$ by Condition~\ref{cutpairmatch}.
  \end{enumerate} 
  \item A short coarsely separating horseshoe-shaped geodesic: a geodesic segment $\gamma \colon [a, b] \to X$ satisfying the following conditions.
  \begin{enumerate}
    \item The segment is short: $b - a \leq N_{max} - 2R + 2\eta$.
    \item Both ends are loose ends: $h(\gamma(a)) = h(\gamma(b)) \geq k$ and $h \composed \gamma$ is decreasing at $a$ and increasing at $b$.
    \item The segment coarsely separates $X$: $\Cyl'_{r, R, K}(\gamma)$ is disconnected.
  \end{enumerate}
\end{enumerate}
\end{theorem}

\begin{figure}
  \begin{center}
    \input{detecting_cut_pairs/pumping_lemma.pdf_tex}
  \end{center}
  \caption{
    \label{figure:pumping_lemma}
    A short coarsely separating geodesic segment in $X$. 
    The partition $\calP$ consists of the red set and the blue set.
    Note that the two ends of the segment, and the partitions around the two ends, match under translation by $g$.
  }
\end{figure}

\begin{proof} 
  First suppose that the first type of feature exists in $X$.
  Define a path $\gamma'$ in $X$ by concatenating translates of $\gamma$:
  \begin{align} 
    \gamma'\left((b - a)m + t\right) = g^m\gamma(a + t)
  \end{align}
  for $m \in \integers$ and $t \in [0, b-a]$. 
  Then $\gamma'$ is an $(8\delta + 1)$-local-geodesic by Condition~\ref{cutpairmatch}.
  (Here we use the condition that $\eta \geq (8\delta + 1)/2$.) 
  It is therefore a $\lambda$-quasi-geodesic by Theorem~\ref{theorem:local_geodesics}. 
  We now show that $\Cyl_{r, R, K}(\gamma')$ is disconnected.

  Note that $N_R(\gamma')$ is a union $\bigcup_{m\in\integers} g^m\cdot N_R(\gamma\restricted{[a, b]})$ of translates of neighbourhoods of $\gamma$.
  Since $\eta \geq \lambda(R + r + \lambda)$, $N_r(\gamma') \intersection N_R(\gamma\restricted{[a, b]})$ is a subset of $N_{r}(\gamma)$, so is equal to $N_r(\gamma) \intersection N_R(\gamma\restricted{[a, b]})$.  
  Therefore $N_{r, R}(\gamma')$ is also a union:
  \begin{align} 
    N_{r, R}(\gamma') = \bigcup_{m \in \integers} g^m\cdot \left(N_{r, R}(\gamma) \intersection N_R\left(\gamma\restricted{ [a,b]}\right)\right).
  \end{align}

  Since $b - a > \lambda(2R+\lambda+1)$, $g^m \cdot N_R(\gamma\restricted{[a, b]})$ and $g^l \cdot N_R(\gamma\restricted{[a, b]})$ contain no adjacent vertices for $\mod{m - l} \geq 2$. 
  Furthermore, if $l = m+1$ then any pair of adjacent vertices in these two sets is contained in $g^m\cdot \Ball_{\rho}(\gamma(b))$ since $\rho \geq (2R + \lambda + 1)\lambda^2 + \lambda + R$.

  For each set $U \in \calP$ define a set $U'$ of vertices of $N_{r, R}(\gamma')$ by letting $u \in U'$ if $g^m u \in U$ for some $m \in \integers$.
  This gives a well defined partition $\calP'$ of the vertices of $N_{r, R}(\gamma')$ such that adjacent vertices lie in the same set by condition \ref{cutpairpartition}. 
  Its restriction to $\Cyl_{r, R, K}(\gamma')$ is non-trivial: $C_K(\gamma')$ contains $C_K(\gamma) \intersection \Ball_{T}(\gamma(c))$ since $\eta \geq \lambda(T + K + \lambda)$ and this set meets both sets in $\calP'$ by condition \ref{cutpairnontrivial}. 
  Therefore $\Cyl_{r, R, K}(\gamma')$ is disconnected and therefore $\Lambda\gamma'$ is a cut pair by the results of Section~\ref{section:geodesics}.

  Now suppose that the second type of feature exists in $X$.
  As in Section~\ref{section:geodesics_in_thin_part}, let $\hat\gamma$ be the $(8\delta + 1)$-local-geodesic obtained by concatenating $\gamma$ with vertical geodesic rays. 
  Then $\Cyl_{r, R, K}(\hat\gamma)$ is disconnected by Lemma~\ref{lemma:geodesics_in_horoballs} and $\Lambda\hat\gamma$ is a cut pair by the results of Section~\ref{section:geodesics}.

  Conversely, suppose that $\boundary X$ does contain a cut pair. 
  Let $\gamma'$ be a geodesic in $X$ such that $\Lambda\gamma'$ is a cut pair. 
  Assume first that some connected component of $\gamma'^{-1} h^{-1}[0, k + R]$ is an interval of length less than $N_{max} + 2R$, say $[a - R, b + R]$ with $h(\gamma'(a - R)) = h(\gamma'(b + R)) = k + R$, so $h(\gamma'(a)) = h(\gamma'(b)) = k$. 
  Shortening $\gamma$ if necessary, we may assume that $h \composed \gamma$ is decreasing at $a$ and increasing at $b$.
  Let $\gamma = \gamma' \restricted{[a, b]}$.  
  Let $c \in [a, b]$ such that $h(\gamma'(c)) = 0$.  
  $\Cyl_{r, R, K}(\gamma')$ is disconnected and each component meets $C_K(\gamma') \intersection \Ball_{T}(\gamma'(c))$ by Proposition~\ref{proposition:cylinder_disconnected_locally}. 
  Since $k \geq T$ this is a subset of $\Cyl'_{r, R, K}(\gamma)$, and $\Cyl'_{r, R, K}(\gamma)$ is  a subset of $\Cyl_{r, R, K}(\gamma')$, so $\Cyl_{r, R, K}'(\gamma)$ is disconnected. 
  Therefore $\gamma'$ is a feature of the second kind described in the proposition.

  On the other hand, suppose that some interval $[-\eta, N_{max} + \eta]$ is a subset of $\gamma'^{-1} h^{-1}[0, k + R]$. 
  Then there exist $a_0 < a_1 < \dots < a_{2^V}$ in $[0, N_{max}]$ such that $h(a_i) = h(a_j)$ for all $i$ and $j$, so $a_i = g_i a_0$ for some $g_i \in G$, such that $\gamma'\restricted{[a_i - \eta, a_i + \eta]} = g_i \cdot \gamma'\restricted{[a_0 - \eta, a_0 + \eta]}$, and such that $a_i - a_{i - 1} \geq N_\text{min}$. 
  Let $\calP'$ be a partition of the vertices of $N_{r, R}(\gamma')$ into two subsets such that adjacent vertices are in the same set and so that both sets meet $C_K(\gamma')$. 
  Such a partition exists by the results of section~\ref{section:cylinders_and_the_boundary} since $\Lambda\gamma'$ is a cut pair. 
  Since $\eta \geq \lambda(R + \rho + \lambda)$, $N_{r, R}(\gamma') \intersection \Ball_{\rho}(\gamma(a_0))$ is equal to $g_i^{-1} N_{r, R}(\gamma') \intersection \Ball_{\rho}(\gamma(a_i))$ for all $i$.  
  This set contains at most $V$ vertices, so there exist $0 \leq i < j \leq 2^V$ such that
  \begin{align}
  g_i^{-1}\calP'\restricted{N_{r, R}(\gamma') \intersection
  \Ball_{\rho}(\gamma(a_i))} = g_j^{-1}\calP'\restricted{N_{r, R}(\gamma')
  \intersection \Ball_{\rho}(\gamma(a_j))}.
  \end{align}

  Let $a = a_i$ and $b = a_j$ and let $\gamma = \gamma'\restricted{[a - \eta, b + \eta]}$. 
  We claim that $\gamma$ is then a feature of the first kind described in the proposition. 
  Setting $g = g_jg_i^{-1}$ and $\calP = \calP'\restricted{N_{r, R}(\gamma') \intersection N_R(\gamma\restricted{[a, b]})}$, conditions \ref{cutpairshort}, \ref{cutpairmatch}, and~\ref{cutpairpartition} are satisfied by definition of the $a_i$. 
  Let $c \in [a, b]$ such that $\gamma'(c) \in X_0$. 
  Then $C_K(\gamma) \intersection \Ball_{T}(\gamma(c))$ is equal to $C_K(\gamma') \intersection \Ball_{T}(\gamma'(c))$ since $\eta \geq \lambda(T+K+\lambda)$ and Proposition~\ref{proposition:cylinder_disconnected_locally} guarantees that both sets in $\calP'$ meet this set, so Condition~\ref{cutpairnontrivial} is satisfied, too.
\end{proof}

\subsection{Circular boundaries}\label{section:detecting_circular_boundaries}

We now show that the boundary is homeomorphic to a circle if and only if a large ball in the cusped space satisfies a geometric condition.
To identify circles we require the following theorem from point-set topology.

\begin{theorem}
  \cite[II.2.13]{wilder49}\label{thm:topologyimpliescircle} 
  Let $X$ be a separable, connected, locally connected space containing more than one point that is without a cut point.
  If every pair of points in $X$ is a cut pair then $X$ is homeomorphic to $S^1$.
\end{theorem}

Recall a theorem of Bowditch~\cite[Theorem 0.1]{bowditch99b}: if $G$ is hyperbolic relative to $\calH$ where every element of $\calH$ is finitely presented, one-{} or two-ended and contains no infinite torsion subgroup then $\boundary(G, \calH)$ is locally connected whenever it is connected.
The boundary of a relatively hyperbolic group is a compact metric space, so it is separable.
Therefore we see that the boundary of a relatively hyperbolic group is homeomorphic to a circle if and only if it is connected, does not contain a cut point and every pair of points is a cut pair.
We call a pair of points that is not a cut pair a \emph{non-cut pair.}
This reduces the problem of identifying circular boundaries to the problem of determining whether or not $\boundary (G, \calH)$ contains a non-cut pair.

By a similar argument to that of Section~\ref{section:detecting_cut_pairs} we now show that the existence of a non-cut pair is equivalent to the presence in a ball of large radius in the cusped space of a short coarsely non-separating geodesic segment.
First we prove the following lemma.

\begin{lemma}\label{lem:equivalenceclass} 
  Suppose that $\boundary X$ does not contain a cut point. 
  Let $(x_n)_{n \in \integers}$ be a sequence of points in $\boundary X$ with $x_n \to x_{\pm \infty}$ as $n\to \pm \infty$. 
  Suppose that each pair $\{x_n, x_{n+1}\}$ is a cut pair. 
  Then so is $\{x_{-\infty}, x_\infty\}$.  
\end{lemma}

\begin{proof} 
  The boundary $\boundary X$ is locally connected by the main theorem of~\cite{bowditch99b}. 
  By assumption $\boundary X$ does not contain a cut point, so the results of~\cite[Sections 2 and 3]{bowditch98} can be applied. 
  We recall some definitions from that paper; these definitions were previously discussed in Section~\ref{section:bowditch_decomposition}.
  For $x \in \boundary X$ we define $\val(x) \in \naturals$ to be the number of ends of $\boundary X - \{x\}$. 
  Then we let $M(n) = \{x \in \boundary X \colon \val(x) = n\}$ and $M(n{+}) = \{x \in \boundary X \colon \val(x) \geq n\}$. 
  For $x$ and $y$ in $M(2)$ we write $x \equivrelation y$ if $x = y$ or $\{x, y\}$ is a cut pair; this defines an equivalence relation. 
  For $x$ and $y$ in $M(3{+})$ we write $x \approx y$ if $\val(x) = \val(y)$ and $\boundary X - \{x, y\}$ has exactly $\val(x)$ components.

  Recall~\cite[Lemma 3.8]{bowditch98}: if $x \approx y$ and $x \approx z$ then $y \approx z$. 
  Therefore $x_n \in M(2)$ for all $n$, so $\{x_n\}_{n \in \integers}$ is a subset of a $\equivrelation$-equivalence class $\sigma$. 
  By~\cite[Lemma 3.2]{bowditch98} $\sigma$ is a cyclically separating set, and so is the closure of $\sigma$ by~\cite[Lem.\ 2.2]{bowditch98}, which implies that $\{x_{-\infty}, x_\infty\}$ is a cut pair as required.
\end{proof}

Let $\lambda$, $r$, $R$, $K$, $T$, $k$, $\rho$, $\eta$, $\Val$ and $V$ be constants as defined in Section~\ref{section:detecting_cut_pairs}. 
Let $N_1$, $N_2$ and $N_3$ be given by the following expressions.
\begin{align} 
  N_1 &= 2(V-1)((k+R+1)\Val^{2\eta}V^{V+1} + 2\eta) + 2\eta + 2((k+R+1)\Val^{2\eta} + 1),\\
  N_2 &= (k+R+1)\Val^{2\eta} + 1,\\
  N_3 &= 2(k + R + 1)\Val^{2\eta}V^{V+1} + 4\eta.
\end{align}

\begin{theorem}\label{theorem:circular_boundary_feature} 
  The boundary $\boundary (G,\calH)$ is homeomorphic to a circle if and only if $X$ contains one of the following two features.
\begin{enumerate}
  \item 
    A short non-separating geodesic: geodesic segments $\gamma_i \colon [a_i - \eta, b_i + \eta] \to X$ with image in $X_k$ for $i = 1, 2, 3$ with $a_2 = b_1$ and $a_3 = b_2$ satisfying the following conditions.
    \begin{enumerate}
      \item\label{noncutpairshort} The segments are short: $1 \leq b_i - a_i \leq N_1$ for $i = 1$ and for $i = 3$, and $1 \leq b_2 - a_2 \leq N_2$.
      \item\label{noncutpairmatching} 
        The ends of the segments match: $\gamma_i\restricted{[b_i - \eta, b_i + \eta]} = \gamma_{i+1}\restricted{[a_i - \eta, a_i + \eta]}$ for $i = 1, 2$. 
        Also, $h(\gamma_i(a_i)) = h(\gamma_i(b_i))$ for $i = 1, 3$, so there exist $g_i \in G$ such that $\gamma_i(b_i) = g_i\gamma(a_i)$.
        Finally, $\gamma_i\restricted{[b_i - \eta, b_i + \eta]} = g_i \cdot\gamma_i\restricted{[a_i - \eta, b_i + \eta]}$ for $i = 1, 3$.
      \item\label{noncutpairconnected} 
        The segment does not coarsely separate: all vertices of $C_K(\gamma_2) \intersection \Ball_{T}(\gamma_2(c))$ lie in the same connected component of $N_{r, R}(\gamma_2) \intersection N_R(\gamma_2\restricted{[a_2, b_2]})$ for some $c \in [a_2, b_2]$ such that $\gamma_2(c) \in X_0$.
    \end{enumerate}
  \item A horseshoe shaped non-separating geodesic: a geodesic segment $\gamma \colon [a, b] \to X$ with image in $X_k$ satisfying the following conditions.
  \begin{enumerate}
    \item The segment is short: $b - a \leq N_3$.
    \item Both ends are loose ends: $h(\gamma(a)) = h(\gamma(b)) = k$ with $\gamma$ descending vertically at $a$ and ascending vertically at $b$.
    \item The segment does not coarsely separate: $\Cyl'_{r, R, K}(\gamma)$ is connected.
  \end{enumerate}
\end{enumerate}
\end{theorem}

\begin{proof} 
  First suppose that $X$ contains a feature of the first kind described in the proposition.  
  Then define a path $\gamma'$ in $X$ as follows:
  \begin{align} \gamma'(t) = 
    \begin{dcases}
      g_1^m \cdot \gamma_1(t') & \text{if $t = m(b_1 - a_1) + t'$ for 
        $m \in \integers_{\leq 0}$ and $t' \in [a_1, b_1]$ }\\
      \gamma_2(t) & \text{if $t \in [a_2, b_2]$}\\
      g_3^m \cdot \gamma_3(t') & \text{if $t = m(b_3 - a_3) + t'$ for 
        $m \in \integers_{\geq 0}$ and $t' \in [a_3, b_3]$ }\\
    \end{dcases}
  \end{align}
  That is, $\gamma'$ is obtained by concatenating infinitely many translates of $\gamma_1$, then a copy of $\gamma_2$, then infinitely many translates of $\gamma_3$. 
  Note that this is an $(8\delta + 1)$-local-geodesic by Condition~\ref{noncutpairmatching} since $\eta \geq 8\delta + 1$ and is therefore a $\lambda$-quasi-geodesic.

  Since $\gamma'\restricted{[a_2 - \eta, b_2 + \eta]} = \gamma_2$ and $\eta \geq \lambda(T + K + \lambda)$, $C_K(\gamma') \intersection \Ball_{T}(\gamma'(c))$ is equal to $C_K(\gamma_2) \intersection \Ball_{T}(\gamma_2(c))$.
  Furthermore, $\eta \geq \lambda (R + r + \lambda)$, which similarly guarantees that $N_{r, R}(\gamma_2) \intersection N_R(\gamma_2\restricted{[a_2, b_2]})$ is a subset of $N_{r, R}(\gamma')$. 
  Therefore $C_K(\gamma') \intersection \Ball_{T}(\gamma'(c))$ lies in a single component of $\Cyl_{r, R, K}(\gamma')$ by Condition~\ref{noncutpairconnected}, so $\Cyl_{r, R, K}(\gamma')$ is connected by Proposition~\ref{proposition:cylinder_disconnected_locally}, so $\Lambda\gamma'$ is a non-cut pair by the results of Section~\ref{section:cylinders_and_the_boundary}.

  Now suppose that a feature of the second type exists in $X$. 
  Let $\hat\gamma$ be the path obtained by concatenating $\gamma$ with vertical geodesic rays.  
  Then $\hat\gamma$ is an $(8\delta+1)$-local-geodesic since $k \geq 8\delta+1$.  
  Also, $\Cyl_{r, R, K}(\hat\gamma)$ is connected by Lemma~\ref{lemma:geodesics_in_horoballs}, so $\Lambda\hat\gamma$ is a non-cut pair by the results of Section~\ref{section:cylinders_and_the_boundary}.

  Conversely, suppose that $\boundary X$ contains a non-cut pair. 
  Let $\gamma'$ be an $(8\delta+1)$-local-geodesic in $X$ such that $\Lambda\gamma'$ is such a pair. 
  Assume first that $\gamma'$ is contained in $X_{k+R}$ and reparametrise $\gamma'$ so that $\gamma'(0) \in X_0$. 
  Then $C_K(\gamma') \intersection \Ball_{T}(\gamma'(0))$ contains at most $V$ vertices and lies in a single component of $N_{r, R}(\gamma')$. 
  Define $n_V^\pm$ to be $\pm\lambda(K+T+\lambda)$ and then for $l$ decreasing from $V-1$ to $1$ let $n_l^+$ and $n_l^-$ be chosen to minimise $n_l^+ - n_l^-$ among pairs such that $C_K(\gamma') \intersection \Ball_{T}(\gamma'(0))$ meets at most $l$ components of 
  \begin{align}
    N_{r, R}(\gamma') \intersection N_R(\gamma'\restricted{[n_l^-, n_l^+]})
  \end{align}
  and $[n_{l+1}^-, n_{l+1}^+] \subset [n_l^-, n_l^+]$.  
  Note that the condition that $\mod{n^\pm_l} \geq \lambda(K + T + \lambda)$ ensures that $C_K(\gamma') \intersection \Ball_{T}(\gamma'(0))$ is a subset of $N_{r, R}(\gamma') \intersection N_R(\gamma'\restricted{[n_l^-, n_l^+]})$.

  Suppose that $n_{l - 1}^+ - n_l^+ > (k + R + 1)\Val^{2\eta} V^{V+1} + 2\eta$. 
  Let $\mathcal{Q}$ be the partition of $C_K(\gamma') \intersection \Ball_{T}(\gamma'(0))$ into $l$ non-empty subsets induced by connectivity in $N_{r, R}(\gamma') \intersection N_R(\gamma'\restricted{[n_{l-1}^-, n_{l-1}^+-1]})$.  
  For each $t \in [n_l + \eta, n_{l-1} - \eta]$ define the following sets:
  \begin{align}
    Y_t &= N_{r, R}(\gamma') \intersection \Ball_{\rho}(\gamma'(t)) \\
    Z_t &= N_{r, R}(\gamma') \intersection \left(\Ball_{\rho}(\gamma'(t)) \union
      N_R(\gamma'\restricted{[n_{l-1}^-, t]})\right)
  \end{align}
  and let $\calP_t$ be the partition of the vertices of $Y_t$ into $l + 1$ subsets: $l$ corresponding to the $l$ sets in $\mathcal{Q}$ by connectivity in $Z_t$ and one containing the part of $Y_t$ not connected to $C_K(\gamma') \intersection \Ball_{T}(\gamma'(0))$ in $Z_t$.  
  As in the proof of Theorem~\ref{theorem:cut_pair_feature} there exist $s_1 < s_2$ in $[n^+_l + \eta, n^+_{l-1} - \eta]$ such that the following conditions hold.
  \begin{enumerate}
    \item The geodesic $\gamma'$ matches at $s_1$ and $s_2$:  $h(\gamma'(s_1)) = h(\gamma'(s_2))$, so $\gamma'(s_2) = g\gamma'(s_1)$ for some $g$ in $G$, and $\gamma'\restricted{[s_2-\eta,s_2+\eta]} = g\cdot\gamma' \restricted{[s_1-\eta,s_1+\eta]}$, which implies that $Y_{s_2} = g\cdot Y_{s_1}$.
    \item The partitions matches at $s_1$ and $s_2$: the partition $\calP_{s_2}$ is equal to the translation of $\calP_{s_1}$ by the group element $g$.
  \end{enumerate}
  Then replace $\gamma'$ by another path defined by
  \begin{align} 
    \gamma''(t) = 
      \begin{dcases}
        \gamma'(t) & \text{if $t \leq s_1$}\\
        g^{-1}\cdot \gamma'(t + (s_2 - s_1)) & \text{if $t \geq s_2$} \\
      \end{dcases}
  \end{align}

  This is an $(8\delta+1)$-local-geodesic since $\eta \geq 8\delta+1$. 
  By definition of $\rho$, the intersection of $Z_{s_2}$ and $N_{r, R}(\gamma') \intersection N_R(\gamma'\restricted{[s_2, n_{l-1}^+]})$ is contained in $Y_{s_2}$. 
  Then by definition of $n_{l-1}^+$, two distinct sets in $\calP_{s_2}$ meet $N_{r, R}(\gamma') \intersection N_R(\gamma'\restricted{[s_2, n_{l-1}^+]})$. 
  Since $\eta \geq \lambda(R + r + \lambda)$,
  \begin{align}
    g^{-1}N_{r, R}(\gamma') \intersection N_R(\gamma'\restricted{[s_2, n_{l-1}^+]})
    = N_{r, R}(\gamma'') \intersection N_R(\gamma'\restricted{[s_1, n_{l-1}^+ -
    (s_2 - s_1)]}),
  \end{align}
  and it follows that $C_K(\gamma'') \intersection \Ball_{T}(\gamma'(0))$ meets at most $l-1$ components of 
  \begin{align}
    N_{r, R}(\gamma'') \intersection N_R(\gamma''\restricted{[n_{l-1}^-, n_{l-1}^+ - (s_2 - s_1)]}). 
  \end{align}
  This implies that the process of replacing $\gamma'$ by $\gamma''$ leaves unchanged $n_{l'}^+$ for all $l' < l$ and $n_{l'}^-$ for all $l'$ and strictly reduces $n_l^+$.  
  Therefore by repeating this process we can assume that $\gamma'$ was chosen to ensure that $\mod{n_{l-1}^\pm - n_l^\pm}$ is at most $(k+R+1)\Val^{2\eta}V^{V+1} + 2\eta$ for all $l$, and therefore that 
  \begin{align}
    \mod{n_1^\pm} \leq \lambda(K + T + \lambda) + (V-1)((k+R+1)\Val^{2\eta}V^{V+1} + 2\eta).
  \end{align}

  There exist $a_1 \leq b_1 \leq n_1^- - \eta$ with $n_1^- - b_1$ and $b_1 - a_1$ both at most $(k+R+1)\Val^{2\eta} + 1$ such that the following matching conditions hold.
  \begin{enumerate}
    \item Firstly, $h(\gamma'(b_1)) = h(\gamma'(a_1))$, so $\gamma'(b_1) = g_1\cdot \gamma'(a_1)$ for some $g_1 \in G$.
    \item Secondly, $\gamma'\restricted{[b_1 - \eta, b_1 + \eta]} = g_1\cdot\gamma'\restricted{[a_1 - \eta, a_1 + \eta]}$.
  \end{enumerate}
  Then let $\gamma_1 = \gamma'\restricted{[a_1 - \eta, b_1+\eta]}$. 
  Similarly define $b_3 \geq a_3 \geq n_1^+$ and let $\gamma_3 = \gamma'\restricted{[a_3 - \eta, b_3 + \eta]}$. 
  Let $a_2 = b_1$ and $b_2 = a_3$ and let $\gamma_2 = \gamma'\restricted{[a_2 - \eta, b_2 + \eta]}$; note that $b_2 - a_2 \leq N_1$.  
  Then the triple $(\gamma_1, \gamma_2, \gamma_3)$ is a feature in $X$ of the first kind listed in the proposition: conditions~\ref{noncutpairshort} and~\ref{noncutpairmatching} clearly hold by construction and Condition~\ref{noncutpairconnected} holds because the condition that $\eta \geq \lambda(R + r + \lambda)$ ensures that $N_{r, R}(\gamma') \intersection N_R(\gamma'\restricted{[n_1^-, n_1^+]})$ is a subset of $N_{r, R}(\gamma_2) \intersection N_R(\gamma_2\restricted{[a_2, b_2]})$.

  If $\gamma'$ is not contained in $X_{k+R}$ let $\gamma'^{-1}h^{-1}[0, k+R]$ be a (possibly infinite) union of (possibly infinite) intervals $\union_{i \in I}[a_i-R, b_i+R]$ where we order the intervals so that $a_{i+1} > b_i$ for all $i$. 
  Then $h(\gamma'(a_i)) = h(\gamma'(b_i)) = k$ for all $i$ and we can apply the results of Section~\ref{section:geodesics_in_thin_part}.  
  For $i \in I$ let $\gamma'_i = \gamma'\restricted{[a_i, b_i]}$. 
  Let $\hat{\gamma}'_i$ be the bi-infinite $(8\delta + 1)$-local-geodesic obtained by concatenating $\gamma_i$ with either one or two vertical geodesic rays.

  Suppose that $\Lambda\hat\gamma'_i$ is a cut pair for all $i$. 
  Then Lemma~\ref{lem:equivalenceclass} tells us that $\Lambda\gamma'$ is a cut pair, which is a contradiction.  
  Therefore there exists $i$ such that $\Lambda\hat\gamma_i$ is a non-cut pair. 

  Arguing as before, the geodesic $\hat\gamma_i$ can be altered to ensure that $b_i - a_i \leq 4\eta + 2(k+R+1)\Val^{2\eta}V^{V+1}$; this yields a feature of the second type described in the proposition.  
\end{proof}

\section{Generalisations}\label{section:cut_pairs_generalisations}

It seems possible that these methods could work for a wider class of relatively hyperbolic groups.
The assumption that the peripheral subgroups are virtually cyclic is used in two essential steps.
Firstly, we need the cusped space to satisfy a double dagger condition.
As discussed in Section~\ref{section:double_dagger}, the cusped space is known by work of Dahmani and Groves~\cite{dahmanigroves08a} to satisfy this condition when the peripheral subgroups are finitely generated abelian, and we showed in Proposition~\ref{proposition:double_dagger_thin_virtually_cyclic} that this result may be extended to allow virtually cyclic peripheral subgroups.
The arguments used make use of fairly explicit descriptions of paths in the Cayley graphs of the peripheral subgroups, so any generalisation to a substantially broader class of peripheral subgroups would require new methods.

Secondly, in Section~\ref{section:geodesics_in_thin_part} we assumed that the peripheral subgroups are virtually cyclic in our analysis of the geometry of cylinders in the thin part of the cusped space.
Lemma~\ref{lemma:geodesics_in_horoballs} followed from the two-endedness of the peripheral subgroups. 
It seems likely that a similar result will hold for one-ended peripheral subgroups: the cylinder $\Cyl_{r, R, K}(\gamma)$ should be connected whenever $\gamma$ strays too deep into a horoball based on a one-ended subgroup.
However, without the double dagger condition this fact does not seem to be particularly useful.

