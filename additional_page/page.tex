\documentclass{article}
\title{Detecting topological properties of boundaries of hyperbolic groups}
\author{Benjamin Barrett}
\date{2018}

\begin{document}
\maketitle
\begin{abstract}

In general, a finitely presented group can have very nasty properties, but many of these properties are avoided if the group is assumed to admit a nice action by isometries on a space with a negative curvature property, such as Gromov hyperbolicity. 
Such groups are surprisingly common: there is a sense in which a random group admits such an action, as do some groups of classical interest, such as fundamental groups of closed Riemannian manifolds with negative sectional curvature. 
If a group admits an action on a Gromov hyperbolic space then large scale properties of the space give useful invariants of the group.  
One particularly natural large scale property used in this way is the Gromov boundary.

The Gromov boundary of a hyperbolic group is a compact metric space that is, in a sense, approximated by spheres of large radius in the Cayley graph of the group. 
The technical results contained in this thesis are effective versions of this statement: we see that the presence of a particular topological feature in the boundary of a hyperbolic group is determined by the geometry of balls in the Cayley graph of radius bounded above by some known upper bound, and is therefore algorithmically detectable. 

Using these technical results one can prove that certain properties of a group can be computed from its presentation.
In particular, we show that there are algorithms that, when given a presentation for a one-ended hyperbolic group, compute Bowditch's canonical decomposition of that group and determine whether or not that group is virtually Fuchsian.

The final chapter of this thesis studies the problem of detecting \v{C}ech cohomological features in boundaries of hyperbolic groups.
Epstein asked whether there is an algorithm that computes the \v{C}ech cohomology of the boundary of a given hyperbolic group.
We answer Epstein's question in the affirmative for a restricted class of hyperbolic groups: those that are fundamental groups of graphs of free groups with cyclic edge groups.
We also prove the computability of the \v{C}ech cohomology of a space with some similar properties to the boundary of a hyperbolic group: Otal's decomposition space associated to a line pattern in a free group.

\end{abstract}
\end{document}


