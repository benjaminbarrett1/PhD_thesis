\section{Splittings of groups with circular boundary}\label{sec:fuchsiangroups}

The question of the existence of a relative splitting of a hyperbolic group
cannot be answered by consideration of the topology of its boundary alone if
the boundary is homeomorphic to a circle: some groups with circular boundary
split and some do not. In this section we deal with this special case. 

\subsection{Circular boundary}

Let $\Gamma$ by a hyperbolic group with a finite collection $\mathcal{H}$ of
subgroups such that $\Gamma$ is hyperbolic relative to $\mathcal{H}$ and
$\boundary(\Gamma, \mathcal{H})$ is homeomorphic to $S^1$. Then $\Gamma$ acts
as a discrete convergence group on $\boundary(\Gamma, \mathcal{H})$
by~\cite{bowditch99c}, so by the Convergence Group Theorem of Tukia, Casson,
Jungreis and Gabai~\cite{tukia88,cassonjungreis94,gabai92} there is a properly
discontinuous action of $\Gamma$ by isometries on $\mathbb{H}^2$ and a
$\Gamma$-equivariant homeomorphism $\boundary(\Gamma, \mathcal{H}) \to
\boundary\mathbb{H}^2$.

Let $K$ be the (finite) kernel of the action of $\Gamma$ on $\boundary(\Gamma,
\mathcal{H})$; note that this is the same as the kernel of the extension of the
action to $\mathbb{H}^2$. Then $\Gamma$ is an extension

  \begin{center}\begin{tikzpicture}[>=angle 90]
  \matrix (a) [matrix of nodes, row sep=3em, column sep = 2em, text height = 
    1.5ex, text depth=0.25ex]
  {
    $1$ & $K$ & $\Gamma$ & $\Gamma'$ & $1$\\
  };
  \path[->](a-1-1) edge (a-1-2)
    (a-1-2) edge (a-1-3)
    (a-1-3) edge (a-1-4)
    (a-1-4) edge (a-1-5);
  \end{tikzpicture}\end{center}
%
and the quotient $\Gamma'$ acts faithfully on $\mathbb{H}^2$. Let
$\mathcal{H}'$ be the set of images of elements of $\mathcal{H}$ in $\Gamma'$.

\begin{lem}\label{lem:quotientbyeffectivekernel} With $\Gamma$ and $\Gamma'$ as
above, $\Gamma$ splits non-trivially over a virtually cyclic subgroup relative to
$\mathcal{H}$ if and only if $\Gamma'$ admits such a splitting relative to
$\mathcal{H}'$.\end{lem}

\begin{proof} One direction is clear: if $\Gamma'$ acts minimally on a
non-trivial tree $T$ with virtually cyclic edge stabilisers then the quotient
map $\Gamma \to \Gamma'$ induces a minimal $\Gamma$-action on $T$ with
edge stabilisers finite extensions of the edge stabilisers of the
$\Gamma'$-action. 

Conversely, suppose that $\Gamma$ acts minimally on a non-trivial tree $T$ with
virtually cyclic edge stabilisers. $K$ is finite, so the restriction of the
action to $K$ fixes a vertex $v$. $K$ is normal, so its action fixes
$\Gamma\cdot v$ pointwise. Any point in $T$ lies on a geodesic path connecting
points in $\Gamma\cdot v$, since the union of all such paths is a
$\Gamma$-invariant subtree of $T$ and the action was assumed to be minimal.
Therefore $K$ acts trivially on $T$.

It follows that the $\Gamma$-action descends to a $\Gamma'$ action, and the
edge stabilisers are quotients of the original edge stabilisers by finite
subgroups.  Furthermore, elements of $\mathcal{H}'$ are elliptic if elements of
$\mathcal{H}$ are. \end{proof}

Any finite normal subgroup of $\Gamma$ is contained in the ball of radius
$4\delta + 2$ centred at the identity by~\cite{bridsonhaefliger99}, so by
checking all finite subsets of this ball using a solution to the word problem,
the set of finite normal subgroups of $\Gamma$ can be computed. $K$ is the
unique maximal finite normal subgroup of $\Gamma$ and is therefore computable. 

As described above, $\Gamma'$ can be realised as a discrete subgroup of
$\Isom\mathbb{H}^2$. The conjugacy classes of elements of $\mathcal{H}'$ are
then precisely the conjugacy classes of maximal parabolic subgroups of
$\Gamma'$. The action of $\Gamma'$ on $\boundary(\Gamma', \mathcal{H}')$ is
minimal, so the limit set of the action is $S^1$. It follows that $\Gamma'$ is
a Fuchsian group of the first kind: $\mathbb{H}^2/\Gamma'$ is a finite volume
orbifold and $\mathcal{H}'$ is a choice of conjugacy class representatives for
the cusp subgroups. Truncating the cusps of the orbifold, we realise $\Gamma'$
as the fundamental group of a compact hyperbolic orbifold such that
$\mathcal{H}'$ is a choice of conjugacy class representatives for the boundary
subgroups.  

\begin{defn}\label{defn:boundedFuchsian} A group is~\emph{bounded Fuchsian} if
it acts properly discontinuously and convex cocompactly (i.e.\ cocompactly on
a convex subset) on $\mathbb{H}^2$. Then a group is bounded Fuchsian if and
only if it surjects with finite kernel onto the fundamental group of a compact
two-dimensional hyperbolic orbifold. The \emph{peripheral
subgroups} of a bounded Fuchsian group are the fundamental groups of the
boundary components of the associated orbifold.\end{defn}

Therefore we have shown that $\Gamma'$ is a bounded Fuchsian group and
$\mathcal{H}'$ is a collection of representatives of its conjugacy classes of
peripheral subgroups. This proves the following lemma:

\begin{lem}\label{lem:grouptoorbifold} There is an algorithm that, when given a
presentation for a hyperbolic group with a set $\mathcal{H}$ of virtually
cyclic peripheral subgroups, returns a presentation for another hyperbolic
group $\Gamma'$ with a set $\mathcal{H}'$ of virtually cyclic subgroups such
that $\boundary(\Gamma', \mathcal{H}')$ is homeomorphic to $\boundary(\Gamma,
\mathcal{H})$ and $\Gamma$ splits over a virtually cyclic subgroup relative
to $\mathcal{H}$ if and only if $\Gamma'$ splits over a virtually cyclic
subgroup relative to $\mathcal{H}'$. Furthermore, if $\boundary(\Gamma',
\mathcal{H}')$ is homeomorphic to a circle then $\Gamma'$ is the fundamental
group of a compact two-dimensional hyperbolic orbifold and $\mathcal{H}'$ is
a set of representatives of fundamental groups of components of the boundary of
the orbifold.  \end{lem}

\subsection{Orbifolds}

Given the results of the previous section, the problem of determining whether
or not a given group $\Gamma$ with circular boundary splits non-trivially over
a virtually cyclic subgroup relative to a collection $\mathcal{H}$ of
peripheral subgroups reduces to the case in which $\Gamma$ is a bounded
Fuchsian group and $\mathcal{H}$ is a set of conjugacy class representatives of
its peripheral subgroups. Let $\Gamma = \pi_1Q$ where $Q$ is a compact
hyperbolic orbifold, so $\mathcal{H}$ is a set of conjugacy class
representatives of the boundary subgroups of $Q$.

We recall some terminology to describe orbifolds. The universal cover of a
hyperbolic orbifold (possibly with boundary) is a convex subspace $\tilde{Q}$
of $\mathbb{H}^2$. The orbifold fundamental group $\pi_1Q$ acts properly
discontinuously and by isometries on $\tilde{Q}$. The \emph{underlying
topological surface} $Q_\text{top}$ of $Q$ is defined to be the topological
quotient of $\tilde{Q}$ by this action. The \emph{topological boundary}
$\boundary_\text{top}$ of $Q$ is the boundary of $Q_\text{top}$, while the
\emph{orbifold boundary} $\boundary Q$ of $Q$ is the image of the boundary of
$\tilde{Q}$ in $Q_\text{top}$ under the quotient projection. A \emph{mirror} in
$Q$ is the image in $Q_\text{top}$ of the set of fixed points of an order 2
orientation reversing isometry of $\tilde{Q}$ in $\pi_1Q$. All mirrors in $Q$
are contained in $\boundary_\text{top}Q$ and any mirror is homeomorphic either
to a circle or to an interval. In the latter case each end point of the
interval is contained in either $\boundary Q$ or in another mirror. The
intersection of two mirrors is called a \emph{corner reflector}; the stabiliser
in $\pi_1Q$ of the preimage in $\tilde{Q}$ of a corner reflector is a finite
dihedral group. If $x$ is a point in $\tilde{Q}$ whose stabiliser in $\pi_1Q$
is non-trivial, cyclic and orientation preserving then the image of $x$ in
$Q_\text{top}$ is called a \emph{cone point}. The \emph{singular locus} of $Q$
is the union of all mirrors, corner reflectors and cone points. If $y$ is not
contained in the singular locus of $Q$ then any preimage of $y$ in $\tilde{Q}$
has trivial stabiliser in $\pi_1Q$.

A \emph{geodesic} in $Q$ is a curve $\gamma$ that is the image of a geodesic
$\tilde\gamma$ in $\tilde{Q}$ under the quotient projection. A geodesic
$\gamma$ is \emph{closed} if it is compact and \emph{simple} if it is closed
and furthermore $g\cdot\tilde\gamma$ is either equal to or disjoint from
$\tilde\gamma$ for each $g$ in $\pi_1Q$. A simple closed geodesic is
homeomorphic to either a circle or an interval with each of its end points
contained in a mirror in $Q$. A simple closed geodesic is \emph{essential} if
it is not contained in $\boundary_\text{top}Q$. For more information about
orbifolds see~\cite{scott83}.

The theory of splittings of fundamental groups of orbifolds relative to their
boundary subgroups is developed in~\cite{guirardellevitt16}; we recall the
following results:

\begin{lem}\cite[Corollary 5.6]{guirardellevitt16}\label{lem:splitiffnonsmall}
$\Gamma$ splits non-trivially relative to $\mathcal{H}$ if and only if $Q$
contains an essential simple closed geodesic.\end{lem}

\begin{defn} We call a compact orbifold without an essential simple closed geodesic 
\emph{small}.\end{defn}

\begin{prop}\cite[Proposition 5.12]{guirardellevitt16}\label{prop:characterisationofsmall}
A hyperbolic 2-orbifold $Q$ is small if and only if it is one of the following. 
\begin{enumerate}
  \item A sphere with three cone points, so $\pi_1Q \isom \langle a, b \vert
    a^p, b^q, ab^r\rangle$ where $p^{-1}+q^{-1}+r^{-1} < 1$ and the peripheral
    structure is empty.
  \item A triangle, all three edges of which are mirrors, so $\pi_1Q$ has
    presentation $\langle a, b, c \vert a^2, b^2, c^2, (ab)^p, (bc)^q,
    (ca)^r\rangle$ where $p^{-1}+q^{-1}+r^{-1} < 1$ and the peripheral
    structure is empty.
  \item A disc with two cone points, so $\pi_1Q \isom \langle a, b \vert a^p,
    b^q\rangle$ where $p, q > 1$ and the peripheral structure is $\{\langle
    ab\rangle\}$.
  \item A cylinder with one cone point, so $\pi_1Q \isom \langle a, b \vert
    (ab)^p\rangle$ where $p >1$ and the peripheral structure is $\{\langle
    a\rangle, \langle b\rangle\}$.
  \item A pair of pants, so $\pi_1Q$ is the free group $\langle a, b
    \vert\rangle$ and the peripheral structure is $\{\langle a\rangle, \langle
    b\rangle, \langle c\rangle\}$.
  \item A disc with one cone point with edge consisting of an interval boundary
    component and a mirror, so $\pi_1Q \isom \langle a, t \vert a^2,
    t^p\rangle$ where $p > 1$ and the peripheral structure is $\{\langle a,
    tat^{-1}\rangle\}$.
  \item A square with an interval boundary component and three mirrors edges,
    so $\pi_1Q \isom \langle a, b, c\vert a^2, b^2, c^2, (ab)^p, (bc)^q\rangle$
    where $p + q \geq 1$ and the peripheral structure is $\{\langle a,
    c\rangle\}$.
  \item An annulus in which one edge comprises an interval boundary component and a
    mirror and the other is a circular boundary component, so $\pi_1Q \isom
    \langle a, t \vert a^2\rangle$ with peripheral structure $\{\langle a,
    tat^{-1}\rangle\}$.
  \item A pentagon with two non-adjacent interval boundary components and three
    mirrors as edges, so $\pi_1Q \isom \langle a, b, c \vert a^2, b^2, c^2,
    (ab)^p\rangle$ with peripheral structure $\{\langle b, c\rangle, \langle c,
    a\rangle\}$.
  \item A hexagon, the six edges of which are alternately interval boundary
    components and mirrors, so $\pi_1Q \isom \langle a, b, c \vert a^2, b^2,
    c^2\rangle$ and the peripheral structure is $\{\langle a, b\rangle, \langle
    b, c\rangle, \langle c, a\rangle\}$.
\end{enumerate}
\end{prop}

\begin{lem}\label{lem:identifyingsmallorbifolds} There is an algorithm that
  takes as input a presentation for a hyperbolic group $\Gamma$ and a
  collection $\mathcal{H}$ of maximal virtually cyclic peripheral subgroups and
  terminates if and only if $\boundary(\Gamma, \mathcal{H})$ is homeomorphic to
  a circle and $\Gamma$ does not split relative to $\mathcal{H}$ over a
  virtually cyclic subgroup.
\end{lem}

\begin{proof} First use the algorithm of Lemma~\ref{lem:grouptoorbifold} and
let the output of that algorithm be $(\Gamma', \mathcal{H}')$; then
$\boundary(\Gamma, \mathcal{H})$ is homeomorphic to a circle and $\Gamma$
does not split over a virtually cyclic subgroup relative to $\mathcal{H}$ if
and only if there is an isomorphism from $\Gamma'$ to one of the groups with
peripheral structure listed in proposition~\ref{prop:characterisationofsmall}
that maps elements of $\mathcal{H}'$ to conjugates of elements of that
peripheral structure.
  
Enumerate the groups and peripheral structures described in
Proposition~\ref{prop:characterisationofsmall} and, in parallel, enumerate
all homomorphisms from these groups to $\Gamma'$ and homomorphisms from
$\Gamma'$ to these groups. Note that this is possible since one can test
whether or not a map defined on the generators of a group extends to a
homomorphism using a solution to the word problem in the codomain of the map,
and $\Gamma'$ and all groups listed in
Proposition~\ref{prop:characterisationofsmall} are hyperbolic. The algorithm
then terminates when an inverse pair of such maps that preserve the peripheral
structures (up to conjugacy) is found.  \end{proof}
