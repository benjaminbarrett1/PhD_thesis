\section{Detecting cut points and pairs}\label{sec:cutpairsalgorithms}

We now use the results of the previous section to prove some computability
results concerting topological features of the Bowditch boundary of a hyperbolic group
under the assumption that the cusped space satisfies a double dagger condition. These
are the main technical results of this paper. The idea is to identify the
topological feature of the boundary with a combinatorial feature of the cusped
space of bounded size, so that the existence of that feature can be
determined by looking at only a finite part of the the cusped space.

Let $\Gamma$ be a group hyperbolic relative to a finite set $\mathcal{H}$ of
maximal virtually cyclic subgroups. Let $S$ be a generating set for $\Gamma$ such that
$S \intersection H$ generates $H$ for each $H$ in $\mathcal{H}$. Let $X$ be the
cusped space associated to $(\Gamma, \mathcal{H}, S)$.

\subsection{Geodesics in horoballs}\label{sec:horoballs}

First we must understand the connectivity of neighbourhoods of geodesics in the
thin part of $X$.  We assume that the peripheral subgroups are virtually cyclic, so
the geometry of the cusps of $X$ is relatively simple. For notational
convenience we initially restrict to the case in which $X$ consists of a single
cusp. Then the vertex set of $X$ can be identified with $H \times
\integers_{\geq 0}$ and for any $k$ there is an inclusion of the Cayley graph $\Cay(H, S)
\hookrightarrow h^{-1}(k)$ mapping a vertex $h \in \Cay(H, S)$ to $(h, k)$; for
$k = 0$ this inclusion is an isomorphism of graphs. 

Let $d_H$ be the word metric in $\Cay(H, S)$ and let $\Cay(H, S)$ be
$\delta_H$-hyperbolic with respect to this word metric. Let $\alpha$ be a
bi-infinite geodesic in $\Cay(H, S)$ with respect to $d_H$; then any point in
$\Cay(H, S)$ is within a distance of at most $2\delta_H + 1$ of $\alpha$.

Let $\gamma \colon [0, \infty) \to X$ be a vertical geodesic ray with
$\gamma(0) = (\alpha(0), 0)$. Then for any $k \in \integers_{\geq 0}$, the
vertex set of $h^{-1}(k) \intersection N_{r, R}(\gamma)$ is
\begin{align*}
  \{g \in H \colon 2^{k + r - 1} \leq d_H(g, \alpha(0)) \leq 2^{k + R - 1}\} \times \{k\}.
\end{align*} 
We will denote by $Y_k$ the set $h^{-1}(k) \intersection N_{r, R}(\gamma)$.

Assume now that $k \geq \log_2(2\delta_H + 1)$. Then every vertex in
$h^{-1}(k)$, and therefore every vertex in $Y_k$, is adjacent to a vertex in
$\alpha \times \{k\}$. Therefore $Y_k$ contains connected components $Y_k^+$
and $Y_k^-$ with
\begin{align*}
  \alpha\restricted{[2^{k+r-1}, 2^{k+R-1}]} \times \{k\} & \subset Y_k^+,\\
  \alpha\restricted{[-2^{k+R-1}, -2^{k+r-1}]} \times \{k\} & \subset Y_k^-,\\
\end{align*}
and each of these components meets $C_K(\gamma)$. Therefore each of the sets
$Y_k^\pm$ is a subset of a component of $A_{r, R, K}(\gamma)$. Any vertex in
the complement of these two components of $Y_k$ is contained in 
\begin{align*}
  \{g \in H \colon d_H(g, \alpha(0)) \leq 2^{k + r}\} \times \{k\}.
\end{align*}
Therefore only those vertices of $Y_k$ that are in $Y_k^+ \union Y_k^-$ are
adjacent in $X$ to vertices of $Y_{k+1}$. Furthermore, $Y_k^+$ is adjacent to
$Y_{k+1}^+$ and not to $Y_{k+1}^-$ and likewise for $Y_k^-$. Finally, vertices
that are in $Y_{k+1}$ but not in $Y_{k+1}^\pm$ are adjacent to vertices in
$Y_k$.

Thus, if $k \geq \log_2(2\delta_H + 1)$ then $A_{r, R, K}(\gamma) \intersection
h^{-1}[k, \infty)$ contains two unbounded components $Y_{\geq k}^+$ and
$Y_{\geq k}^-$ containing $\union_{l \geq k} Y_l^+$ and $\union_{l \geq k}
Y_l^-$ respectively and the complement of these two components is contained in
\begin{align*}
  \{g \in H \colon d_H(g, \alpha(0)) \leq 2^{k + r}\} \times \{k\}.
\end{align*}

To make precise the consequences of this description of $A_{r, R, K}(\gamma)$,
we make the following definition, now allowing $X$ to consist of more than a
single cusp. Let $\gamma\colon [a, b] \to X$ be a geodesic segment such that
$h(\gamma(a)) = h(\gamma(b)) = k > R$ such that $h \composed \gamma$ is decreasing
at $a$ and increasing at $b$. Let $\hat\gamma \colon (-\infty, \infty) \to X$
be the path obtained by concatenating $\gamma$ with two vertical geodesic
rays.  Note that this is a $k$-local-geodesic. Let $A_{r, R, K}'(\gamma)$ be
\begin{align*}
A_{r, R, K}(\hat\gamma) - h^{-1}[k, \infty) \intersection
  N_R\left(\hat\gamma (-\infty, a] \union \hat\gamma[b, \infty)\right).
\end{align*} 

The results of this section together give the following lemma, which we shall
use to control the depth to which geodesics in $X$ connecting cut pairs in
$\boundary X$ penetrate into the thin part of $X$.

\begin{lem}\label{lem:geodesicsinhoroballs} Let $\gamma$ and $\hat\gamma$ be as
in the previous paragraph. Let $\delta_\mathcal{H}$ be such that each $H \in
\mathcal{H}$ is $\delta_\mathcal{H}$-hyperbolic with respect to the
generating set $H \intersection S$.  Suppose that $h(\gamma(a)) =
h(\gamma(b)) \geq \min\{R, \log_2(2\delta_\mathcal{H} + 1)\}$. Then the
inclusion $A'_{r, R, K}(\gamma) \hookrightarrow A_{r, R, K}(\hat\gamma)$
induces a bijection between the sets of connected components of those
spaces.\qed\end{lem}

\subsection{Cut points}

We now show that there is an algorithm that determines whether or not
$\boundary X$ contains a cut point under the assumption that $X$
satisfies a double dagger condition.

\begin{prop}\label{prop:cutpointscomputable} There is an algorithm that takes
as input a presentation for a hyperbolic group $\Gamma$ with generating set
$S$, a list of subsets of $S$ generating a collection $\mathcal{H}$ of maximal
virtually cyclic subgroups of $\Gamma$ and integers $\delta$ and $n$ such that the
cusped space is $\delta$-hyperbolic and satisfies $\ddag_n$ and returns the
answer to the question ``does $\boundary(\Gamma, \mathcal{H})$ contain a cut
point?'' \end{prop}

\begin{proof} It is shown in~\cite[Theorem 0.2]{bowditch99a} that any cut point in
$\boundary(\Gamma, \mathcal{H})$ must be the limit point of $gHg^{-1}$ for some
$H \in \mathcal{H}$ and $g \in \Gamma$. Therefore it is sufficient to check
whether or not $\Lambda H$ is a cut point for each $H \in \mathcal{H}$.  Choose
$r$, $R$ and $K$ to simultaneously satisfy the conditions of
lemmas~\ref{lem:disjointcoverbyshadows}, \ref{lem:nonemptyshadows},
\ref{lem:shadowsclopen}, \ref{lem:shadowsconnected}
and~\ref{lem:replaceinftybyR} with $\lambda = 1$ and $\epsilon = 0$. Let
$\delta_\mathcal{H}$ be large enough that each $H$ in $\mathcal{H}$ is
$\delta_\mathcal{H}$-hyperbolic with respect to the generating set $S \intersection H$ and
let $k \geq \log_2(2\delta_\mathcal{H} + 1)$.

Then for each vertical geodesic ray $\gamma$ starting at the identity element
$1 \in X_0$ check whether or not $A_{r, R, K}(\gamma) \intersection h^{-1}([0,
k])$ is connected; as in Lemma~\ref{lem:geodesicsinhoroballs}, it is connected
if and only if $A_{r, R, K}(\gamma)$ is connected. Also, $A_{r, R, K}(\gamma)$
is disconnected if and only if $\Lambda\gamma = \Lambda H$ is a cut point by
lemmas~\ref{lem:disjointcoverbyshadows}, \ref{lem:nonemptyshadows},
\ref{lem:shadowsclopen}, \ref{lem:shadowsconnected}
and~\ref{lem:replaceinftybyR}, where $H$ is the element of $\mathcal{H}$ such
that the combinatorial horoball based on $H \subset \Cay(X, S)$ contains
$\gamma$. This check can be completed in finite time. \end{proof}

\subsection{Cut pairs}\label{subsec:cutpairs}

We now assume that $X$ is $\delta$-hyperbolic and satisfies $\ddag_n$, that
$\boundary X$ contains no cut point and that the Cayley graph of $H$ with
respect to its generating set $S \intersection H$ is $\delta_\mathcal{H}$-hyperbolic for
each $H$ in $\mathcal{H}$.  Then all results of sections~\ref{sec:annulus}
and~\ref{sec:horoballs} can be applied.

First we show that the existence of a cut pair in $\boundary X$ is equivalent
to the existence of a feature in $X$ of known bounded size. Then by searching
for such a feature one can determine whether or not $\boundary X$ contains a
cut pair. We use a pumping lemma argument: we aim to replace an
arbitrary geodesic joining the two points in a cut pair in $\boundary X$ with a
periodic quasi-geodesic with bounded period that also joins the points of a
(possibly different) cut pair.

Before stating the proposition we define some constants. Take $\lambda$ and
$\epsilon$ so that any $(8\delta + 1)$-local-geodesic is a $(\lambda,
\epsilon)$-quasi-geodesic; for example let $\lambda = (12\delta + 1)/(5\delta +
1)$ and let $\epsilon = 2\delta$. Fix $r$, $R$ and $K$ to simultaneously
satisfy the conditions of propositions~\ref{lem:disjointcoverbyshadows},
\ref{lem:nonemptyshadows}, \ref{lem:shadowsclopen}, \ref{lem:shadowsconnected}
and~\ref{lem:replaceinftybyR} and fix $T$ to satisfy the conditions of
Lemma~\ref{lem:Adisconnectedlocally} with this choice of $\lambda$ and
$\epsilon$. Also let 
\begin{align*}
     k &= \max\{8\delta + 1, \log_2(2\delta_\mathcal{H} + 1), T + R\},\\
  \rho &= (2R + \epsilon)\lambda^2 + \epsilon + R \text{ and}\\
  \eta &= \max\{(8\delta + 1)/2, \lambda(T + K) + \lambda\epsilon, \lambda(R + r) + \lambda\epsilon, \lambda(R + \rho) + \lambda\epsilon\}.
\end{align*}
Let $B$ be the maximum valence of any
vertex in $X_{k + R}$ and let $V$ be the maximum number of vertices in any ball
of radius $\rho$ around any vertex in $X_{k + R}$. Then define 
\begin{align*}
  N_\text{min} &= \max\{8\delta+1, \lambda(2R+1) + \lambda\epsilon + 1\} \text{ and}\\
  N_\text{max} &= N_\text{min} \left(k + R + 1\right) B^{2\eta} 2^V + 1.
\end{align*}

\begin{prop}\label{prop:cutpairsfeature} $\boundary X$ contains a cut pair
if and only if $X$ contains one of the following two features:
\begin{enumerate}
  \item A short period geodesic at shallow depth in $X$: a geodesic segment
    $\gamma \colon [a - \eta, b + \eta] \to X$ contained in $X_{k + R}$ such
    that
  \begin{enumerate}
    \item\label{cutpairshort} $N_\text{min} \leq b - a \leq N_{max}$,
    \item\label{cutpairdepth} $h(\gamma(a)) = h(\gamma(b))$, so there exists $g
      \in \Gamma$ such that $\gamma(b) = g \cdot \gamma(a)$,
    \item\label{cutpairlocal} $\gamma\restricted{[b - \eta, b + \eta]} = g\cdot
      \gamma\restricted{[a - \eta, a + \eta]}$,
    \item\label{cutpairnontrivial} there is a partition $\mathcal{P}$ of the
      vertices of $N_{r, R}(\gamma) \intersection
      N_R(\gamma\restricted{[a,b]})$ into two subsets such that adjacent
      vertices lie in the same subset and each of the sets meets $C_K(\gamma)
      \intersection \Ball_T(\gamma(c))$ for some $c \in [a, b]$, and
    \item\label{cutpairpartition} the partition on the vertices of $N_{r,
      R}(\gamma) \intersection \Ball_{\rho}(\gamma(b))$ induced by the restriction of
      $\mathcal{P}$ to that subset is the same as the translate by $g$ of the 
      partition on the vertices of $N_{r, R}(\gamma) \intersection
      \Ball_{\rho}(\gamma(a))$ obtained by the restriction of $\mathcal{P}$; note
      that $N_{r, R}(\gamma) \intersection \Ball_\rho(\gamma(b))$ is equal to $g
      \cdot N_{r, R}(\gamma) \intersection \Ball_\rho(\gamma(a))$ by
      condition~\ref{cutpairlocal}.
  \end{enumerate}
  \item A short horseshoe-shaped geodesic: a geodesic segment $\gamma \colon
    [a, b] \to X$ such that
  \begin{enumerate}
    \item $b - a \leq N_{max} - 2R + 2\eta$
    \item $h(\gamma(a)) = h(\gamma(b)) \geq k$,
    \item $h \composed \gamma$ is decreasing at $a$ and increasing at $b$,
    \item $A'_{r, R, K}(\gamma)$ is disconnected.
  \end{enumerate}
\end{enumerate}
\end{prop}

\begin{proof} First suppose that the first type of feature exists in $X$.
Define a path $\gamma'$ in $X$ by
\begin{align*} \gamma'\left((b - a)m + t\right) = g^m\gamma(a + t)
\end{align*}
for $m \in \integers$ and $t \in [0, b-a]$. $\gamma'$ is an $(8\delta +
1)$-local-geodesic by condition~\ref{cutpairlocal} since $\eta \geq (8\delta +
1)/2$.  It is therefore a $(\lambda, \epsilon)$-quasi-geodesic
by~\cite{bridsonhaefliger99}. We now aim to show that $A_{r, R, K}(\gamma')$ is
disconnected.

Note that $N_R(\gamma')$ is a union $\bigcup_{m\in\integers} g^m\cdot
N_R(\gamma\restricted{[a, b]})$ of translates of neighbourhoods of $\gamma$.
Since $\eta \geq \lambda(R + r) + \lambda\epsilon$, $N_r(\gamma') \intersection
N_R(\gamma\restricted{[a, b]})$ is a subset of $N_{r}(\gamma)$, so is equal to
$N_r(\gamma) \intersection N_R(\gamma\restricted{[a, b]})$.  Therefore $N_{r,
R}(\gamma')$ decomposes as a union
\begin{align*} N_{r, R}(\gamma') = \bigcup_{m \in \integers} g^m\cdot
  \left(N_{r, R}(\gamma) \intersection N_R\left(\gamma\restricted{ 
    [a,b]}\right)\right).
\end{align*}

Since $b - a > \lambda(2R+1) + \epsilon\lambda$, $g^m \cdot
N_R(\gamma\restricted{[a, b]})$ and $g^l \cdot N_R(\gamma\restricted{[a, b]})$
contain no adjacent vertices for $\mod{m - l} \geq 2$. Furthermore, if $l =
m+1$ then any pair of adjacent vertices in these two sets is contained in
$g^m\cdot \Ball_\rho(\gamma(b))$ since $\rho \geq (2R + \epsilon)\lambda^2 +
\epsilon + R$.

For each set $U \in \mathcal{P}$ define a set $U'$ of vertices of $N_{r,
R}(\gamma')$ by letting $u \in U'$ if $g^m u \in U$ for some $m \in \integers$.
This gives a well defined partition $\mathcal{P}'$ of the vertices of $N_{r,
R}(\gamma')$ such that adjacent vertices lie in the same set by condition
\ref{cutpairpartition}. Its restriction to $A_{r, R, K}(\gamma')$ is
non-trivial: $C_K(\gamma')$ contains $C_K(\gamma) \intersection \Ball_T(\gamma(c))$
since $\eta \geq \lambda(T + K) + \lambda\epsilon$ and this set meets both sets
in $\mathcal{P}'$ by condition \ref{cutpairnontrivial}; therefore $A_{r, R,
K}(\gamma')$ is disconnected and therefore $\Lambda\gamma'$ is a cut pair by
the results of section~\ref{sec:annulus}.

Now suppose that the second type of feature exists in $X$. Let $\hat\gamma$ be
the $(8\delta + 1)$-local-geodesic obtained by concatenating $\gamma$ with
vertical geodesic rays. Then $A_{r, R, K}(\hat\gamma)$ is disconnected by
Lemma~\ref{lem:geodesicsinhoroballs} and $\Lambda\hat\gamma$ is a cut pair by
the results of section~\ref{sec:annulus}.

Conversely, suppose that $\boundary X$ does contain a cut pair. Let $\gamma'$
be a geodesic in $X$ such that $\Lambda\gamma'$ is a cut pair. Assume first
that some connected component of $\gamma'^{-1} h^{-1}[0, k + R]$ is an interval of length
less than $N_{max} + 2\eta$, say $[a - R, b + R]$ with $h(\gamma'(a - R)) =
h(\gamma'(b + R)) = k + R$, so $h(\gamma'(a)) = h(\gamma'(b)) = k$. Let $\gamma
= \gamma' \restricted{[a, b]}$.  Let $c \in [a, b]$ such that $h(\gamma'(c)) =
0$.  $A_{r, R, K}(\gamma')$ is disconnected and each component meets
$C_K(\gamma') \intersection \Ball_T(\gamma'(c))$ by
Lemma~\ref{lem:Adisconnectedlocally}. Since $k \geq T$ this is a subset of
$A'_{r, R, K}(\gamma)$, and $A'_{r, R, K}(\gamma)$ is  a subset of $A_{r, R,
K}(\gamma')$, so $A_{r, R, K}'(\gamma)$ is disconnected. Therefore $\gamma'$ is
a feature of the second kind described in the proposition.

On the other hand, suppose that some interval $[-\eta, N_{max} + \eta]$ is a
subset of $\gamma'^{-1} h^{-1}[0, k + R]$. Then there exist $a_0 < a_1 < \dots
< a_{2^V}$ in $[0, N_{max}]$ such that $h(a_i) = h(a_j)$ for all $i$ and $j$,
so $a_i = g_i a_0$ for some $g_i \in \Gamma$, such that
$\gamma'\restricted{[a_i - \eta, a_i + \eta]} = g_i \cdot
\gamma'\restricted{[a_0 - \eta, a_0 + \eta]}$, and such that $a_i - a_{i - 1}
\geq N_\text{min}$. Let $\mathcal{P}'$ be a partition of the vertices of $N_{r,
R}(\gamma')$ into two subsets such that adjacent vertices are in the same set
and so that both sets meet $C_K(\gamma')$. Such a partition
exists by the results of section~\ref{sec:annulus} since $\Lambda\gamma'$ is a
cut pair. Since $\eta \geq \lambda(R + \rho) + \lambda\epsilon$, $N_{r,
R}(\gamma') \intersection \Ball_\rho(\gamma(a_0))$ is equal to
$g_i^{-1} N_{r, R}(\gamma') \intersection \Ball_\rho(\gamma(a_i))$ for all $i$.
This set contains at most $V$ vertices, so there exist $0 \leq i < j \leq 2^V$
such that
\begin{align*}
g_i^{-1}\mathcal{P}'\restricted{N_{r, R}(\gamma') \intersection
  \Ball_\rho(\gamma(a_i))} = g_j^{-1}\mathcal{P}'\restricted{N_{r, R}(\gamma')
  \intersection \Ball_\rho(\gamma(a_j))}.
\end{align*}

Let $a = a_i$ and $b = a_j$ and let $\gamma = \gamma'\restricted{[a - \eta, b +
\eta]}$. We claim that $\gamma$ is then a feature of the first kind described
in the proposition. Setting $g = g_jg_i^{-1}$ and $\mathcal{P} =
\mathcal{P}'\restricted{N_{r, R}(\gamma') \intersection
N_R(\gamma\restricted{[a, b]})}$, conditions \ref{cutpairshort},
\ref{cutpairdepth}, \ref{cutpairlocal} and~\ref{cutpairpartition} are
satisfied by definition of the $a_i$. Let $c \in [a, b]$ such that $\gamma'(c) \in X_0$. Then
$C_K(\gamma) \intersection \Ball_T(\gamma(c))$ is equal to $C_K(\gamma')
\intersection \Ball_T(\gamma'(c))$ since $\eta \geq \lambda(T+K) +
\lambda\epsilon$ and Lemma~\ref{lem:Adisconnectedlocally} guarantees that
both sets in $\mathcal{P}'$ meet this set, so condition~\ref{cutpairnontrivial}
is satisfied, too.\end{proof}

The existence of a geodesic segment with the properties described in the statement of
Proposition~\ref{prop:cutpairsfeature} can be checked by looking at just a
finite ball in $X$. Such a ball can be computed from a solution to the word
problem in $\Gamma$, which exists since $\Gamma$ is hyperbolic. Therefore we
immediately obtain the following corollary:

\begin{cor}\label{cor:cutpaircomputable} There is an algorithm that takes as
input a presentation for a hyperbolic group $\Gamma$ with generating set $S$,
a list of subsets of $S$ generating a collection $\mathcal{H}$ of maximal virtually cyclic
subgroups of $\Gamma$ and integers $\delta$, $\delta_\mathcal{H}$ and $n$ such that the
cusped space associated to $(\Gamma, \mathcal{H}, S)$ is $\delta$-hyperbolic
and satisfies $\ddag_n$ and such that the Cayley graph of each element of
$\mathcal{H}$ with respect to its given generating set is
$\delta_\mathcal{H}$-hyperbolic and returns the answer to the question ``does
$\boundary(\Gamma, \mathcal{H})$ contain a cut pair?'' \qed \end{cor}

\subsection{Non-cut pairs}

By a similar argument, we now show that there is an algorithm that determines
whether or not $\boundary X$ contains a non-cut pair. For the detection of
non-cut pairs we will need the following lemma.

\begin{lem}\label{lem:equivalenceclass} Suppose that $\boundary X$ does not
contain a cut point. Let $(x_n)_{n \in \integers}$ be a sequence of points in
$X$ with $x_n \to x_{\pm \infty}$ as $n\to \pm \infty$. Suppose that each
pair $\{x_n, x_{n+1}\}$ is a cut pair. Then so is $\{x_{-\infty},
x_\infty\}$.  \end{lem}

\begin{proof} $\boundary X$ is locally connected by the main theorem
of~\cite{bowditch99b}. $\boundary X$ is assumed not to contain a cut point,
so the results of sections 2 and~3 of~\cite{bowditch98} can be applied. We
recall some definitions from that paper. For $x \in \boundary X$ we define
$\val(x) \in \naturals$ to be the number of ends of $\boundary X - \{x\}$.
Then we let $M(n) = \{x \in \boundary X \colon \val(x) = n\}$ and $M(n{+}) =
\{x \in \boundary X \colon \val(x) \geq n\}$. For $x$ and $y$ in $M(2)$ we
write $x \sim y$ if $x = y$ or $\{x, y\}$ is a cut pair; this defines an
equivalence relation. For $x$ and $y$ in $M(3{+})$ we write $x \isom y$ if
$\val(x) = \val(y)$ and $\boundary X - \{x, y\}$ has exactly $\val(x)$
components.

Recall~\cite[Lemma 3.8]{bowditch98}: if $x \isom y$ and $x \isom z$ then $y
\isom z$. Therefore $x_n \in M(2)$ for all $n$, so $\{x_n\}_{n \in
\integers}$ is a subset of a $\sim$-equivalence class $\sigma$. By~\cite[Lem.\
3.2]{bowditch98} $\sigma$ is a cyclically separating set, and so is the closure
of $\sigma$ by~\cite[Lem.\ 2.2]{bowditch98}, which implies that $\{x_{-\infty},
x_\infty\}$ is a cut pair as required.  \end{proof}

Let $\lambda$,
$\epsilon$, $r$, $R$, $K$, $T$, $k$, $\rho$, $\eta$, $B$ and $V$ be as defined
in section~\ref{subsec:cutpairs}. Let $N_1$, $N_2$ and $N_3$ be given by
\begin{align*} 
  N_1 &= 2(V-1)((k+R+1)B^{2\eta}V^{V+1} + 2\eta) + 2\eta + 2((k+R+1)B^{2\eta} + 1),\\
  N_2 &= (k+R+1)B^{2\eta} + 1,\\
  N_3 &= 2(k + R + 1)B^{2\eta}V^{V+1} + 4\eta.
\end{align*}

\begin{prop}\label{prop:noncutpairfeature} $\boundary X$ contains a non-cut
pair if and only if $X$ contains one of the following two features:
\begin{enumerate}
  \item Geodesic segments $\gamma_i \colon [a_i - \eta, b_i + \eta] \to X$ with
    image in $X_k$ for $i = 1, 2, 3$ with $a_2 = b_1$ and $a_3 = b_2$ such that
    \begin{enumerate}
      \item\label{noncutpairshortout} $1 \leq b_i - a_i \leq N_1$ for $i = 1,
        3$,
      \item\label{noncutpairshortin} $1 \leq b_2 - a_2 \leq N_2$,
      \item\label{noncutpairlocal1} $\gamma_i\restricted{[b_i - \eta, b_i +
        \eta]} = \gamma_{i+1}\restricted{[a_i - \eta, a_i + \eta]}$ for $i = 1,
        2$,
      \item\label{noncutpairlocal2} $\gamma_i\restricted{[b_i - \eta, b_i +
        \eta]} = g_i \cdot\gamma_i\restricted{[a_i - \eta, b_i + \eta]}$ for $i
        = 1, 3$, and
      \item\label{noncutpairdepth} $h(\gamma_i(a_i)) = h(\gamma_i(b_i))$ for $i
        = 1, 3$, so there exist $g_i \in \Gamma$ such that $\gamma_i(b_i) =
        g_i\gamma(a_i)$,
      \item\label{noncutpairconnected} all vertices of $C_K(\gamma_2)
        \intersection \Ball_T(\gamma_2(c))$ lie in the same connected component of
        $N_{r, R}(\gamma_2) \intersection N_R(\gamma_2\restricted{[a_2, b_2]})$
        for some $c \in [a_2, b_2]$ such that $\gamma_2(c) \in X_0$.
    \end{enumerate}
  \item A geodesic segment $\gamma \colon [a, b] \to X$ with image in $X_k$ such that
  \begin{enumerate}
    \item $b - a \leq N_3$,
    \item $h(\gamma(a)) = h(\gamma(b)) = k$ with $\gamma$ descending vertically
      at $a$ and ascending vertically at $b$, and
    \item $A'_{r, R, K}(\gamma)$ is connected.
  \end{enumerate}
\end{enumerate}
\end{prop}

\begin{proof} First suppose that $X$ contains a feature of the first kind
described in the proposition.  Then define a path $\gamma'$ in $X$ as follows:
\begin{align*} \gamma'(t) = 
  \begin{dcases}
    g_1^m \cdot \gamma_1(t') & \text{if $t = m(b_1 - a_1) + t'$ for 
      $m \in \integers_{\leq 0}$ and $t' \in [a_1, b_1]$ }\\
    \gamma_2(t) & \text{if $t \in [a_2, b_2]$}\\
    g_3^m \cdot \gamma_3(t') & \text{if $t = m(b_3 - a_3) + t'$ for 
      $m \in \integers_{\geq 0}$ and $t' \in [a_3, b_3]$ }\\
  \end{dcases}
\end{align*}
That is, $\gamma'$ is obtained by concatenating infinitely many translates of
$\gamma_1$, then a copy of $\gamma_2$, then infinitely many translates of
$\gamma_3$. Note that this is an $(8\delta + 1)$-local-geodesic by
conditions~\ref{noncutpairlocal1} and~\ref{noncutpairlocal2} since $\eta \geq
8\delta + 1$ and is therefore a $(\lambda, \epsilon)$-quasi-geodesic.

$C_K(\gamma') \intersection \Ball_T(\gamma'(c))$ is equal to $C_K(\gamma_2)
\intersection \Ball_T(\gamma_2(c))$ since $\gamma'\restricted{[a_2 - \eta, b_2 +
\eta]} = \gamma_2$ and $\eta \geq \lambda(T + K) + \lambda\epsilon$.
Furthermore, $\eta \geq \lambda (R + r) + \lambda\epsilon$, which similarly
guarantees that $N_{r, R}(\gamma_2) \intersection N_R(\gamma_2\restricted{[a_2,
b_2]})$ is a subset of $N_{r, R}(\gamma')$. Therefore $C_K(\gamma')
\intersection \Ball_T(\gamma'(c))$ lies in a single component of $A_{r, R,
K}(\gamma')$ by condition~\ref{noncutpairconnected}, so $A_{r, R, K}(\gamma')$
is connected by Lemma~\ref{lem:Adisconnectedlocally}, so $\Lambda\gamma'$ is a
non-cut pair by the results of section~\ref{sec:annulus}.

Now suppose that a feature of the second type exists in $X$. Let $\hat\gamma$
be the path obtained by concatenating $\gamma$ with vertical geodesic rays.
Then $\hat\gamma$ is an $(8\delta+1)$-local-geodesic since $k \geq 8\delta+1$.
$A_{r, R, K}(\hat\gamma)$ is connected by Lemma~\ref{lem:geodesicsinhoroballs},
so $\Lambda\hat\gamma$ is a non-cut pair by the results of
section~\ref{sec:annulus}.

Conversely, suppose that $\boundary X$ contains a non-cut pair. Let $\gamma'$
be an $(8\delta+1)$-local-geodesic in $X$ such that $\Lambda\gamma'$ is such a
pair. Assume first that $\gamma'$ is contained in $X_{k+R}$ and reparametrise
$\gamma'$ so that $\gamma'(0) \in X_0$. $C_K(\gamma') \intersection
\Ball_T(\gamma'(0))$ contains at most $V$ vertices and lies in a single component
of $N_{r, R}(\gamma')$. Define $n_V^\pm$ to be $\pm\lambda(K+T+\epsilon)$ and
then for $l$ decreasing from $V-1$ to $1$ let $n_l^+$ and $n_l^-$ be chosen to
minimise $n_l^+ - n_l^-$ among pairs such that $C_K(\gamma') \intersection
\Ball_T(\gamma'(0))$ meets at most $l$ components of $N_{r, R}(\gamma')
\intersection N_R(\gamma'\restricted{[n_l^-, n_l^+]})$ and $[n_{l+1}^-,
n_{l+1}^+] \subset [n_l^-, n_l^+]$.  Note that the condition that
$\mod{n^\pm_l} \geq \lambda(K + T) + \lambda\epsilon$ ensures that $C_K(\gamma')
\intersection \Ball_T(\gamma'(0))$ is a subset of $N_{r, R}(\gamma') \intersection
N_R(\gamma'\restricted{[n_l^-, n_l^+]})$.

Suppose that $n_{l - 1}^+ - n_l^+ > (k + R + 1)B^{2\eta} V^{V+1} + 2\eta$. Let
$\mathcal{Q}$ be the partition of $C_K(\gamma') \intersection \Ball_T(\gamma'(0))$
into $l$ non-empty subsets induced by connectivity in $N_{r,
R}(\gamma') \intersection N_R(\gamma'\restricted{[n_{l-1}^-, n_{l-1}^+-1]})$.
For each $t \in [n_l + \eta, n_{l-1} - \eta]$ define the following sets:
\begin{align*}
Y_t &= N_{r, R}(\gamma') \intersection \Ball_\rho(\gamma'(t)) \\
Z_t &= N_{r, R}(\gamma') \intersection \left(\Ball_\rho(\gamma'(t)) \union
  N_R(\gamma'\restricted{[n_{l-1}^-, t]})\right)
\end{align*}
and let $\mathcal{P}_t$ be the partition of the vertices of $Y_t$ into $l + 1$
subsets: $l$ corresponding to the $l$ sets in $\mathcal{Q}$ by connectivity in
$Z_t$ and one containing the part of $Y_t$ not connected to $C_K(\gamma')
\intersection \Ball_T(\gamma'(0))$ in $Z_t$.  As in the proof of
Proposition~\ref{prop:cutpairsfeature} there exist $s_1 < s_2$ in $[n^+_l + \eta,
n^+_{l-1} - \eta]$ such that such that
\begin{enumerate}
\item $h(\gamma'(s_1)) = h(\gamma'(s_2))$, so $\gamma'(s_2) = g\gamma'(s_1)$
  for some $g$ in $\Gamma$,
\item $\gamma'\restricted{[s_2-\eta,s_2+\eta]} = g\cdot\gamma'
  \restricted{[s_1-\eta,s_1+\eta]}$, which implies that $Y_{s_2} = g\cdot
  Y_{s_1}$, and
\item 
  $\mathcal{P}_{s_2} = g\cdot\mathcal{P}_{s_1}$. 
\end{enumerate}
Then replace $\gamma'$ by another path defined by
\begin{align*} \gamma''(t) = 
  \begin{dcases}
    \gamma'(t) & \text{if $t \leq s_1$}\\
    g^{-1}\cdot \gamma'(t + (s_2 - s_1)) & \text{if $t \geq s_2$} \\
  \end{dcases}
\end{align*}

This is an $(8\delta+1)$-local-geodesic since $\eta \geq 8\delta+1$. By
definition of $\rho$, the intersection of $Z_{s_2}$ and $N_{r, R}(\gamma')
\intersection N_R(\gamma'\restricted{[s_2, n_{l-1}^+]})$ is contained in
$Y_{s_2}$. Then by definition of $n_{l-1}^+$, two distinct sets in
$\mathcal{P}_{s_2}$ meet $N_{r, R}(\gamma') \intersection
N_R(\gamma'\restricted{[s_2, n_{l-1}^+]})$. Since $\eta \geq \lambda(R + r +
\epsilon)$,
\begin{align*}
  g^{-1}N_{r, R}(\gamma') \intersection N_R(\gamma'\restricted{[s_2, n_{l-1}^+]})
  = N_{r, R}(\gamma'') \intersection N_R(\gamma'\restricted{[s_1, n_{l-1}^+ -
  (s_2 - s_1)]}),
\end{align*}
and it follows that $C_K(\gamma'') \intersection \Ball_T(\gamma'(0))$ meets at
most $l-1$ components of $N_{r, R}(\gamma'') \intersection
N_R(\gamma''\restricted{[n_{l-1}^-, n_{l-1}^+ - (s_2 - s_1)]})$. This implies
that the process of replacing $\gamma'$ by $\gamma''$ leaves unchanged
$n_{l'}^+$ for all $l' < l$ and $n_{l'}^-$ for all $l'$ and strictly reduces
$n_l^+$.  Therefore by repeating this process we can assume that $\gamma'$ was
chosen to ensure that $\mod{n_{l-1}^\pm - n_l^\pm}$ is at most
$(k+R+1)B^{2\eta}V^{V+1} + 2\eta$ for all $l$, and therefore that 
\begin{align*}
  \mod{n_1^\pm} \leq \lambda(K + T + \epsilon) + (V-1)((k+R+1)B^{2\eta}V^{V+1} + 2\eta).
\end{align*}

There exist $a_1 \leq b_1 \leq n_1^- - \eta$ with $n_1^- - b_1$ and $b_1 - a_1$
both at most $(k+R+1)B^{2\eta} + 1$ such that 
\begin{enumerate}
\item $h(\gamma'(b_1)) = h(\gamma'(a_1))$, so $\gamma'(b_1) = g_1\cdot
  \gamma'(a_1)$ for some $g_1 \in \Gamma$.
\item $\gamma'\restricted{[b_1 - \eta, b_1 + \eta]} =
  g_1\cdot\gamma'\restricted{[a_1 - \eta, a_1 + \eta]}$.
\end{enumerate}
Then let $\gamma_1 = \gamma'\restricted{[a_1 - \eta, b_1+\eta]}$. Similarly
define $b_3 \geq a_3 \geq n_1^+$ and let $\gamma_3 = \gamma'\restricted{[a_3 -
\eta, b_3 + \eta]}$. Let $a_2 = b_1$ and $b_2 = a_3$ and let $\gamma_2 =
\gamma'\restricted{[a_2 - \eta, b_2 + \eta]}$; note that $b_2 - a_2 \leq N_1$.
Then the triple $(\gamma_1, \gamma_2, \gamma_3)$ is a feature in $X$ of the
first kind listed in the proposition: conditions~\ref{noncutpairshortout},
\ref{noncutpairshortin}, \ref{noncutpairdepth}, \ref{noncutpairlocal1}
and~\ref{noncutpairlocal2} clearly hold by construction and
condition~\ref{noncutpairconnected} holds because the condition that $\eta \geq
\lambda(R + r) + \lambda\epsilon$ ensures that $N_{r, R}(\gamma') \intersection
N_R(\gamma'\restricted{[n_1^-, n_1^+]})$ is a subset of $N_{r, R}(\gamma_2)
\intersection N_R(\gamma_2\restricted{[a_2, b_2]})$.

If $\gamma'$ is not contained in $X_{k+R}$ let $\gamma'^{-1}h^{-1}[0, k+R]$ be
a (possibly infinite) union of (possibly infinite) intervals $\union_{i \in
I}[a_i-R, b_i+R]$ where we order the intervals so that $a_{i+1} > b_i$ for all
$i$. Then $h(\gamma'(a_i)) = h(\gamma'(b_i)) = k$ for all $i$ and we can apply
the results of section~\ref{sec:horoballs}.  For $i \in I$ let $\gamma'_i =
\gamma'\restricted{[a_i, b_i]}$. Let $\hat{\gamma}'_i$ be the bi-infinite
$(8\delta + 1)$-local-geodesic obtained by concatenating $\gamma_i$ with
either one or two vertical geodesic rays.

Suppose that $\Lambda\hat\gamma'_i$ is a cut pair for all $i$. Then
Lemma~\ref{lem:equivalenceclass} tells us that $\Lambda\gamma'$ is a cut pair,
which is a contradiction.  Therefore there exists $i$ such that
$\Lambda\hat\gamma_i$ is a non-cut pair. 

Arguing as before, the geodesic $\hat\gamma_i$ can be altered to ensure that
$b_i - a_i \leq 4\eta + 2(k+R+1)B^{2\eta}V^{V+1}$; this yields a feature of the
second type described in the proposition.  \end{proof}

From this we deduce the following corollary:

\begin{cor}\label{cor:noncutpaircomputable} There is an algorithm that takes as
input a presentation for a hyperbolic group $\Gamma$ with generating set $S$,
a list of subsets of $S$ generating a collection $\mathcal{H}$ of maximal virtually cyclic
subgroups of $\Gamma$ and integers $\delta$, $\delta_\mathcal{H}$ and $n$ such that the
cusped space associated to $(\Gamma, \mathcal{H}, S)$ is $\delta$-hyperbolic
and satisfies $\ddag_n$ and such that the Cayley graph of each element of
$\mathcal{H}$ with respect to its given generating set is
$\delta_\mathcal{H}$-hyperbolic and returns the answer to the question ``does
$\boundary(\Gamma, \mathcal{H})$ contain a non-cut pair?'' \qed \end{cor}

