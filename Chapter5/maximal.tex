\section{Maximal splittings}\label{sec:maximal}

We now apply the technical results of sections~\ref{sec:cutpairsalgorithms}
and~\ref{sec:fuchsiangroups} to the problem of finding a maximal splitting of a
one-ended hyperbolic group.

\subsection{JSJ decompositions}

We begin by recalling the definition a JSJ decomposition. For a more detailed
account see~\cite{guirardellevitt16}. Let $\Gamma$ be a group and let
$\mathcal{A}$ be a collection of subgroups of $\Gamma$ that is closed under
conjugation and taking subgroups. Let $\mathcal{H}$ be another collection of
subgroups of $\Gamma$. An \emph{$(\mathcal{A}, \mathcal{H})$-tree} is a tree
with a $\Gamma$-action with no edge inversions in which the stabiliser of each
edge is in $\mathcal{A}$ and each element of $\mathcal{H}$ is \emph{elliptic},
that is, it fixes a point in the tree. An $(\mathcal{A}, \mathcal{H})$-tree $T$
\emph{dominates} another such tree $T'$ if there is a $\Gamma$-equivariant
morphism from $T$ to $T'$.  $T$ is \emph{elliptic with respect to $T'$} if the
stabiliser of each edge in $T$ is elliptic in $T'$. For $v$ a vertex of $T$ we
denote by $\Gamma_v$ the stabiliser of $v$, and similarly if $e$ is an edge of
$T$ we let $\Gamma_e$ denote the stabiliser of $e$. We will sometimes refer to
a $(\mathcal{A}, \mathcal{H})$-tree as a \emph{splitting of $\Gamma$ over
$\mathcal{A}$ relative to $\mathcal{H}$}. A splitting is \emph{trivial} if
$\Gamma$ is elliptic. We will frequently implicitly move between the languages
of group actions on trees and graphs of groups.

Fixing $\mathcal{A}$ and $\mathcal{H}$, a subgroup of $\Gamma$ is
\emph{universally elliptic} if it is elliptic with respect to any
$(\mathcal{A}, \mathcal{H})$-tree. An $(\mathcal{A}, \mathcal{H})$-tree is
universally elliptic if the stabiliser of each if its edges is universally
elliptic. An $(\mathcal{A}, \mathcal{H})$ tree is a \emph{JSJ tree} if it is
universally elliptic and is maximal for domination among universally elliptic
$(\mathcal{A}, \mathcal{H})$ trees. 

\subsection{The Bowditch JSJ}

In~\cite{bowditch98} Bowditch defines a canonical JSJ tree $\Sigma$ for a
one-ended hyperbolic group $\Gamma$ where $\mathcal{A}$ is the set of virtually
cyclic subgroups of $\Gamma$ and $\mathcal{H}$ is empty. (To ensure that
$\mathcal{A}$ is closed under taking subgroups we should strictly speaking also
include the finite subgroups of $\Gamma$ in $\mathcal{A}$. However, since
$\Gamma$ is assumed to be one-ended it does not split over any finite subgroup,
so this change is not important.) We recall a description of the tree $\Sigma$ here.

$\Sigma$ is a tree with a three-colouring: its vertex set $V(\Sigma)$ admits a
partition $V_1(\Sigma) \disjointunion V_2(\Sigma) \disjointunion V_3(\Sigma)$
preserved by the action of $\Gamma$ such that no two vertices in either
$V_1(\Sigma)$ or $V_2(\Sigma) \union V_3(\Sigma)$ are adjacent.

The stabiliser of a vertex in $V_1(\Sigma)$ is a maximal virtually cyclic subgroup of
$\Gamma$ and therefore contains the stabiliser of each incident edge at finite
index.

To describe the vertices of the second type we require the following
definition. Recall definition~\ref{defn:boundedFuchsian} of a bounded Fuchsian
group.

\begin{defn} A \emph{hanging Fuchsian} subgroup $Q$ of $\Gamma$ is a subgroup
of $\Gamma$ that is the stabiliser of a vertex in some finite splitting of
$\Gamma$ over virtually cyclic subgroups such that $Q$ admits an isomorphism
with a bounded Fuchsian group that maps the stabilisers of incident
edges precisely to the peripheral subgroups.\end{defn}

Stabilisers of vertices in $V_2(\Sigma)$ are precisely the maximal hanging
Fuchsian subgroups of $\Gamma$ and the stabilisers of incident edges at such a
vertex in the JSJ decomposition are precisely the peripheral subgroups referred
to in the definition.

The stabiliser of a vertex in $V_3(\Sigma)$ is not virtually cyclic and is not
a hanging Fuchsian subgroup.

\subsection{Splittings and the topology of the boundary}\label{sec:splittingstopology}

In this section we recall and extend some results linking the existence of a
non-trivial splitting of a hyperbolic group relative to a collection of
virtually cyclic subgroups to the topology of a particular Bowditch boundary. 

For the remainder of Section~\ref{sec:splittingstopology}, we fix a hyperbolic
group $\Gamma$ and a collection $\mathcal{H}$ of virtually cyclic subgroups. If
$H$ is any virtually cyclic subgroup of $\Gamma$, let $\widehat{H}$ be the unique
maximal virtually cyclic subgroup of $\Gamma$ containing $H$. Then let
$\widehat{\mathcal{H}}$ be $\{\widehat{H} \vert H \in \mathcal{H}\}$. Recall
Lemma~\ref{lem:vcycperipheral}: since each group in $\widehat{\mathcal{H}}$ is
maximal virtually cyclic, $\Gamma$ is hyperbolic relative to
$\widehat{\mathcal{H}}$. We may therefore study the boundary $\boundary(\Gamma,
\widehat{\mathcal{H}})$.

\subsubsection{Boundaries with cut points}

We first recall results that deal with the case in which $\boundary(\Gamma,
\widehat{\mathcal{H}})$ contains a cut point. First recall
Bowditch's definition~\cite{bowditch01} of a \emph{peripheral splitting}.

\begin{defn} If $\Gamma$ is a group and $\mathcal{H}$ is a finite collection of
subgroups of $\Gamma$, a \emph{peripheral splitting} of $\Gamma$ with respect
to $\mathcal{H}$ is a representation of $\Gamma$ is a finite bipartite graph of
groups such that each vertex group of one colour is conjugate to an element of
$\mathcal{H}$, and each element of $\mathcal{H}$ is conjugate to a vertex group
of that colour.

As for any splitting, we say that a peripheral splitting is \emph{trivial} if
some vertex group is equal to $\Gamma$.\end{defn}

Note that, in our setting, a peripheral splitting of $\Gamma$ with respect to
$\widehat{\mathcal{H}}$ is a splitting of $\Gamma$ over virtually cyclic subgroups
relative to $\mathcal{H}$.

Then the following proposition, which we obtain by putting together two
theorems of Bowditch, completes our treatment of the case in which the boundary
contains a cut point.

\begin{prop}\label{prop:cutpointimpliessplitting} Suppose that
  $\boundary(\Gamma, \widehat{\mathcal{H}})$ is connected and contains a cut point.
Then the pair $\Gamma$ admits a non-trivial splitting over virtually cyclic
subgroups relative to $\mathcal{H}$.
\end{prop}

\begin{proof} By~\cite[Theorem 0.2]{bowditch99a} the global cut
  point of $\boundary(\Gamma, \widehat{\mathcal{H}})$ is a parabolic fixed point.
By~\cite[Theorem 1.2]{bowditch99b} $\Gamma$ admits a non-trivial peripheral
splitting with respect to $\widehat{\mathcal{H}})$. The edge groups in this splitting
are automatically virtually cyclic, since each edge meets a vertex with vertex
group conjugate into $\widehat{\mathcal{H}}$, and the splitting is automatically
relative to $\mathcal{H}$: in fact it is relative to $\widehat{\mathcal{H}}$.
\end{proof}

\subsubsection{Boundaries with cut pairs}

In the absence of cut points, the existence of a relative splitting is
reflected in the existence of cut pairs in the boundary.  In the absolute case,
recall the following theorem of~\cite{bowditch98}.

\begin{thm}\cite[Theorem 6.2]{bowditch98}\label{thm:cutpairimpliessplitting}
Let $\Gamma$ be a one-ended hyperbolic group such that $\boundary(\Gamma,
\emptyset)$ contains a cut pair and is not homeomorphic to $S^1$. Then $\Gamma$
admits a non-trivial splitting over a virtually cyclic subgroup.  \end{thm}

We require a relative version of this theorem. We only require such a result in the case
when $\Gamma$ arises as a vertex group in a splitting of a larger group over
virtually cyclic subgroups, and $\mathcal{H}$ is the collection of edge groups
incident at that vertex. In this section we show that in this simple case the
relative result follows from Theorem~\ref{thm:cutpairimpliessplitting}, and one
can avoid $\reals$-trees machinery. For a discussion in greater generality,
see~\cite{groff13}.

\begin{prop}\label{prop:cutpairimpliesrelativesplitting} Let $v$ be a vertex in
a minimal $\Gamma$-tree $T$ with virtually cyclic edge groups where $\Gamma$
is hyperbolic and one-ended. Let $\inc v$ be a set of representatives of
$\Gamma_v$-conjugacy classes of stabilisers of edges in $T$ incident at $v$.
Suppose that $\boundary(\Gamma_v, \inc v)$ is not a single point, does not
contain a cut point and is not homeomorphic to a circle but does contain a
cut pair. Then $\Gamma_v$ admits a non-trivial splitting over virtually cyclic
subgroups of $\Gamma_v$ relative to $\inc v$.  \end{prop}

First we need the following lemma. Recall that for a vertex $v$ in a
$\Gamma$-tree, we defined $\inc v$ to be a set of conjugacy class
representatives of the stabilisers of the edges of the $\Gamma$-tree incident
at $v$, and let $\widehat{\inc v}$ be the set of maximal virtually cyclic
subgroups of $\Gamma_v$ that contain the elements of $\inc v$.

\begin{lem}\label{lem:fixedcomponent} Let $f \colon T_1 \to T_2$ be an
  equivariant map of $\Gamma$-trees with virtually infinite cyclic edge
  stabilisers such that the action of $\Gamma$ on $T_1$ is cocompact and the
  action of $\Gamma$ on $T_2$ is minimal.  (This means that there is no proper
  $\Gamma$-invariant subtree of $T_2$.) Let $v$ be a vertex of $T_2$ such that
  $\boundary(\Gamma_v, \widehat{\inc v})$ is not a single point, is connected
  and does not contain a cut point. Then the action of $\Gamma_v$ on $T_1$
  fixes a component of $f^{-1}(v)$.\end{lem}

\begin{proof} First we show that there is a vertex $w \in T_1$ such that
$\Gamma_w \intersection \Gamma_v$ is non-elementary. If this is not the case
then the stabiliser of each edge of $T_1$ with respect to the action of
$\Gamma_v$ on $T_1$ is either finite or commensurable with the stabilisers of
its end points. Therefore the action induces a splitting of $\Gamma_v$ with
finite edge groups and virtually cyclic vertex groups. By minimality of the
action of $\Gamma$ on $T_2$ each edge $e$ incident at $v$ is $f(e')$ for some
edge $e'$ of $T_1$, and then $\Gamma_{e'}$ is a finite index subgroup of
$\Gamma_e$, since each is virtually infinite cyclic. In particular,
$\widehat{\Gamma_e}$ is elliptic, and the splitting is relative to $\widehat{\inc v}$.
But $\boundary(\Gamma_v, \widehat{\inc v})$ was assumed to be connected and not a
single point, which is a contradiction.

Then $f(w) = v$, otherwise any edge separating $f(w)$ from $v$ in $T_2$ has
non-elementary stabiliser. Let $S$ be the component of $f^{-1}(v)$ containing
$w$. We now show that any other vertex $w'$ of $T_1$ such that $\Gamma_{w'}
\intersection \Gamma_v$ is non-elementary is also in $S$. Suppose that $e$ is
an edge of $T_1$ that is not in $f^{-1}(v)$. As in Section~1
of~\cite{bowditch98} there exists a partition of $\boundary(\Gamma, \emptyset) -
\Lambda\Gamma_e$ as $U_1 \disjointunion U_2$.  The intersection of $\Gamma_e$
with $\Gamma_v$ is either finite or commensurable with a conjugate of an
element of $\widehat{\inc v}$, so the images of $U_1 \intersection
\Lambda\Gamma_v$ and $U_2 \intersection \Lambda\Gamma_v$ under the projection
map $\Lambda\Gamma_v \to \boundary(\Gamma, \widehat{\inc v})$ cover all but at
most a point of $\boundary(\Gamma, \widehat{\inc v})$. These sets are disjoint,
so one must be empty, say $U_2$. But $\Lambda\Gamma_w$ and $\Lambda\Gamma_{w'}$
each contain more than two points, so must both be contained in $U_1 \union
\Lambda\Gamma_e$. This implies that $w$ and $w'$ are on the same side of $e$.

Therefore the action of $\Gamma_v$ on $T_1$ fixes $S$, for any element of
$\Gamma_v$ must send $w$ to a vertex of $S$.\end{proof}

\begin{proof}[Proof of Proposition~\ref{prop:cutpairimpliesrelativesplitting}]
Let $\Sigma$ be Bowditch's JSJ tree for $\Gamma$. $\Sigma$ is
then elliptic with respect to $T$, so by~\cite[Proposition
2.2]{guirardellevitt16} there exists a $\Gamma$-tree $\widehat\Sigma$ and maps $p
\colon \widehat\Sigma \to \Sigma$ and $f \colon \widehat\Sigma \to T$ such that:
\begin{enumerate}
  \item $p$ is a collapse map. (i.e.\ a map given by collapsing some edges of
    $\widehat\Sigma$ to points.)
  \item For $w \in \Sigma$, the restriction of $f$ to $p^{-1}(w)$ is injective.
\end{enumerate}

Let $S \subset \widehat\Sigma$ be the component of $f^{-1}(v)$ fixed by the action
of $\Gamma_v$ constructed in Lemma~\ref{lem:fixedcomponent}. Suppose that a
vertex $w$ of $S$ is fixed by the $\Gamma_v$-action.

If $e$ is any edge of $\widehat\Sigma$ that is adjacent to but not contained in $S$
then $\Gamma_e \leq \Gamma_{f(e)}$ and the subgroup is necessarily of finite
index. Conversely the stabiliser of any edge incident at $v$ contains the
stabiliser of an edge adjacent to $S$ at finite index. The stabiliser of any
edge adjacent to $S$ is then commensurable with the stabiliser of an edge
incident at $w$, and vice versa. Therefore $\boundary(\Gamma_v, \widehat{\inc w})$ is
homeomorphic to $\boundary(\Gamma_v, \widehat{\inc v})$. We assumed that
$\boundary(\Gamma_v, \widehat{\inc v})$ was neither a point nor homeomorphic to a circle,
so $pw$ is not in $V_1(\Sigma)$ or $V_2(\Sigma)$, and is therefore in
$V_3(\Sigma)$.

Let $x$ and $y$ be points in $\boundary(\Gamma_v, \widehat{\inc v})$ and choose preimages
$\widetilde{x}$ and $\widetilde{y}$ in $\boundary(\Gamma_v, \emptyset)$, which
we identify with $\Lambda\Gamma_v \subset \boundary(\Gamma, \emptyset)$.  
The set of components of $\boundary(\Gamma_v, \widehat{\inc v}) - \{x, y\}$ is in
bijection with the set of those components of $\boundary(\Gamma, \emptyset) -
\{\widetilde{x}, \widetilde{y}\}$ that meet $\Lambda\Gamma_v$. 

Suppose then
that $\{x, y\}$ is a cut pair in $\boundary(\Gamma_v, \widehat{\inc v})$, so at least two
components of $\boundary(\Gamma, \emptyset) - \{\widetilde{x}, \widetilde{y}\}$ meet
$\Lambda\Gamma_v$.

Then by Theorem~\ref{thm:cutpairimpliessplitting} there is a type 1 or type 2
vertex $u$ of $\Sigma$ such that $\Lambda\Gamma_u$ contains $\{\widetilde{x},
\widetilde{y}\}$. Then $\{\widetilde{x}, \widetilde{y}\} = \Lambda\Gamma_u
\intersection \Lambda\Gamma_v$, so $\{\widetilde{x}, \widetilde{y}\}$ is the
limit set of an edge incident at $v$. Therefore $\{x, y\} \subset
\boundary(\Gamma_v, \widehat{\inc v})$ is a single point, which is a contradiction
because $\boundary(\Gamma_v, \widehat{\inc v})$ was assumed not to contain a cut
point. Hence the action of $\Gamma_v$ on $S$ does not fix any vertex and
therefore gives rise to a non-trivial splitting of $\Gamma_v$ relative to $\inc v$.

\end{proof}

\subsubsection{Boundaries without cut points or pairs}

Our description of the relationship between the existence of splittings and the
topology of the boundary is completed by the following proposition, which serves as a
converse to Propositions~\ref{prop:cutpointimpliessplitting}
and~\ref{prop:cutpairimpliesrelativesplitting}.

\begin{prop}\label{prop:relativesplittingimpliescutpair} Let $\Gamma$ be a
  hyperbolic group and let $\mathcal{H}$ be a finite set of virtually cyclic
  subgroups of $\Gamma$ such that $\boundary(\Gamma, \widehat{\mathcal{H}})$ is
    connected.  Suppose that $\Gamma$ admits a non-trivial splitting over a
  virtually cyclic subgroup relative to $\mathcal{H}$.  Then $\boundary(\Gamma,
\widehat{\mathcal{H}})$ contains a cut point or pair.\end{prop}

\begin{proof} Let $T$ be the $\Gamma$-tree associated to such a non-trivial splitting.
Without loss of generality assume that the action of $\Gamma$ on $T$ is
minimal. Let $e$ be any edge in $T$. Let $T_1$ and $T_2$ be the two
components of the complement of the interior of $e$ in $T$. Then as in the proof of Lemma~\ref{lem:fixedcomponent}
 we obtain a partition of $\boundary(\Gamma,
\emptyset) - \Lambda\Gamma_e$ as $U_1\disjointunion U_2$ where $U_i$ are open
sets given by
\begin{align*}
  U_i = \boundary T_i \union \bigunion_{w \in T_i} (\Lambda\Gamma_w - \Lambda\Gamma_e)
\end{align*}

If a subgroup $H$ of $\Gamma$ is in $\mathcal{H}$, $H \leq \Gamma_w$ for some vertex $w \in \Sigma$ and either $\Lambda H =
\Lambda\Gamma_e$ or $\Lambda H \intersection \Lambda\Gamma_e = \emptyset$.
In the latter case either
$\Lambda H \subset U_1$ or $\Lambda H \subset U_2$. Let $\pi \colon
\boundary(\Gamma, \emptyset) \to \boundary(\Gamma, \widehat{\mathcal{H}})$ be the
quotient projection of Lemma~\ref{lem:bowditchfromgromov}. It follows that the images
of $U_1$ and $U_2$ under $\pi$ in the complement of $\pi(\Lambda\Gamma_e)$ in
$\boundary(\Gamma, \widehat{\mathcal{H}})$ are disjoint open sets; they are non-empty by
the minimality of the action of $\Gamma$ on $T$. The image of $\Lambda\Gamma_e$
is either one or two points, and therefore $\boundary(\Gamma, \widehat{\mathcal{H}})$
contains a cut point or pair. \end{proof}

\subsection{Virtually cyclic subgroups and finding splittings}

We will need the following lemma, which allows us to do various computations
related to virtually cyclic subgroups of hyperbolic groups.

\begin{lem}\cite[Lemma 2.8]{dahmaniguirardel11}\label{lem:twoendedsubgroups}
There is an algorithm that, when given a presentation for a hyperbolic group
$\Gamma$ and a finite subset $S \subset \Gamma$, returns an answer to the
question ``is $\langle S \rangle \leq \Gamma$ virtually cyclic?'' If the
answer is ``yes'' then the algorithm also determines
\begin{enumerate}
\item the (unique) maximal finite normal subgroup of $\langle S \rangle$, 
\item a presentation for $\langle S\rangle$,
\item whether $\langle S \rangle$ is of type $\mathcal{Z}$ or $D_\infty$.
  (Recall that we say that a virtually cyclic group of type $\mathcal{Z}$
  (respectively $D_\infty$) if it surjects onto $\integers$ (respectively
  $D_\infty$).)
\item a generating set for the maximal virtually cyclic subgroup of $\Gamma$
  containing $\langle S\rangle$.\end{enumerate}
\end{lem}

The proof of this lemma in~\cite{dahmaniguirardel11} uses Makanin's algorithm
for solving equations in hyperbolic groups. We modify that part of the argument
to use only elementary methods in keeping with the themes of this paper.

\begin{proof} We give an alternative method to determine whether or not
$\langle S\rangle$ is virtually cyclic and to produce a maximal finite normal
subgroup of $\langle S\rangle$ in the case that it is; the rest of the
argument can be copied verbatim from~\cite{dahmaniguirardel11}. First compute
$\delta$ with respect to which $\Gamma$ is $\delta$-hyperbolic.

Use the algorithm of~\cite[Proposition 4]{kapovich96} to search for a constant
$K$ with respect to which $\langle S\rangle$ is $K$-quasi-convex in $\Gamma$.
This algorithm finds such a constant if it exists and does not terminate of
$\langle S\rangle$ is not quasi-convex; note that if $\langle S\rangle$ is
virtually cyclic then it is guaranteed to be quasi-convex. If the algorithm
terminates use $K$ and $\delta$ to compute $\delta'$ such that $\langle
S\rangle$ is $\delta'$-hyperbolic.  Then all finite subgroups of $\langle
S\rangle$ can be conjugated into a ball of radius at most $4\delta' + 2$ with
respect to the word metric in $\langle S\rangle$, so all finite normal
subgroups of $S$ can be computed using a solution to the word and conjugacy
problems in $\Gamma$. Once this is computed the algorithm
of~\cite{dahmaniguirardel11} can be used to determine whether or not $\langle
S\rangle$ is virtually cyclic.

In parallel, search for a pair of elements $g$ and $h$ in $\langle S\rangle$
such that $[g^2, h^2]$ has infinite order. This can be checked since the order
of an element of $\Gamma$ of finite order is bounded above by the number of
elements of $\Gamma$ in the ball of radius $4\delta + 2$.

If $\langle S\rangle$ is not quasi-convex then it contains a free group on two
generators, so a pair $(g, h)$ as in the previous paragraph certainly exists.
Conversely, if such a pair exists then $\langle S\rangle$ cannot be virtually cyclic,
since any virtually cyclic group contains a subgroup of index two that surjects onto
$\integers$ with finite kernel.  \end{proof}

\begin{lem}\label{lem:findingasplitting} There is an algorithm that takes as
  input a presentation for a hyperbolic group $\Gamma$ and a collection
  $\mathcal{H}$ of peripheral subgroups and either returns the graph of groups
  associated to a non-trivial splitting of $\Gamma$ relative to $\mathcal{H}$
over virtually cyclic subgroups or does not terminate if no such splitting
exists.  \end{lem}

\begin{proof} Suppose that $\Gamma$ admits a proper splitting relative to
  $\mathcal{H}$ as an amalgamated product over a virtually cyclic subgroup; the
  case of an HNN extension is similar. Then $\Gamma$ admits a presentation that
  makes this splitting explicit in the following sense. There are finite
  disjoint symmetric sets of symbols $S_1$, $S_2$ and $S_3$, finite subsets
  $R_1$, $R_2$ and $R_3$ of the free monoids $S_1^\star$, $S_2^\star$ and
  $S_3^\star$ respectively, and maps $\iota_1$ and $\iota_2$ from $S_3^\star$
  to $S_1^\star$ and $S_2^\star$ respectively, such that $\Gamma$ admits an
  isomorphism to the group $\mod{\mathcal{P}}$ with presentation $\mathcal{P}$ of the form
\begin{align*}
  \langle S_1 \union S_2 \union S_3 \vert R_1 \union R_2 \union R_3 \union
    \{s^{-1}\iota_1(s), s^{-1}\iota_2(s) \colon s \in S_3\}\rangle
\end{align*}
where $\{\iota_1(r) \vert r \in R_3\} \subset R_1$ and $\{\iota_2(r) \vert r
\in R_3\} \subset R_2$. This ensures that $\iota_1$ and $\iota_2$ induce group
homomorphisms from $\langle S_3 \vert R_3\rangle$ to $\langle S_1 \vert
R_1\rangle$ and  $\langle S_2 \vert R_2\rangle$, which we shall denote
$\widehat\iota_1$ and $\widehat\iota_2$. Let $\widehat\iota$ be the induced map from
$\langle S_3 \vert R_3\rangle$ to $\Gamma$. Then the assumptions on the nature
of the splitting give the following conditions:
\begin{enumerate}
\item that $\langle S_3 \vert R_3 \rangle$ be virtually cyclic,
\item that $\widehat\iota_1$ and $\widehat\iota_2$ be injective,
\item that $\widehat\iota_1$ and $\widehat\iota_2$ not be surjective and
\item that for each $H \in \mathcal{H}$ with generating set $S_H$ there exists
  $g_H$ in $\Gamma$ such that either $S_1$ or $S_2$ contains the image of
  $g_HS_Hg_H^{-1}$ under the isomorphism from $\Gamma$ to $\mod{\mathcal{P}}$.
\end{enumerate}

We show that there is an algorithm that finds such a presentation for
$\Gamma$ if it exists. Using Tietze transformations, there is an algorithm that
enumerates all presentations of $\Gamma$ and for each presentation gives an
explicit isomorphism from $\Gamma$ to the realisation of that presentation;
therefore it is sufficient to show that each of the four conditions above can
be checked algorithmically.

The first condition can be checked using the first part of the algorithm of
Lemma~\ref{lem:twoendedsubgroups}.

$\widehat\iota_1$ and $\widehat\iota_2$ are both injective if and only if $\widehat\iota$
is injective: one direction is trivial, the other is a consequence of the
normal form theorem for the amalgamated product. To check this condition,
compute the maximal finite normal subgroup of $\langle S_3\vert R_3\rangle$ and
check the triviality of the image under $\widehat\iota$ of each of these elements.
Then find an element of $\langle S_3\vert R_3\rangle$ of infinite order and
check whether or not the image under $\widehat\iota$ of this element has infinite
order.

To check the surjectivity of $\widehat\iota_1$ first check whether or not $\langle
S_1 \rangle$ is virtually cyclic. If it is, check whether or not the maximal finite
normal subgroup of $\im\widehat\iota_1$ is equal to the maximal finite normal
subgroup of $\langle S_1 \rangle$. If it is, next check whether or not $\langle
S_1 \rangle$ and $\langle S_3 \rangle$ are either both of $\mathcal{Z}$-type or
both of $D_\infty$ type. If they are both of the same type, pass to an index 2
subgroup of each if necessary to ensure that they are both of $\mathcal{Z}$
type, then check whether or not the composition of $\widehat\iota_1$ with the
natural surjection to $\integers$ is surjective. If it is then $\widehat\iota_1$ is
surjective; if any of these tests produced the opposite answer then
$\widehat\iota_1$ is not surjective. Repeat this process for $\widehat\iota_2$.

The final condition can be checked using a solution to the simultaneous
conjugacy problem in $\Gamma$.

The existence of such a presentation for $\Gamma$ guarantees that $\Gamma$
splits non-trivially as an internal amalgamated product 
\begin{align*}
  \Gamma \isom \langle S_1 \vert R_1 \rangle \freeprod_{\langle S_3 \vert R_3
    \rangle} \langle S_2 \vert R_2 \rangle.
\end{align*}
This splitting is over a virtually cyclic
subgroup of $\Gamma$ and is relative to $\mathcal{H}$. 
\end{proof}

\subsection{Computing a maximal splitting}

We will need to be able to determine algorithmically whether or not the
boundary of the given hyperbolic group is homeomorphic to a circle. This is
achieved using the algorithm of Corollary~\ref{cor:noncutpaircomputable} and a
theorem from point-set topology, which we note here.

\begin{thm}\cite[II.2.13]{wilder49}\label{thm:topologyimpliescircle} Any
separable, connected, locally connected space containing more than one point
that is without a cut point and in which every pair is a cut pair is
homeomorphic to $S^1$.\end{thm}

\begin{prop}\label{prop:isthereasplitting} There is an algorithm that takes as
  input a presentation for a hyperbolic group $\Gamma$ with a collection
  $\mathcal{H}$ of virtually cyclic subgroups such that $\boundary(\Gamma,
  \widehat{\mathcal{H}})$ is connected and $\Gamma$ appears as the stabiliser
  of some vertex in the action of a hyperbolic group on a tree and
  $\mathcal{H}$ is a set of conjugacy class representatives of incident edge
  groups and returns the answer to the question ``does $\Gamma$ split
non-trivially over a virtually cyclic subgroup relative to
$\mathcal{H}$?''\end{prop}

\begin{figure}
\centering
\begin{tikzpicture}[node distance = 2cm, auto]

\tikzstyle{decision} = [diamond, draw,
  text width=4.5em, text badly centered, node distance=3cm, inner sep=0pt]
\tikzstyle{startblock} = [rectangle, draw, 
  text width=5em, text centered, rounded corners, minimum height=4em]
\tikzstyle{acceptblock} = [rectangle, draw,
  text width=5em, text centered, rounded corners, minimum height=4em]
\tikzstyle{rejectblock} = [rectangle, draw,
  text width=5em, text centered, rounded corners, minimum height=4em]
\tikzstyle{line} = [draw, -latex']

\node [startblock] (start) {Start};
\node [decision, below of=start] (vcyclic) {Virtually cyclic?};
\node [decision, right of=start] (ddag) {$\ddag(n, 0)$ for some $n$?};
\node [decision, below of=ddag] (cutpoint) {Cut point?};
\node [acceptblock, right of=ddag, node distance=6cm] (splitting) {Splitting};
\node [decision, below of=cutpoint] (cutpair) {Cut pair?};
\node [decision, right of=cutpair, node distance=6cm] (small) {Small orbifold?};
\node [rejectblock, left of=cutpair, node distance=3cm] (nosplitting) {No splitting};

\path [line] (start) -- (vcyclic);
\path [line] (vcyclic) --  node {No} (ddag);
\path [line] (vcyclic) --  node {Yes} (nosplitting);
\path [line] (ddag) --  node {No} (splitting);
\path [line] (ddag) --  node {Yes} (cutpoint);
\path [line] (cutpoint) -- node {No} (cutpair);
\path [line] (cutpoint) -- node {Yes} (splitting);
\path [line] (cutpair) --  node {No} (nosplitting);
\path [line] (cutpair) --  node {Yes} (small);
\path [line] (small) --  node [near start] {No} (splitting);
\path [line] (small.south) -- ++(0,-.3cm) node[pos=.3] {Yes} -- ++(-9cm,0) -- (nosplitting);

\end{tikzpicture}


\caption{The decision process in the algorithm of Proposition~\ref{prop:isthereasplitting}.}
\end{figure}

\begin{proof} Let the given group be $\Gamma$ and the peripheral structure be
$\mathcal{H}$. First check whether or not $\Gamma$ is virtually cyclic. If it
is then $\Gamma$ does not split properly over a virtually cyclic subgroup.
If it is not then compute $\widehat{\mathcal{H}}$; then $\boundary(\Gamma,
\widehat{\mathcal{H}})$ contains more than a single point and the results
of this section can be applied.

Next compute $\delta$ such that the cusped space $X$ associated to the pair
$(\Gamma, \widehat{\mathcal{H}})$ is $\delta$-hyperbolic.  Search for a
non-trivial splitting of $\Gamma$ relative to $\mathcal{H}$ using
Lemma~\ref{lem:findingasplitting} and, in parallel, search for $n$ such that
$\ddag_n$ holds in $X$; one of these processes must terminate by
Proposition~\ref{prop:Xsatisfiesddag} and
Proposition~\ref{prop:cutpointimpliessplitting}.

If a splitting is found then $\Gamma$ does split non-trivially over a virtually
cyclic subgroup relative to $\mathcal{H}$, and the algorithm can return
``yes''. If $X$ satisfies $\ddag_n$ then use the algorithm of
Corollary~\ref{cor:cutpaircomputable} to check whether or not
$\boundary(\Gamma, \widehat{\mathcal{H}})$ contains a cut point. If it does then $\Gamma$
does split properly over a virtually cyclic subgroup by
Proposition~\ref{prop:cutpointimpliessplitting}

If there is no cut point, use the algorithm of
Corollary~\ref{cor:cutpaircomputable} on $X$ to determine whether or
$\boundary(\Gamma, \widehat{\mathcal{H}})$ contains a cut pair; if it does not then
$\Gamma$ does not split relative to $\mathcal{H}$ by
Proposition~\ref{prop:relativesplittingimpliescutpair}.

If there is a cut pair then simultaneously run the algorithms of
Lemma~\ref{lem:findingasplitting} and
Lemma~\ref{lem:identifyingsmallorbifolds}. If the former terminates then a
splitting has been found; if the latter does then no splitting exists.
\end{proof}

Note that a subprocess of this algorithm, together with the algorithm
of~\cite{gerasimov} that determines whether or not a hyperbolic group is
one-ended, provides the algorithm promised in
Theorem~\ref{thm:S1boundarycomputable}.

\begin{prop}\label{prop:maximalsplittingcomputable} There is an algorithm that,
when given a presentation for a hyperbolic group, computes the graph of groups
associated to a splitting of that group that is maximal for domination.
\end{prop}

\begin{proof} The algorithm iteratively constructs a sequence of
$\Gamma$-marked graphs of groups $G_i$. Let $G_1$ consist of a single vertex
with vertex group $\Gamma$. Then to obtain $G_{i+1}$ from $G_i$, use
Proposition~\ref{prop:isthereasplitting} to check whether each vertex group
splits non-trivially relative to its incident edge groups. If no vertex group
does split then halt the algorithm here. If the group at a vertex $v$ does
split then find the non-trivial graph of groups $G'$ with fundamental group
$\Gamma_v$ using Lemma~\ref{lem:findingasplitting}. Then define $G_{i+1}$ by
replacing the vertex $v$ of $G_i$ by the graph $G'$ and connecting edges
corresponding to the edges of $G_i$ incident at $v$ to $G'$ in the obvious
way.

This process must eventually stabilise: this follows from an accessibility
theorem~\cite{bestvinafeighn91} since the associated group actions on trees
constructed are minimal and reduced by construction. Let the corresponding
$\Gamma$-trees stabilise at a $\Gamma$-tree $T$. 

Suppose that another $\Gamma$-tree $T'$ with virtually cyclic edge stabilisers
dominates $T$. Let $v$ be a vertex of $T$. If $e$ is an incident edge then
$\Gamma_e$ is elliptic with respect to the action on $T'$, since it contains
the stabiliser of an edge of $T'$ as a subgroup of finite index.  Therefore
each incident edge subgroup of $\Gamma_v$ is elliptic with respect to $T'$. The
vertex stabiliser $\Gamma_v$ does not split over a virtually cyclic subgroup
relative to its incident edge groups, so $\Gamma_v$ is elliptic with respect to
$T'$. The vertex $v$ was arbitrary, so $T$ dominates $T'$, and so $T$ is maximal
for domination.  \end{proof}

