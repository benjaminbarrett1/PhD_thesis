\section{JSJ Decompositions}\label{sec:JSJcomputable}

In this section we show that three closely related types of JSJ splittings are
computable for hyperbolic groups. Fix a one-ended hyperbolic group $\Gamma$ and recall that 
$\mathcal{VC}$ is defined to be the set of all virtually cyclic subgroups of $\Gamma$,
$\mathcal{Z}$ is defined to be the set of all virtually cyclic subgroups of $\Gamma$
with infinite centre and $\mathcal{Z}_\text{max}$ is defined to be the set of
subgroups of $\Gamma$ in $\mathcal{Z}$ that are maximal for inclusion. We
consider the JSJ splittings over groups in these three sets.

\subsection{Two-ended edge groups}

We now prove the first part of Theorem~\ref{thm:maintheorem}.

\begin{thm}\label{thm:VC-JSJcomputable} There is an algorithm that takes as
input a presentation for a one-ended hyperbolic group and returns as output the graph
of groups associated to a $\mathcal{VC}$-JSJ decomposition for that group.
This decomposition can be taken to be Bowditch's canonical
decomposition.\end{thm}

We first prove the following lemma.

\begin{lem}\label{lem:JSJfrommaximal} The tree obtained from the tree
associated to a maximal splitting by collapsing each edge whose stabiliser is
not universally elliptic is a $\mathcal{VC}$-JSJ tree.\end{lem}

\begin{proof} Let $T$ be the tree associated to a maximal splitting and let
$T'$ be the tree obtained by collapsing each edge of $T$ that is not
universally elliptic. Then certainly $T'$ is universally elliptic, so it is
sufficient to show that if $\Sigma$ is another universally elliptic
$\Gamma$-tree then $T'$ dominates $\Sigma$.

The tree $\Sigma$ can be refined to dominate $T$, so there exists a map $f
\colon T \to \Sigma$.  Let $v$ be a vertex of $T'$ and let $S$ be a
component of its preimage in $T$.  Then $f\restricted{S}$ is constant: if
an edge $e$ in $S$ is mapped into an edge $e'$ in $\Sigma$ then $\Gamma_e \leq
\Gamma_{e'}$, which is universally elliptic.  But then the image of $S$ in
$T'$ contains more than a single vertex.  Therefore $\Gamma_v$ fixes the
vertex $f(S)$ in $\Sigma$, so is elliptic with respect to $\Sigma$. This shows that
$T'$ dominates $\Sigma$.\end{proof}

We must now identify the edges in the tree associated to the maximal splitting
that are not maximally elliptic. We make the following definitions.

\begin{defn}\label{defn:mobiusstripgroup} An \emph{extended M\"obius strip
group} is a virtually cyclic group of $\mathcal{Z}$ type with peripheral
structure consisting of a single index 2 subgroup.\end{defn}

\begin{defn}\label{defn:surfaceboundaryedges} We say that an edge $e$
connecting vertices $v_1$ and $v_2$ of a $\Gamma$-tree is a \emph{internal
surface edge} if, for each $i$, either $\Gamma_{v_i}$ is a hanging Fuchsian
group and $\Gamma_e$ is maximal among virtually cyclic subgroups of
$\Gamma_{v_i}$, or $\Gamma_{v_i}$ is an extended M\"obius strip group and
$\Gamma_e \leq \Gamma_{v_i}$ is the peripheral subgroup of $\Gamma_{v_i}$.
\end{defn}

\begin{lem}\label{lem:nonuniversallyellipticedges} If $T$ is reduced (that
is, no proper collapse of $T$ dominates $T$) then the edges of
$T$ that are not universally elliptic are precisely the internal surface
edges.\end{lem}

\begin{proof} Let $T'$ be the tree obtained by collapsing each edge of
$T$ that is not universally elliptic as in
Lemma~\ref{lem:JSJfrommaximal}, so $T'$ is a JSJ tree and there is a
collapse map from $T$ to $T'$. The edges of $T$ that are not
universally elliptic are precisely those edges that are mapped to flexible
vertices of $T'$ under the collapse map; by~\cite[Theorem
6.2]{guirardellevitt16} all flexible vertices of $T'$ are hanging
Fuchsian vertices. 

Any splitting of a hanging Fuchsian group is dual to a family of curves on the
associated orbifold, so any edge in such a splitting is an internal surface
edge, so all edges that are not universally elliptic are internal surface
edges.

Conversely, let $e$ be an internal surface edge. Let $T'$ be the tree
obtained by collapsing each edge in the orbit of $e$; let $v$ be the vertex of
$T'$ in the image of $e$. Then $T$ is obtained from $T'$ by
refining at $v$. The tree $T$ was assumed to be reduced and $v$ is a
hanging Fuchsian vertex so this refinement is dual to an essential simple
closed curve $\ell$ on the associated orbifold $Q$. Then $Q$ contains another
essential simple closed curve $\ell'$ that is not homotopic to a curve disjoint
from $\ell$. Refine $T'$ at $v$ dual to $\ell'$ to obtain a tree
$T''$; then $\Gamma_e$ is not elliptic with respect to $T''$.
\end{proof}

We now have sufficient tools to prove the computability of a $\mathcal{VC}$-JSJ
for a given hyperbolic group $G$. In~\cite{bowditch98}, Bowditch defines a
canonical JSJ in the class of all $\mathcal{VC}$-JSJs of a given hyperbolic
group. In the language of~\cite{guirardellevitt16} this is the decomposition
corresponding to the \emph{tree of cylinders} of any other $\mathcal{VC}$-JSJ.

\begin{defn} Let $T$ be a $\mathcal{VC}$-tree. Define the \emph{commensurability}
equivalence relation $\sim$ on $\mathcal{VC}$ by letting $A \sim B$ if
and only if $A$ and $B$ lie in the same maximal virtually cyclic subgroup of
$\Gamma$. Also denote by $\sim$ the equivalence relation on the set of
edges of $T$ defined by letting $e \sim e'$ if and only if $\Gamma_e \sim
\Gamma_{e'}$. A \emph{cylinder} is a subset $Y\subset T$ that is the union of
all edges in a $\sim$-equivalence class.\end{defn}

\begin{defn} Let $T$ be a $\mathcal{VC}$-tree. The corresponding \emph{tree of
  cylinders} $T_c$ is a bipartite tree with vertex set $V_1 \disjointunion
  V_2$, where $V_1$ is the set of vertices of $T$ that lie in at least two
  cylinders and $V_2$ is the set of cylinders in $T$. A vertex $v \in V_1$ is
connected by an edge to $Y \in V_2$ if and only if $v \in Y$.\end{defn}

\begin{proof}[Proof of Theorem~\ref{thm:VC-JSJcomputable}] First compute a
maximal splitting of the group over virtually cyclic subgroups by
Theorem~\ref{prop:maximalsplittingcomputable}. Let $T$ be the associated
tree. By construction $T$ is reduced; in any case, $T$ can easily be made
reduced using the processes of Lemma~\ref{lem:twoendedsubgroups}. For each
edge $e$ connecting vertices $v_1$ and $v_2$ of the graph of groups
$T/\Gamma$ determine whether $\Gamma_e$ is maximal in $\Gamma$ using the
algorithm of Lemma~\ref{lem:twoendedsubgroups} and whether the two vertex
groups $\Gamma_{v_1}$ and $\Gamma_{v_2}$ have circular boundary relative to
their incident edge groups by Theorem~\ref{thm:S1boundarycomputable}. Check
also whether each of $\Gamma_{v_1}$ and $\Gamma_{v_2}$ is virtually cyclic of
$\mathcal{Z}$-type, and, if it is, whether or not $\Gamma_e$ has index 2{} in
that group. One of these possibilities is the case if and only if $e$ is not
universally elliptic by Lemma~\ref{lem:nonuniversallyellipticedges}; collapse
all edges where this is the case.

Bowditch's canonical decomposition is the graph of cylinders of the
decomposition obtained in this way. The operation of replacing a decomposition
with the decomposition associated to its tree of cylinders can be done
algorithmically using~\ref{lem:twoendedsubgroups}. This is result the content
of~\cite[Lemma 2.34]{dahmaniguirardel11}; note that while the result is stated
for a $\mathcal{Z}$-tree, replacing this with a $\mathcal{VC}$-tree makes no
difference to the proof.\end{proof}

\subsection{$\mathcal{Z}$ edge groups}

We now prove the second part of Theorem~\ref{thm:maintheorem}.

\begin{thm} There is an algorithm that takes as input a presentation for a
one-ended hyperbolic group and returns the graph of groups associated to a
$\mathcal{Z}$-JSJ decomposition for that group.\end{thm}

In~\cite{dahmaniguirardel11} it is shown that the $\mathcal{Z}$-JSJ
decomposition is closely related to the $\mathcal{VC}$-JSJ: a $\mathcal{Z}$-JSJ
tree can be obtained from a $\mathcal{VC}$-JSJ tree $T$ by first refining $T$
by applying the so-called mirrors splitting to each hanging Fuchsian vertex group and
then collapsing each edge with stabiliser of dihedral type. 
The second of these processes can be done algorithmically using the part of the
algorithm of Lemma~\ref{lem:twoendedsubgroups} that determines whether or not a
given virtually cyclic group is of dihedral type.  Therefore we must now show
that the mirrors splitting is computable.

Recall the definition of the mirrors splitting of the fundamental group of a
compact 2-dimensional orbifold $Q$ from~\cite{dahmaniguirardel11}. 

\begin{defn} Let $N$ be
a regular neighbourhood of the union of the mirrors and $D_\infty$-boundary
components of $Q$ that does not contain any cone point of $Q$. If $Q - N$ is an
annulus or a disc with at most one cone point then the \emph{mirrors splitting} of
$\pi_1Q$ is defined to be trivial; otherwise it is the splitting obtained by
cutting $Q$ along each component of $\boundary N$. \end{defn}

If the mirrors splitting is non-trivial then the graph of groups associated to the splitting is a star; the group at the central
vertex is the fundamental group of an orbifold with no mirrors and the group at
each leaf is the fundamental group of an orbifold with no cone points and
underlying surface an annulus, one of whose topological boundary components is
a circular orbifold boundary component and the other a union of interval
boundary components and at least one mirror. If $\Gamma$ is any hyperbolic
group with a collection $\mathcal{H}$ of virtually cyclic subgroups such that
$\boundary(\Gamma, \widehat{\mathcal{H}})$ is homeomorphic to a circle then the mirrors
splitting of $\Gamma$ relative to $\mathcal{H}$ is defined to be the splitting
induced by the mirrors splitting of the quotient $\Gamma$ by a maximal finite
normal subgroup of $\Gamma$ as in Proposition~\ref{lem:grouptoorbifold}.

\begin{lem} The mirrors splitting of a hyperbolic group $\Gamma$ with a set
$\mathcal{H}$ of virtually cyclic subgroups such that $\boundary(\Gamma,
\widehat{\mathcal{H}})$ is homeomorphic to a circle is computable.\end{lem}

\begin{proof} Using Proposition~\ref{lem:grouptoorbifold} it is enough to show
  that the mirrors splitting is computable in the case where $\Gamma$ is
  bounded Fuchsian and $\mathcal{H}$ is the a collection of representatives of
  peripheral subgroups of $\Gamma$. To do this we enumerate all mirrors
  splittings: for each non-negative integer $k$ enumerate all fundamental
  groups of compact orbifolds without mirrors and with at least $k$ boundary
  components and all $k$-tuples of fundamental groups of orbifolds homeomorphic
  to an annulus with no cone points and such that one topological boundary
  component of the orbifold is a circular orbifold boundary component. In each
  case form the graph of groups in which the underlying graph is a $k$-pointed
  star, the group at the central vertex is the fundamental group of the
  orbifold without mirrors, the group at each leaf is the fundamental group of
  an orbifold homeomorphic to an annulus and the group at each edge is infinite
  cyclic and is identified with the fundamental group of a circular orbifold
  boundary component of each of the orbifolds associated to the end points of
  that edge. Compute the fundamental group of each such graph of groups and
  record also the peripheral structure consisting of conjugacy class
  representatives of the fundamental groups of components of the orbifold
  boundary of the orbifold.

  Also enumerate all groups with trivial mirrors splitting, i.e.\ fundamental
  groups of orbifolds homeomorphic as topological spaces to a disc with at most
  one cone point, or homeomorphic to an annulus with no cone points and such
  that one topological boundary component is a circular orbifold boundary
  component.

  In parallel enumerate all homomorphisms from the fundamental groups of these
  graphs of groups to $\Gamma$ and all homomorphisms from $\Gamma$ to the
  fundamental groups of these graphs of groups. Some such pair of homomorphisms
  is an inverse pair that preserves the peripheral structure up to conjugacy.
  On finding this pair the algorithm returns the associated mirrors splitting.
\end{proof}

\subsection{$\mathcal{Z}_\text{max}$ edge groups}

In~\cite{dahmaniguirardel11} it is shown that the $\mathcal{Z}_\text{max}$-JSJ
decomposition can be obtained from a $\mathcal{Z}$-JSJ decomposition by
performing the so-called $\mathcal{Z}_\text{max}$-fold. We note that this can
be done algorithmically, which completes the proof of the final part of
Theorem~\ref{thm:maintheorem}, which we restate here as Theorem~\ref{thm:ZmaxJSJ}.

\begin{thm}\label{thm:ZmaxJSJ} There is an algorithm that takes as input a presentation for a
one-ended hyperbolic group and outputs the graph of groups associated to a
$\mathcal{Z}_\text{max}$-JSJ decomposition for that group.\end{thm}

\begin{lem} There is an algorithm that takes as input a graph of groups
decomposition of a hyperbolic group $\Gamma$ over virtually cyclic subgroups
and returns the graph of groups associated to the
$\mathcal{Z}_\text{max}$-fold of the associated $\Gamma$-tree.\end{lem}

\begin{proof} This can be done by iterating some simple folds described
in~\cite{dahmaniguirardel11}. We describe this fold at the level of the tree
$T$. Take some edge $e$ in $T$ such that the maximal subgroup $\hat\Gamma_e$
of $\Gamma$ containing $\Gamma_e$ is a subgroup of $\Gamma_{o(e)}$ where $o(e)$ is
the origin of $e$. Then take the quotient of $T$ by the $\Gamma$-equivariant
equivalence relation generated by $e \sim \hat\Gamma_e \cdot e$. Note that this
does not change the underlying graph of the associated graph of groups.

At the level of the graph of groups this fold is achieved by taking an edge $e$
in the graph such that $\hat \Gamma_e$ is a subgroup of $\Gamma_{o(e)}$,
replacing the group at the edge $e$ by $\hat\Gamma_e$ and replacing the group
at the terminal vertex $t(e)$ of $e$ by $\langle\hat\Gamma_e,
\Gamma_{t(e)}\rangle$. Repeat this process until there is no edge $e$ such that
$\Gamma_e \neq \hat\Gamma_e$. \end{proof}

