\begin{abstract} We present an algorithm that computes Bowditch's canonical JSJ
decomposition of a given one-ended hyperbolic group over its virtually cyclic
subgroups. The algorithm works by identifying topological features in the boundary
of the group. As a corollary we also show how to compute the JSJ
decomposition of such a group over its virtually cyclic subgroups with
infinite centre. We also give a new algorithm that determines whether or not a
given one-ended hyperbolic group is virtually fuchsian. Our approach uses only
the geometry of large balls in the Cayley graph and avoids Makanin's
algorithm.\end{abstract}

\section{Introduction}

When studying a group it is natural and often useful to try to cut it into
simpler pieces by means of amalgamated free products and HNN extensions over
particularly simple subgroups. Sometimes this can be done in a canonical way
analogous to the characteristic submanifold decomposition of Jaco, Shalen and
Johannson~\cite{jacoshalen79, johannson79}, in which the family of embedded
tori along which the 3-manifold is cut is unique up to isotopy. Such JSJ
decompositions were introduced to group theory by Sela~\cite{sela97} to answer
questions about rigidity and the isomorphism problem for torsion-free
hyperbolic groups. In~\cite{bowditch98} Bowditch developed a related type of
decomposition for hyperbolic groups possibly with torsion. This decomposition
is built from the structure of local cut points in the boundary of the group
and is therefore an automorphism invariant of the group; this property of the
Bowditch JSJ was used in Levitt's work~\cite{levitt05} on outer automorphism
groups of one-ended hyperbolic groups. For more general constructions of JSJ
decompositions of groups see~\cite{ripssela97, dunwoodysageev99,
fujiwarapapsoglu06, guirardellevitt16}.

The above results describe and prove the existence of various types of JSJ
decompositions but do not give an algorithm to construct them.
Gerasimov~\cite{gerasimov} proved that there exists an algorithm that
determines whether or not the Gromov boundary of a given hyperbolic group is
connected. This algorithm is unpublished; see also~\cite{dahmanigroves08a}. The
connectedness of the boundary is determined by the so-called double-dagger
condition of Bestvina and Mess~\cite{bestvinamess91}; it is this condition that
Gerasimov showed to be computable. Equipped with this algorithm and Stallings's
theorem on ends of groups it is not difficult to compute a maximal decomposition of
a given hyperbolic group over its finite subgroups. With Gerasimov's result in
hand, we may restrict to the case of one-ended hyperbolic groups and consider
the computability of Bowditch's JSJ decomposition over virtually cyclic
subgroups. 

In this paper we present an algorithm that computes Bowditch's decomposition.
Like Gerasimov's algorithm, our approach uses the geometry of large balls in
the Cayley graph. This is in contrast to existing algorithms computing JSJ
decompositions over restricted families of virtually cyclic subgroups, most of
which rely on Makanin's algorithm for solving equations in free groups.

In~\cite{dahmaniguirardel11} Dahmani and Guirardel show that a canonical
decomposition of a one-ended hyperbolic group over a particular family of
virtually cyclic subgroups is computable; the family in question is the set of
virtually cyclic subgroups with infinite centre that are maximal for inclusion
among such subgroups. Crucial to this method is an algorithm that determines
whether or not the outer automorphism group of such a group is infinite. If a
group admits such a splitting then that splitting gives rise to an infinite set
of distinct elements of the outer automorphism group that are analogous to Dehn
twists in the mapping class group of a surface. The converse of this statement
is a theorem of Paulin~\cite{paulin91} that is refined by Dahmani and
Guirardel.

Dahmani and Guirardel comment that it is not known whether or not Bowditch's JSJ
decomposition is computable. Their approach is not suitable to this problem:
only central elements of the edge groups in a splitting contribute Dehn twists
to the automorphism group, so it is quite possible for a group to admit a
splitting over an infinite dihedral group, say, while having only a finite
outer automorphism group; in this case the decomposition computed by Dahmani
and Guirardel is trivial while Bowditch's JSJ decomposition is not. For
examples of hyperbolic groups exhibiting this property
see~\cite{millerneumannswarup96}. 

In the absence of torsion, the JSJ decomposition of a hyperbolic group over its
cyclic subgroups was shown to be computable by Dahmani and Touikan
in~\cite{dahmanitouikan13}. Their result is based on Touikan's
algorithm~\cite{touikan09}, which determines whether or not a given one-ended
hyperbolic group without 2-torsion splits acylindrically. Touikan's methods are
based on application of the Rips machine.

The existence of a splitting of a one-ended hyperbolic group over a virtually
cyclic subgroup is reflected in the existence of certain topological features
in its Gromov boundary by results of Bowditch~\cite{bowditch98, bowditch99a,
bowditch99b}; in this paper we show that these topological features can be
detected algorithmically.

\begin{thm}\label{thm:maintheorem} There is an algorithm that takes as input a
  presentation for a one-ended hyperbolic group and returns the graph of groups
  associated to the three following JSJ decompositions:
  \begin{enumerate}
    \item A JSJ decomposition over virtually cyclic subgroups of $\Gamma$,
      which we shall call a $\mathcal{VC}$-JSJ. This decomposition can be
      taken to be Bowditch's canonical decomposition.
    \item A JSJ decomposition over virtually cyclic subgroups of $\Gamma$
      with infinite centre, which we shall call a $\mathcal{Z}$-JSJ.
    \item A decomposition over maximal virtually cyclic subgroups of $\Gamma$
      with infinite centre that is universally elliptic over (not necessarily
      maximal) virtually cyclic subgroups of $\Gamma$ and is maximal for
      domination in the class of such decompositions. We shall call this a
      $\mathcal{Z}_\text{max}$-JSJ.
  \end{enumerate}
\end{thm}

The main content of Theorem~\ref{thm:maintheorem} is the computability of the
$\mathcal{VC}$-JSJ; it is shown in~\cite{dahmaniguirardel11} to be closely
related to the $\mathcal{Z}$-JSJ and $\mathcal{Z}_\text{max}$-JSJ and can
converted into either algorithmically. The $\mathcal{Z}_\text{max}$-JSJ is the
decomposition shown to be computable by Dahmani and
Guirardel~\cite{dahmaniguirardel11}.

The $\mathcal{Z}$-JSJ also plays a result in Dahmani and Guirardel's work,
although they comment that their methods cannot compute this decomposition,
since such a decomposition does not necessarily give rise to infinitely many
distinct outer automorphisms of the group.  For example, let $\Gamma$ be a
rigid hyperbolic group (such as the fundamental group of a closed hyperbolic
3-manifold) and let $g$ be an element of $\Gamma$ that is not a proper power.
Let $k > 1$ and consider the group $\Gamma' = \Gamma \freeprod_{g=t^k} \langle
t\rangle$ obtained by adjoining a $k$th root of $g$ to $\Gamma$. In this case
the $\mathcal{Z}_\text{max}$ decomposition computed by Dahmani and Guirardel is
trivial while the JSJ decomposition over virtually cyclic subgroups with
infinite centre is not. 

Central to the algorithm of Theorem~\ref{thm:maintheorem} is an algorithm that
determines whether or not a given hyperbolic group with a (possibly empty)
finite collection of virtually cyclic subgroups admits a proper splitting as an
amalgamated product or HNN extension over a virtually cyclic subgroup, relative
to that collection of subgroups. Recall that a \emph{cut pair} in a connected
topological space $S$ is a pair of points $p$ and $q$ such that $S - \{p, q\}$
is disconnected. It is shown in~\cite{bowditch98} that in the absolute case
(that is, if the collection of subgroups is empty), a one-ended hyperbolic
group admits such a splitting if and only if its Gromov boundary contains a cut
pair, at least as long as its boundary is not homeomorphic to a circle. In the
case of interest here we obtain a relative version of this statement by
replacing the Gromov boundary with the Bowditch boundary of the group relative
to the given family of subgroups. Unlike the Gromov boundary, the Bowditch
boundary might contain a cut point, in which case the group admits a relative
splitting. This is the peripheral splitting in the sense of
Bowditch~\cite{bowditch99b}. In the absence of a cut point the existence of a
relative splitting is determined by the existence of a cut pair, as in the
absolute case. It is the presence of these topological features of the boundary
that we show to be computable.

To detect the presence of a cut pair in the Bowditch boundary of a hyperbolic
group relative to a given family of subgroups  we first show that the
connectivity of the complement of a pair of points in the boundary is
equivalent to the connectivity of a thickened cylinder around a geodesic
connecting that pair of points in the cusped space defined
in~\cite{grovesmanning08}. Then, supposing that there is a cut pair in the
boundary, we use a pumping lemma argument to show that the geodesic $\gamma$
connecting the points in a cut pair may be assumed to be periodic and with
bounded period: we take a short subsegment $\gamma\restricted{[a, b]}$ of that
geodesic such that both the geodesic and the components of the thickened
cylinder are identical in small neighbourhoods of $a$ and $b$ and form a new
(local) geodesic that also connects the two points in a (possibly different)
cut pair by concatenating infinitely many copies of $\gamma\restricted{[a,
b]}$. A similar method is used in~\cite{cashenmacura11} to control cut pairs in
the decomposition space of a line pattern in a free group. This is sufficient
to detect a cut pair: the existence of such a periodic geodesic can be detected
in finite time by searching a large finite ball in the Cayley graph.

A maximal splitting is obtained from a JSJ decomposition by refining at the
flexible vertices. Conversely, to obtain a JSJ decomposition we must decide
which edges of the maximal splitting should be collapsed to reassemble the
flexible vertices in the JSJ decomposition. In Bowditch's JSJ decomposition the
stabilisers of flexible vertex groups are the maximal hanging fuchsian
subgroups. These are those subgroups that occur as a vertex stabiliser in some
splitting such that the Bowditch boundary of the subgroup relative the
stabilisers of the incident edges is homeomorphic to a circle. Therefore we
prove the following theorem, which is interesting in its own right, and does
not seem to appear in the literature.  Recall that the Convergence Group
Theorem of Tukia, Casson, Jungreis and Gabai~\cite{tukia88, cassonjungreis94,
gabai92} implies that the Gromov boundary of a hyperbolic group is homeomorphic
to a circle if and only if the group surjects with finite kernel onto the
fundamental group of a compact hyperbolic orbifold.

\begin{thm}\label{thm:S1boundarycomputable} There is an algorithm that takes as
input a hyperbolic group $\Gamma$ and a (possibly empty) collection of
virtually cyclic subgroups $\mathcal{H}$ and returns an answer to the
question ``is $\boundary(\Gamma, \mathcal{H})$ homeomorphic to
$S^1$?''\end{thm}

The algorithm of Theorem~\ref{thm:S1boundarycomputable} is similar to the
algorithm that detects the presence of a cut pair in the Bowditch boundary: it
follows from a result in point-set topology that the boundary is homeomorphic
to a circle if and only if every pair of points in the boundary is a cut pair.
We show that if there is a non-cut pair in the Bowditch boundary then there is
a non-cut pair connected in the cusped space by a local geodesic with bounded
period. 

In section~\ref{sec:cuspedspace} we first review the definition of the cusped
space and the Bowditch boundary. We then recall some important properties: the
computability of the hyperbolicity constant of the cusped space, the existence
of a visual metric on the Bowditch boundary, and most importantly the so-called
double-dagger condition, which will be vital in linking the connectivity of the
boundary to that of subsets of the cusped space. We then define a thickened
cylinder around a geodesic in the cusped space and show that its connectivity
determines the connectivity of the complement of the limit set of that geodesic
in the boundary.

Section~\ref{sec:cutpairsalgorithms} contains the main technical results of the
paper we describe the algorithms that determine whether or not the Bowditch
boundary of a hyperbolic group contains the three topological features of
interest to us: cut points, cut pairs and non-cut pairs.

In section~\ref{sec:fuchsiangroups} we deal with the special case of a group
with circular Bowditch boundary. We first recall some results that reduce the
problem of determining whether or not such a group admits a proper
splitting relative to its given virtually cyclic subgroups to the case in which the
group is the fundamental group of a compact two-dimensional hyperbolic orbifold
and the given subgroups are precisely conjugacy class representatives of the
fundamental groups of the boundary components of that orbifold. In
\cite{guirardellevitt16} a complete list of such orbifolds that do not admit
such a splitting is described; we use this to complete this special case.

In section~\ref{sec:maximal} we first record the general definition of a JSJ
decomposition and a description of Bowditch's canonical JSJ decomposition over
virtually cyclic subgroups. We then recall the theorem of~\cite{bowditch98}
that links the topology of the Gromov boundary of a hyperbolic group to the
existence of a proper splitting of that group and extend it to a relative
version. We then show how to use the algorithms described so far to compute a
maximal splitting of a one-ended hyperbolic group over virtually cyclic
subgroups.

In section~\ref{sec:JSJcomputable} we complete the proof of
Theorem~\ref{thm:maintheorem} by describing the processes that convert a
maximal splitting of a one-ended hyperbolic group over virtually cyclic
subgroups into its $\mathcal{VC}$-JSJ, $\mathcal{Z}$-JSJ and
$\mathcal{Z}_\text{max}$-JSJ.

It seems plausible that the techniques of this paper might extend to the
problem of detecting the presence of splittings of relatively hyperbolic groups
with parabolic subgroups in some restricted class. In particular, it is natural
to try to solve this problem for groups that are hyperbolic relative to
finitely generated virtually nilpotent subgroups. Such groups arrise as
fundamental groups of complete finite volume Riemannian manifolds with pinched
negative sectional curvature. However, it is of fundamental importance to the
argument presented in this paper that the cusped space associated to the group
satisfies Bestvina and Mess's double-dagger condition, and we do not know under
what circumstances this condition holds for virtually nilpotent parabolic
subgroups. In~\cite{dahmanigroves08a} it is established that the double-dagger
condition holds if the parabolic subgroups are abelian, so it seems likely that
the methods of this paper could be extended to the case of toral relatively
hyperbolic groups. (A group is toral relatively hyperbolic if it is torsion
free and hyperbolic relative to a finite collection of abelian subgroups.)
However, the JSJ decomposition of a toral relatively hyperbolic group is shown
to be computable in~\cite{dahmanitouikan13}, so we do not introduce additional
technical complexity by trying to give a new proof of this result here.

\subsection*{Acknowledgements}

I thank my supervisor Henry Wilton for suggesting this problem and for his
guidance towards its solution.  I am also grateful to the anonymous reviewer
for pointing out several problems with the clarity and accuracy of the earlier
draft.

