\section{The cusped space and cut pairs} \label{sec:cuspedspace}

\subsection{The cusped space and the boundary} 

In this section we recall some technical results about the geometry of
relatively hyperbolic groups.  Fix a group $\Gamma$ with a finite generating
set $S$ and a finite collection $\mathcal{H}$ of subgroups of $\Gamma$ such
that $H \intersection S$ is a generating set for $H$ for each $H$ in
$\mathcal{H}$. For now we may allow the groups in $\mathcal{H}$ to be
arbitrary, although in our application they will be virtually cyclic.

In~\cite{grovesmanning08} the cusped space $X$ associated to the triple
$(\Gamma, S, \mathcal{H})$ is defined; we recall the definition here.
See~\cite{grovesmanning08} for more details and properties of the cusped space.
We use only the 1-skeleton of the cusped space so we omit the 2-cells from the
definition.

\begin{defn}\label{defn:combhoroball} For $C$ a 1-complex define the
\emph{combinatorial horoball} $\widehat{C}$ based on $C$ to be the 1-complex with
vertex set $C^{(0)} \times \{0 \union \naturals\}$ and three types of edges:
\begin{enumerate}
  \item One \emph{horizontal edge} from $(v, 0)$ to $(w, 0)$ for each edge from
    $v$ to $w$ in $C$. (Note that we allow vertices to be connected by multiple edges.)
  \item For $k \geq 1$ a \emph{horizontal edge} from $(v, k)$ to $(w, k)$
    whenever $0 < d(v, w) \leq 2^k$.
  \item A \emph{vertical edge} from $(v, k)$ to $(v, k+1)$ for each $v \in
    C^{(0)}$ and each $k \in \integers_{\geq 0}$.
\end{enumerate}
We define a height function  $h \colon \widehat{C} \to \reals_{\geq 0}$ sending a
vertex $(v, k)$ to $k$ and interpolating linearly along edges. \end{defn}

\begin{defn}\label{defn:cuspedspace} If $\Gamma$, $S$ and $\mathcal{H}$ are as
above then the \emph{cusped space} $X$ associated to this triple is defined
as follows. For $H$ in $\mathcal{H}$ let $T_H$ be a left transversal of $H$
in $\Gamma$. For $H$ in $\mathcal{H}$ and $t$ in $T_H$ let $C_{H, t}$ be the
full subgraph of $\Cay(\Gamma, S)$ containing $tH$; note that this is
isomorphic to $\Cay(H, H \intersection S)$. Then let $X$ be the union
\begin{align*}
  \Cay(\Gamma, S) \union \bigunion_{H\in\mathcal{H}, t \in T_H} \widehat{C_{H, t}}
\end{align*}
where we identify $C_{H, t} \subset \Cay(\Gamma, S)$ with the part of
$\widehat{C_{H, t}}$ with height 0. Then the height function defined on each
horoball extends to a height function $h \colon X \to \reals_{\geq 0}$. 

For $k \geq 0$ the \emph{$k$-thick part} $X_k$ of $X$ is defined to be $h^{-1}[0, k]$.
We say that a path $\gamma$ in $X$ is \emph{vertical} if $h \composed \gamma$
is strictly monotonic and \emph{horizontal} if $h \composed \gamma$ is
constant. \end{defn} 

\begin{defn} A (quasi-)isometric embedding $\alpha$ from a subinterval of
$\reals$ to $X$ is called a \emph{(quasi)-geodesic}. If that interval is
closed and bounded then $\alpha$ is a \emph{segment}. If the interval is $[0,
\infty)$ then $\alpha$ is a \emph{ray}. If the interval is all of $\reals$
then $\alpha$ is \emph{bi-infinite}.

$X$ is called a \emph{geodesic space} if for any $x$ and $y$ in $X$ there
exists a geodesic segment $\alpha \colon [a, b]$ with $\alpha(a) = x$ and
$\alpha(b) = y$.\end{defn}

\begin{defn}\label{defn:hyperbolic} A geodesic metric space $X$ is
\emph{$\delta$-hyperbolic} if every edge of any geodesic triangle in $X$ is
contained in a $\delta$-neighbourhood of the union of the other two edges.
\end{defn}

\begin{defn}\label{defn:relativelyhyperbolic} $\Gamma$ is \emph{hyperbolic
relative to $\mathcal{H}$} if the cusped space associated to $(\Gamma, S,
\mathcal{H})$ is hyperbolic for some (equivalently for any) generating set
$S$ in which $S \intersection H$ is a generating set for $H$ for each $H \in
\mathcal{H}$. \end{defn}

This definition of relative hyperbolicity is shown to be equivalent to several
other standard definitions in~\cite{grovesmanning08}.

We recall a lemma from~\cite{bowditch12} that characterises hyperbolicity of
the pair $(\Gamma, \mathcal{H})$ when $\Gamma$ is itself a hyperbolic group.
Recall that a collection $\mathcal{H}$ of subgroups of $\Gamma$ is \emph{almost
malnormal} if, for $H_1$ and $H_2$ in $\mathcal{H}$ and $g$ in $\Gamma$, $H_1
\intersection gH_2g^{-1}$ is finite unless $H_1 = H_2$ and $g \in H_1$.

\begin{lem}\label{lem:qcalmostmalnormal}\cite[Theorem 7.11]{bowditch12} Let
  $\Gamma$ be a non-elementary hyperbolic group and let $\mathcal{H}$ be a
  finite set of subgroups of $\Gamma$. Then $\Gamma$ is hyperbolic relative to
  $\mathcal{H}$ if and only if $\mathcal{H}$ is almost malnormal in $\Gamma$
and each element of $\mathcal{H}$ is quasi-convex in $\Gamma$.\end{lem}

In the cases of interest to us all groups in $\mathcal{H}$ are virtually
cyclic. A virtually cyclic subgroup of a hyperbolic group is always
quasi-convex, and is almost malnormal if and only if it is maximal among
virtually cyclic subgroups of $\Gamma$. Putting this together with
Lemma~\ref{lem:qcalmostmalnormal} we obtain the following:

\begin{lem}\label{lem:vcycperipheral} If $\Gamma$ is hyperbolic and each group
in $\mathcal{H}$ is maximal virtually cyclic then $\Gamma$ is hyperbolic
relative to $\mathcal{H}$.\end{lem}

Fix $X$ as in Definition~\ref{defn:cuspedspace} and assume that it is $\delta$-hyperbolic.
We shall require the following two lemmas describing the geometry of $X$. The
first is~\cite[Lemma 2.11]{dahmanigroves08a} and the second is the Morse lemma
for a hyperbolic metric space.

\begin{lem}\label{lem:closeray} Let $C = 3\delta$. For any $v$ in $X_0$ and $x$
in $X$ there is a geodesic ray starting at $v$ that passes within a distance
$C$ of $x$.\end{lem}

\begin{lem}\label{lem:uniformlyclose} There exists a (computable) function
$D(\lambda, \epsilon, \delta)$ such that whenever $\gamma$ is a geodesic and
$\gamma'$ is a $(\lambda, \epsilon)$-quasi-geodesic with the same end points in
$X \union \boundary X$ the Hausdorff distance from $\gamma$ to $\gamma'$ is at
most $D$. \end{lem}

Fixing $\delta$ so that $X$ is $\delta$-hyperbolic we fix the constant $C =
3\delta$ and function $D = D(\lambda, \epsilon, \delta)$ as in Lemmas
\ref{lem:closeray} and~\ref{lem:uniformlyclose} for the remainder of this and
the following section.

We shall need an algorithm to compute the constant $\delta$ with respect to
which the cusped space is $\delta$-hyperbolic. This is dealt with by the
following results.

\begin{prop}\cite[Prop.\ 2.3]{dahmani08}
\label{thm:linearisoperimetricinequality} There is an algorithm that takes as
input a presentation for a group $\Gamma$, the generators in $\Gamma$ for a
finite set of subgroups of $\Gamma$ with respect to which $\Gamma$ relatively
hyperbolic, and a solution to the word problem in $\Gamma$ and returns the
constant of a linear relative isoperimetric inequality satisfied by the given
presentation of $\Gamma$.\end{prop}

In the case of interest here, $\Gamma$ is hyperbolic. Hyperbolic groups have
uniformly solvable word problem, so the requirement that the algorithm be given
a solution to the word problem in $\Gamma$ is no restriction to its
applicability.

The linear relative isoperimetric inequality satisfied by the group is closely
related to a linear combinatorial isoperimetric inequality satisfied by the
coned-off Cayley complex. The definition of this space 
is~\cite[Definition 2.47]{grovesmanning08}.

In~\cite{grovesmanning08} the cusped 2-complex is defined; this is a simply
connected 2-complex, the 1-skeleton of which is the cusped 1-complex defined
in Definition~\ref{defn:cuspedspace}. The length of the attaching map of each
2-cell in this complex is bounded above by the maximum of $5$ and the length of
the longest relator in the given presentation for $\Gamma$.

\begin{thm}\cite[Theorem 3.24]{grovesmanning08}\label{thm:cuspedtwocomplex}
Suppose that the coned-off Cayley complex of $\Gamma$ with respect to $S$ and
$\mathcal{H}$ satisfies a linear combinatorial isoperimetric inequality with
constant $K$. Then the cusped two-complex $X$ associated to the triple
satisfies a linear combinatorial isoperimetric inequality with constant $3K(2K
+ 1)$.\end{thm}

The computation of $\delta$ from a presentation for $\Gamma$ and a set of
generators for each group $H$ in $\mathcal{H}$ is therefore completed by the
following proposition:

\begin{prop}\cite[Prop.\ 2.23]{grovesmanning08}\label{prop:linearimplieshyperbolic}
Suppose that a 2-complex $X$ is simply connected, that each attaching map has
length at most $M$ and that $X$ satisfies a linear combinatorial
isoperimetric inequality. Then the 1-skeleton $X^{(1)}$ of $X$ is
$\delta$-hyperbolic for some $\delta$ and this $\delta$ is computable from
$M$ and the constant of the isoperimetric inequality. \end{prop}

\subsection{The Bowditch boundary}

\begin{defn}\label{defn:gromovboundary} Let $X$ be a hyperbolic geodesic metric
space. The \emph{Gromov boundary $\boundary X$} of $X$ is defined to be the
quotient of the set of geodesic rays starting at some chosen base point in
which two rays are identified if they are a finite Hausdorff distance apart.
It is endowed the quotient of the compact-open topology.\end{defn}

This definition is quasi-isometry invariant and independent of the chosen base
point up to homeomorphism. 

\begin{defn}\label{defn:bowditchboundary} Let $\Gamma$ be a group that is hyperbolic
relative to a collection $\mathcal{H}$ of subgroups. The \emph{Bowditch
boundary} $\boundary(\Gamma, \mathcal{H})$ is defined be to the Gromov boundary
of the cusped space associated to $\Gamma$ and $\mathcal{H}$ with some
generating set.\end{defn}

This definition of the Bowditch boundary is shown to be equivalent to other
definitions in~\cite{bowditch12}. If each group in $\mathcal{H}$ is maximal
virtually cyclic then we have the following description of the Bowditch
boundary from~\cite{manning15}:

\begin{lem}\label{lem:bowditchfromgromov} If $\Gamma$ is hyperbolic and each
group in $\mathcal{H}$ is maximal virtually cyclic then $\boundary(\Gamma,
\mathcal{H})$ is the quotient of $\boundary(\Gamma, \emptyset)$ by the
equivalence relation in which $x \sim y$ if and only if either $x = y$ or
$\{x, y\} = g\cdot\Lambda H$ for some $g$ in $\Gamma$ and $H$ in $\mathcal{H}$.
The topology on $\boundary (\Gamma, \mathcal{H})$ is the quotient of the
topology on $\boundary(\Gamma, \emptyset)$.
\end{lem}

The boundary of a hyperbolic metric space has a natural quasi-conformal class
of metrics.  Recall the definition of the Gromov product of a pair of points
$p$ and $q$ in $X$ with respect to a base point $v$ in $X$:
\begin{align*}
\gromprod[v]{p}{q} = \frac{1}{2}(d(v, p) + d(v, q) - d(p, q)).
\end{align*}
This definition is extended to allow $p$ and $q$ to be in $X \union \boundary X$ by
\begin{align*}
\gromprod[v]{p}{q} = \sup\liminf_{i, j \to \infty}\gromprod[v]{p_i}{q_j}.
\end{align*}
Here the supremum is taken over pairs of sequences $(p_i)$ and $(q_j)$ in $X$
converging to $p$ and $q$ respectively.

By~\cite[III.H.3.21]{bridsonhaefliger99} $\boundary X$ admits a visual metric
at any base point $v$; that is, a metric $d_v$ on $\boundary X$ satisfying
\begin{align*}
k_1 a^{-\gromprod[v]{p}{q}} \leq d_v(p, q) \leq k_2 a^{-\gromprod[v]{p}{q}}.
\end{align*}
Here $a$, $k_1$ and $k_2$ can be taken to be $2^{1/4\delta}$, $3-2\sqrt{2}$ and
$1$ respectively.

\subsection{The double-dagger condition}

If the boundary of $X$ does not contain a cut point then its local
connectivity is controlled by the so-called double-dagger condition. This was
defined first in the absolute case in~\cite{bestvinamess91} and later in the
relative case in~\cite{dahmanigroves08a}. We now record the definition of the
condition and some important properties. Fix a base point $v$ in $X_0$.

Let $M = 6(C + 45\delta) + 2\delta + 3$. For $\epsilon \geq 0$ we say that a
pair of points $x$ and $y$ in $X$ satisfy $\star_\epsilon$ if $\vert d(v, x) -
d(v, y)\vert \leq \epsilon$ and $d(x, y) \leq M$. 

For $n \geq 0$ we say that a pair of points $x$ and $y$ satisfying
$\star_\epsilon$ satisfy $\ddag(\epsilon, n)(x, y)$ if there exists a path of
length at most $n$ from $x$ to $y$ that avoids the ball of radius $m - C -
45\delta + 3\epsilon$ centred at $v$, where $m = \min\{d(v, x), d(v, y)\}$.
After this section we will not need to allow $\epsilon$ to be non-zero; we
will say that $X$ satisfies $\ddag_n$ if $\ddag(0, n)(x, y)$ holds for all $x$
and $y$ in $X$ satisfying $\star_0$. However, the full definition is required
in the proof of Proposition~\ref{prop:Xsatisfiesddag}.

We require the following proposition, which is essentially Proposition 5.1
of~\cite{dahmanigroves08a}.

\begin{prop}\label{prop:Xsatisfiesddag} Suppose that $\boundary(\Gamma,
\mathcal{H})$ is connected and without a cut point. Then there exists $n$
such that $X$ satisfies $\ddag_n$. Furthermore, there is an algorithm that
computes such an $n$ if $\boundary(\Gamma, \mathcal{H})$ is connected and
without a cut point and that does not terminate if it is not connected.
\end{prop}

For clarity we recall the proof of this proposition
from~\cite{dahmanigroves08a} here.

\begin{proof} We begin by recalling some constants
from~\cite{dahmanigroves08a}. In this paper these constants will only appear
in this proof. Let $k = 2M$, let $K = 3(2^{2M+3}) + M + 3$ and for any $n$
let $R(n) = 4(n + M) + 3k + 50\delta + 3$. For each $n \geq K$ in turn, check
whether $\ddag(10\delta, n)(x, y)$ holds for all pairs of vertices $(x, y)$
in $\Ball_{R(n)}(v) \intersection X_k$ satisfying $\star_{10\delta}$. 

If $\boundary(\Gamma, \mathcal{H})$ is connected and does not contain a cut
point then such an $n$ exists by~\cite[Lemma 4.2]{dahmanigroves08a}. It is
commented that $\ddag(10\delta, n)(x, y)$ holds for all pairs of vertices $(x,
y)$ in $\Ball_{R(n)}(v)$ satisfying $\star_{10\delta}$ with $x \notin X_k$ in the
first paragraph of the proof of~\cite[Lemma 2.16]{dahmanigroves08a}.
Then~\cite[Corollary 4.6]{dahmanigroves08a} says that $X$ satisfies $\ddag_n$; note
that this corollary applies because the conclusion 
of~\cite[Lemma 2.16]{dahmanigroves08a} holds in the case of virtually cyclic
peripheral subgroups.

If $\boundary(\Gamma, \mathcal{H})$ is disconnected then there is no $n$ such
that the condition $\ddag_n$ holds by~\cite[Lemma~4.1]{dahmanigroves08a}.  \end{proof}

\subsection{Cut points and pairs}\label{sec:annulus} 

We now investigate cut points and pairs in $\boundary X$ under the assumption
that $X$ satisfies $\ddag_n$; we assume that this condition holds for the
remainder of this section. We relate the existence of such a point or pair to
the connectedness of thickened cylinders around quasi-geodesics in $X$.

Let $\gamma$ be a bi-infinite $(\lambda, \epsilon)$-quasi-geodesic in $X$ or a
$(\lambda, \epsilon)$-quasi-geodesic ray passing through $X_0$ in $X$.
Fix a point $v$ in $\gamma \intersection X_0$. Since $\gamma$ is a
quasi-geodesic, its limit set, which we shall denote $\Lambda\gamma$, is either
a pair of points $\gamma(\pm\infty)$ or a single point $\gamma(\infty)$
depending on whether $\gamma$ is bi-infinite or a ray.

Recall the definitions $C = 3\delta$ and $D = D(\lambda, \epsilon, \delta)$ as
in Lemma~\ref{lem:uniformlyclose}. 

We define a subset of $X$, the connectivity of which will be seen to
reflect the connectivity of $\boundary X - \Lambda \gamma$.  For $R \geq 0$ let
$N_R(\gamma)$ be the closed $R$-neighbourhood of $\gamma$ and for $0 \leq r
\leq R \leq \infty$ let $N_{r, R}(\gamma)$ be $\{x \in X \colon r \leq d(x,
\gamma) \leq R\}$. For $K \geq 0$ let $C_K(\gamma)$ be $\{x \in X \colon d(x,
\gamma) = K\}$. Finally, for $0 \leq r \leq K \leq R \leq \infty$ let $A_{r, R,
K}(\gamma)$ be the union of those connected components of $N_{r, R}(\gamma)$
that meet $C_K(\gamma)$. Note that $A_{r, R, K}(\gamma)$ contains $N_{K,
R}(\gamma)$.

For a component $U$ of $A_{r, \infty, K}(\gamma)$ define its \emph{shadow}
$\shadow U$ to be the set of points $p$ in $\boundary X$ such that for any
geodesic ray $\alpha$ from $v$ to $p$, $\alpha(t)$ is in $U$ for $t$ sufficiently
large. 

\begin{lem}\label{lem:disjointcoverbyshadows} Let $r > D$. Then
$\bigunion_U\shadow U = \boundary X - \Lambda\gamma$, where the union is
taken over the set of connected components of $A_{r, \infty, K}(\gamma)$.
Furthermore, $\shadow U \intersection \shadow V = \emptyset$ for distinct
components $U$ and $V$.
\end{lem}

\begin{proof} The statement that shadows are disjoint is clear from the
  definition. The assumption that $r > D$ ensures that when $U$ is a subset of
  $A_{r, \infty, K}(\gamma)$, $\shadow U$ does not contain a point in
  $\Lambda\gamma$ by Lemma~\ref{lem:uniformlyclose}. 

  If $p$ is any point in $\boundary X - \Lambda\gamma$ then any geodesic ray
  $\alpha$ from $v$ to $p$ diverges arbitrarily far from $\gamma$, so
  $d(\alpha(t), \gamma) \geq r + D$ for $t$ at least some number $t_0$. Let $U$
  be the component of $A_{r, \infty, K}(\gamma)$ that contains $\alpha(t)$ for
  $t \geq t_0$.
  If $\alpha'$ is another geodesic ray from $v$ to $p$ then for any $t$ there
  exists $t'$ such that $d(\alpha'(t), \alpha(t')) \leq D$ and then $\mod{t -
  t'}$ is guaranteed to be at most $D$. If $t \geq t_0 + D$ then $t' \geq t_0$,
  so $\alpha'(t)$ is in the same component of $A_{r, \infty, K}(\gamma)$ as
  $\alpha(t')$, which is in $U$. Therefore $\alpha'(t) \in U$ for $t \geq t_0 +
  D$. It follows that $p \in \shadow U$.\end{proof}

\begin{lem}\label{lem:nonemptyshadows} Let $U$ be a component of $A_{r, \infty,
K}(\gamma)$. Then $\shadow U$ is non-empty as long as $K \geq r + D + \delta +
C$ and $r > D$.  \end{lem}

\begin{proof} Let $x \in C_K(\gamma) \intersection U$. Then by
Lemma~\ref{lem:closeray} there exists a geodesic ray $\alpha$ from $v$ and $t
\geq 0$ such that $d(\alpha(t), x) \leq C$. Any geodesic segment in $X$ from
$x$ to $\alpha(t)$ is contained in $U$, so $\alpha(t) \in U$. Also
$d(\alpha(t), \gamma) \geq r + D + \delta$; it follows from
Lemma~\ref{lem:uniformlyclose}, hyperbolicity of $X$ and the assumption that
$r > D$ that $d(\alpha(t'), \gamma) \geq r$ for $t' \geq t$, and therefore
that $\alpha(t') \in U$ for $t' \geq t$. As in the proof of
Lemma~\ref{lem:disjointcoverbyshadows} it follows from this and the fact that
$r > D$ that $\alpha(\infty) \in \shadow U$.\end{proof}

\begin{lem}\label{lem:shadowsclopen} If $U$ is a component of $A_{r, \infty,
K}(\gamma)$, $\shadow U$ is closed and open in $\boundary X - \Lambda\gamma$ as
long as $r > D$. \end{lem}

\begin{proof} Let $p$ be a point in $\shadow U$ and let $p = \alpha(\infty)$
where $\alpha$ is a geodesic ray from $v$. For $t \geq 0$ let
$V_{t}(\alpha)$ be the set of end points of geodesic rays $\beta$ from $v$ such that
$d(\beta(t), \alpha(t)) < 2\delta + 1$. The collection of such sets as $t$
varies forms a fundamental system of neighbourhoods of $p \in \boundary X$. 

Then there exists $t_0$ such that for $t \geq t_0$, $d(\alpha(t), \gamma) \geq
r + 2D + 7\delta + 1$.  We claim that $V_{t_0}(\alpha) \subset \shadow U$
for $t_0$ as defined in the previous paragraph. To see this, let $q \in
V_{t_0}(\alpha)$ and let $\beta$ be a geodesic ray from $v$ to $q$, so
$d(\beta(t_0), \alpha(t_0)) < 2\delta+1$.  Let $\beta'$ be another geodesic ray
from $v$ with $\beta'(\infty) = q$, so $d(\beta'(t_0), \beta(t_0)) < 4\delta$.
Suppose that there exists $t \geq t_0$ such that $\beta'(t) \notin U$. Then
there exists $t' \geq t_0$ such that $d(\beta'(t'), \gamma) \leq r$; without
loss of generality assume that $d(\beta'(t'), \gamma\restricted{[0, \infty)})
\leq r$. Let $\gamma'$ be a geodesic ray from $v$ to $\gamma(\infty)$, so the
Hausdorff between $\gamma\restricted{[0, \infty)}$ and $\gamma'$ is at most
$D$. Then $d(\beta'(t'), \gamma') \leq r + D$ and $d(\beta'(t_0), \gamma')
\leq r + D + \delta$. Putting these inequalities together, $d(\alpha(t_0),
\gamma) \leq r + 2D + 7\delta + 1$, which is a contradiction. Hence
$\beta(t) \in U$ for $t \geq t_0$, and so $q \in \shadow U$.

Since $p$ was arbitrary in $\shadow U$, it follows that $\shadow U$ is open. As
$U$ ranges over the connected components of $A_{r, \infty, K}(\gamma)$,
$\shadow U$ ranges over a cover of $\boundary X - \Lambda\gamma$ by disjoint
open subsets (since $r > D$), so each is also closed.\end{proof}

The proof of the following lemma is based on the proof 
of~\cite[Proposition 3.2]{bestvinamess91}.

\begin{lem}\label{lem:shadowsconnected} If $U$ is a connected component of 
$A_{r, \infty, K}(\gamma)$ then $\shadow U$ is contained in one connected
component of $\boundary X - \Lambda \gamma$ as long as $r$ satisfies the
following inequality.
\begin{align*}
  r > 2\log_a\left(\frac{k_2}{k_1}\frac{n-1}{1-a^{-1}}\right) + M + 12\delta + D.
\end{align*}
\end{lem}

\begin{proof} Let $p$ and $q$ be points in $\shadow U$ and let $\alpha_1$ and
$\alpha_2$ be geodesic rays from $v$ to $p$ and $q$ respectively. Then there
exist $t_1$ and $t_2$ such that $\alpha_1(t_1)$ and $\alpha_2(t_2)$ are in
$U$. Let $\phi\colon [0,\ell] \to X$ be a path in $U$ parametrised by arc
length connecting the two points $\alpha_1(t_1)$ and $\alpha_2(t_2)$.

For each integer $i$ in $[0,\ell]$ let $z_i$ be a point within a distance $C$
of $\phi(i)$ so that there is a geodesic ray $\beta_i$ from $v$ with
$\beta_i(m_i) = z_i$; we can assume that $\beta_0 = \alpha_1$ and $\beta_\ell =
\alpha_2$.  Then, following the argument of~\cite[Prop.\ 3.2]{bestvinamess91},
we show that $r_i(\infty)$ and $r_{i+1}(\infty)$ can be connected in $\boundary
X - \Lambda\gamma$ for each $i$; for notational convenience we prove it for
$i=0$.

Using the condition $\ddag_n$, define geodesic rays $\beta_t$ for each
$n$-adic rational $t$ in $[0, 1]$ inductively on the power $k$ of the
denominator of $t$ to satisfy:
\begin{align*} 
  d\left(\beta_{j/n^k}(m_i + k), \beta_{j+1/n^k}(m_i + k)\right) \leq M 
  \text{ for each $j$ with $0 \leq j \leq n^k-1$}.
\end{align*}
Note that the first step of the induction holds since $M$ is at least $2C+1$.
The triangle inequality gives the following lower bound on the Gromov product
of these points.
\begin{align*}
  \gromprod[v]{\beta_{j/n^k}(\infty)}{\beta_{j+1/n^k}(\infty)} & \geq
\liminf_{n_1,n_2}\gromprod[v]{\beta_{j/n^k}(n_1)}{\beta_{j+1/n^k}(n_2)}\\
& \geq \gromprod[v]{\beta_{j/n^k}(m_0 + k)}{\beta_{j+1/n^k}(m_0 + k)}\\
& = m_0 + k - M/2
\end{align*}
Let $d_v$ be a visual metric on $\boundary X$ with base point $v$, visual
parameter $a$ and multiplicative constants $k_1$ and $k_2$. We obtain:
\begin{align*}
  d_v\left(\beta_{j/n^k}(\infty), \beta_{j+1/n^k}(\infty)\right) \leq k_2a^{-m_0 - k + M/2}.\tag{*}
\end{align*}

Inductively applying the triangle inequality we arrive at the following inequality
\begin{align*}
  d_v\left(\beta_0(\infty), \beta_t(\infty)\right) \leq \frac{k_2(n-1)a^{-m_0 +
  M/2}}{1-a^{-1}}\quad\text{for each $n$-adic rational $t \in [0, 1]$.}
\end{align*}
Define a path $\psi \colon [0,1] \to \boundary X$ with $\psi(t) =
\beta_t(\infty)$ for each $n$-adic rational $t$ in $[0, 1]$; this extends
continuously to a path from $\beta_0(\infty)$ to $\beta_1(\infty)$ by the
uniform continuity of the map $t \to \beta_t(\gamma)$ defined on the $n$-adic
rationals, which is established by equation~$(*)$. This path is
contained in the ball of radius $k_2(n-1)a^{-m_0 + M/2}/(1-a^{-1})$ around
$\beta_0(\infty)$.

We now bound below the distance $d_v(\beta_0(\infty), \Lambda\gamma)$. Let
$\gamma'$ be a geodesic ray from $v$ to $\gamma(\infty)$, so the Hausdorff
distance between $\gamma$ and $\gamma'$ is at most $D$.
By~\cite[III.H.3.17]{bridsonhaefliger99},
\begin{align*}
  \gromprod[v]{\beta_0(\infty)}{\gamma'(\infty)} \leq
  \liminf_{n_1,n_2}\gromprod[v]{\beta_0(n_1)}{\gamma'(n_2)} + 2\delta.
\end{align*}
Let $n_1$ and $n_2$ each be at least $m_0$. Certainly $d(\beta_0(m_0), \gamma') >
\delta$ since $r > \delta + D$, so there exists a point $p$ on $[\beta_0(n_1),
\gamma'(n_2)]$ within a distance $\delta$ of $\beta_0(m_0)$. In fact, $d(\beta_0(m_0),
\gamma) > 2\delta + D$, so $d(\beta_0, \gamma'(m_0)) > 2\delta$. Therefore there
exists a point $q$ on $[\beta_0(n_1), \gamma'(n_2)]$ within a distance $\delta$ of
$\gamma'(m_0)$. 

Suppose that $q$ is closer to $\beta_0(n_1)$ than $p$. Then by considering the
geodesic triangle with vertices $\beta_0(n_1)$, $\beta_0(m_0)$ and $p$ we see
that $q$ is within distance $2\delta$ of $\beta_0$, and therefore the distance
from $\gamma'(m_0)$ to $\beta_0(m_0)$ is at most $6\delta$. But we assumed that
$r > 6\delta + D$, which gives a contradiction. This implies that
$d(\beta_0(n_1), \gamma'(n_2))$ is equal to the sum of the distances
$d(\beta_0(n_1), p)$, $d(p, q)$ and $d(q, \gamma'(n_2))$.

Then we have the following inequality.
\begin{align*} \gromprod[v]{\beta_0(n_1)}{\gamma'(n_2)} - \gromprod[v]{\beta_0(m_0)
    }{\gamma'(m_0)} 
    & = d(\beta_0(n_1), \beta_0(m_0)) - d(\beta_0(n_1), p) \\
    & + d(\beta_0(m_0), \gamma'(m_0)) - d(p, q) \\
    & + d(\gamma'(m_0), \gamma'(n_2)) - d(q, \gamma'(n_2))\\
    & \leq \delta + 2\delta + \delta = 4\delta
\end{align*}

This implies a lower bound on the distance from $\beta_0(\infty)$ to
$\gamma(\infty)$ with respect to the visual metric.
\begin{align*}
  d_v(\beta_0(\infty), \gamma(\infty)) \geq k_1a^{-m_0 + (r-D)/2 - 6\delta}.
\end{align*}
In the case that $\gamma$ is bi-infinite, $d_v(\beta_0(\infty),
\gamma(-\infty))$ similarly satisfies the same bound.  Therefore, by the
assumption on $r$ the path constructed from $\beta_0(\infty)$ to
$\beta_1(\infty)$ avoids $\Lambda\gamma$.
\end{proof}

\begin{lem}\label{lem:replaceinftybyR} The inclusion map $A_{r, R, K}(\gamma)
\hookrightarrow A_{r, \infty, K}(\gamma)$ induces a bijection between the sets
of connected components of those subspaces of $X$ as long as $R \geq 4\delta + D +
\max\{r + 4\delta + 1, K\}$.\end{lem}

\begin{proof} Surjectivity is clearly guaranteed by the fact that $R \geq K$.
For injectivity, let $x$ and $y$ be points in $C_K(\gamma)$ that lie in the
same connected component of $A_{r, \infty, K}(\gamma)$. We show that the
shortest path from $x$ to $y$ in $N_{r, \infty}(\gamma)$ stays within a
distance $R$ of $\gamma$. Let $\phi\colon [0,\ell] \to X$ be such a
shortest path parametrised by arc length. Suppose that $d(\phi(s),\gamma) >
R$. Let $[t_0,t_1]$ be a maximal subinterval of $[0, \ell]$ containing $s$
such that $d(\phi(t),\gamma) \geq r + 4\delta + 1$ for $t \in [t_0,t_1]$. 

Then for $t \in [t_0,t_1]$, $\phi\restricted{[t-4\delta-1,t+4\delta+1]
\intersection[t_0, t_1]}$ has image in $N_{r+4\delta+1, \infty}(\gamma)$.
Therefore any geodesic segment from $\phi(\min\{t-4\delta-1, t_0\})$ to
$\phi(\max\{t+4\delta+1, t_1\})$ is contained in $N_{r, \infty}(\gamma)$, so by
minimality of the length of $\phi$, $\phi\restricted{[t-4\delta-1,t+4\delta+1]
\intersection [t_0, t_1]}$ is a geodesic. This means that
$\phi\restricted{[t_0,t_1]}$ is an $(8\delta+2)$-local geodesic. Therefore by
\cite[III.H.1.13]{bridsonhaefliger99} it is contained in a
$2\delta$-neighbourhood of any geodesic from $\phi(t_0)$ to $\phi(t_1)$.

By maximality of $[t_0,t_1]$, either $d(\phi(t_0),\gamma) = r + 4\delta+1$ or
$t_0=0$, so certainly $d(\phi(t_0),\gamma) \leq \max\{r+4\delta+1,K\}$, and
similarly $d(\phi(t_1),\gamma)$ satisfies the same inequality. By
$\delta$-hyperbolicity applied to the geodesic quadrilateral with vertices
$\phi(t_0)$, $\phi(t_1)$ and the points $\gamma(s_0)$ and $\gamma(s_1)$ on
$\gamma$ minimising the distances to $\phi(t_0)$ and $\phi(t_1)$, any geodesic
from $\phi(t_0)$ to $\phi(t_1)$ is contained in a $2\delta + \max\{r + 4\delta
+ 1, K\}$ neighbourhood of a geodesic from $\gamma(s_0)$ to $\gamma(s_1)$, so
is a subset of $N_{2\delta+\max\{r+4\delta+1, K\} + D}(\gamma)$. Hence
$d(\phi(s),\gamma) \leq 4\delta + \max\{r+8\delta+2,K\} + D$, which is a
contradiction.  \end{proof}

From the results of this section we conclude the following:

\begin{prop}\label{prop:summaryofannulusresults} The map that sends a component
$U$ of $A_{r, R, K}(\gamma)$ to the shadow (with respect to some base point
$v \in \gamma$) of the component of $A_{r, \infty, K}(\gamma)$ containing $U$
is a well defined bijection between the set of connected components of $A_{r,
R, K}(\gamma)$ and the set of connected components of $\boundary X -
\Lambda\gamma$ as long as $r$, $R$ and $K$ are taken to simultaneously
satisfy the conditions of lemmas~\ref{lem:disjointcoverbyshadows},
\ref{lem:nonemptyshadows}, \ref{lem:shadowsclopen}, \ref{lem:shadowsconnected}
and~\ref{lem:replaceinftybyR}.\qed\end{prop}

\begin{rem}\label{rem:rKRcomputable} The conditions on $r$, $R$ and $K$ depend
only on $\delta$, $n$, $\lambda$ and $\epsilon$ and suitable values can be
computed from these data.\end{rem}

We end this section with the following lemma, which shows that connectedness of
$A_{r, R, K}(\gamma)$ can be detected locally.

\begin{lem}\label{lem:Adisconnectedlocally} Suppose that $\gamma$ is a
bi-infinite $(\lambda, \epsilon)$-quasi-geodesic and that neither point in
$\Lambda\gamma$ is a cut point. Let $r$ and $R$ be chosen to satisfy
lemmas~\ref{lem:shadowsconnected} and~\ref{lem:replaceinftybyR}. Let $T$ be
at least $\log_a(2k_1/k_2) + 3D + 2\delta + K$. Then every component of $A_{r, R,
K}(\gamma)$ meets $C_K(\gamma) \intersection \Ball_T(\gamma(t))$ for any $t$
such that $\gamma(t)$ is in $X_0$.  \end{lem}

\begin{proof} Fix the base point $v = \gamma(t)$.
Let $U$ be a component of $A_{r, R, K}(\gamma)$ and let $U'$ be the component
of $A_{r, \infty, K}(\gamma)$ containing $U$.
Then it is sufficient to show that $U'$ meets $C_K(\gamma) \intersection
\Ball_T(v)$ since $U' \intersection C_K(\gamma) = U \intersection C_K(\gamma)$
by Lemma~\ref{lem:replaceinftybyR}. 

Suppose that $U'$ does does not meet $C_K(\gamma) \intersection \Ball_T(v)$.
Let $p$ be a point in $\shadow U$ and let $\alpha$ be a geodesic ray from $v$
to $p$.
Then $\alpha(s) \in U \intersection C_K(\gamma)$ for some $s$; by assumption
$d(\alpha(s), v) \geq T$.

Let $\gamma'$ be a geodesic connecting the points of $\Lambda\gamma$ so that
the Hausdorff distance between $\gamma$ and $\gamma'$ is at most $D$.
Parametrise $\gamma'$ so that $d(\gamma'(0), v) \leq D$.
Then $d(\alpha(s), \gamma') \leq K + D$; let $d(\alpha(s), \gamma'(s')) \leq K
+ D$.
This implies that $d(\gamma'(s'), v)\geq T - D - K$.
We therefore have
\begin{align*}
  \gromprod[v]{\alpha(s)}{\gamma'(s')} \geq T - D - K.
\end{align*}
Assume that $s'\geq0$; this implies that
\begin{align*}
  \gromprod[v]{p}{\gamma'(\infty)} & \geq \liminf_{m, n \to \infty}\gromprod[v]{\alpha(m)}{\gamma'(n)} \\
                                   & \geq \gromprod[v]{\alpha(s)}{\gamma'(s')} - D \\
                                   & \geq T - 2D - K.
\end{align*}
So $d_v(p, \gamma'(\infty)) \leq k_2a^{-T + 2D + K}$. Similarly, if $s' \leq 0$, $d_v(p,
\gamma'(-\infty)) \leq k_2a^{-T + 2D + K}$. Therefore $\shadow U'$
is contained in a $k_2a^{-T + 2D + K}$ neighbourhood of $\Lambda\gamma$.  Also,
for any $s$, the geodesic from $\gamma(t+s)$ to $\gamma(t-s)$ passes within a
distance $D$ of $\gamma(t)$, so $\gromprod[v]{\gamma(t+s)}{\gamma(t-s)} \leq
D$. Then $\gromprod[v]{\gamma(\infty)}{\gamma(-\infty)} \leq 2\delta + D$, and
so $d_v(\gamma(\infty), \gamma(-\infty)) \geq k_1a^{-(2\delta + D)}$. It follows by
the inequality satisfied by $T$ that the closed balls of radius $k_2a^{-T + 2D
+ K}$ around $\gamma(\infty)$ and $\gamma(-\infty)$ are disjoint. By
Lemma~\ref{lem:shadowsconnected} $\shadow U'$ is connected, so is 
contained in one of these two balls, say in the ball around $\gamma(\infty)$.
But then $\shadow U$ is a non-empty proper subset of $\boundary X -
\{\gamma(\infty)\}$ that is closed and open, so $\gamma(\infty)$ is a cut
point, which is a contradiction.\end{proof}

